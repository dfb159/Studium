\clearpage
\section{Schlussfolgerung}
	
Die Ziele dieser Untersuchung des Stirling-Motors wurden weitgehend erreicht.
Bei den Ergebnissen handelt es sich um physikalisch sinnvolle: die Reibungskraft entspricht \SI{3.0+-0.5}{\joule} pro Umdrehung, die Kühlleistung $\varepsilon = 0,0233\pm 0,0015$ und die Heizleistung $\varepsilon = 0,19\pm 0,05$.
Zudem liegt der ermittelte Wert für die spezifische Wärme von Eis exakt auf dem Literaturwert.
Abgesehen von der Schmelzwärme, welche ca. 35,4\% von dem Literatur abweicht, ist die Untersuchung erfolgreich.
Eine Wiederholung des ganzen Versuches ist demnach nicht nötig, eine erneute Betrachtung der Schmelzwärme könnte sich jedoch als sinnvoll erweisen.