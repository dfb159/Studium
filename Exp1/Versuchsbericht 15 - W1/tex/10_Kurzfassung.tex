\section{Kurzfassung}

Dieser Bericht beschäftigt sich mit der Untersuchung des Stirling-Motors.
Bei diesem handelt es sich um eine der ersten Wärmekraftmaschinen.
In Abhängigkeit der Drehrichtung des Motors, lässt sich dieser flexibel als Kältemaschine oder auch Wärmepumpe verwenden.
Wie auch bei vielen mechanischen Systemen spielt Reibung eine Rolle.
Die dadurch verrichtete Reibungsarbeit wird bei dieser Untersuchung über die Erwärmung des Kühlwassers ermittelt.
Dazu wird ein Milliliter destilliertes Wassers in das System gegeben und die Temperatur dessen für Kältemaschine und Wärmepumpe aufgezeichnet.
Daraus werden Heiz- und Kühlleistung des Systems bestimmt.
Neben den Leistungen lassen sich auch die Schmelzwärme des Wassers und die spezifische Wärme von Eis bestimmen.
Ziel der Untersuchung ist eine Übereinstimmung dieser Schmelz- und spezifischen Wärme mit der Literatur, wie auch physikalisch sinnvolle Ergebnisse für die Reibungsarbeit, sowie der Heiz- und Kühlleistung.

Die Ergebnisse stimmen im Wesentlichen mit den Zielen dieser Untersuchung überein.
Bei der Reibungsarbeit ergab sich \SI{3.0+-0.5}{\joule} pro Umdrehung, bei $\varepsilon = 0,0233\pm 0,0015$ und $\varepsilon = 0,19\pm 0,05$.
Diese Werte liegen alle in physikalisch sinnvollen Größenordnungen.
Das Ergebnis für die spezifische Wärme von Eis liegt mit \SI{2.108+-0.010}{\joule\per\gram\per\kelvin} exakt auf dem Literaturwert.
Somit ist das einzige Ergebnis dieser Untersuchung welches von der Erwartung abweicht die Schmelzwärme.
Mit \SI{215.8+-1.2}{\joule\per\gram} weicht diese um ca. 35,4\% von dem Literaturwert ab.
