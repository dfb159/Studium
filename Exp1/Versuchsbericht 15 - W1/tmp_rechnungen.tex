% Autor: Alex Oster, Jonathan Sigrist
% Datum: 2018-01
% basiert auf der Vorlage für Versuchsprotokolle von Simon May

\documentclass[a4paper]{scrartcl}

% Mathepaket (intlimits: Grenzen über/unter Integralzeichen)
\usepackage[intlimits]{amsmath}
% Mathe-Symbole, \mathbb etc.
\usepackage{amssymb}
% Weitere Mathebefehle
\usepackage{mathtools}
% „Schöne“ Brüche im Fließtext
\usepackage{xfrac}
% Ermöglicht die Nutzung von \SI{Zahl}{Einheit} u.a.
\usepackage{siunitx}
% Ermöglicht Nutzung von \pdv als Ableitungen
\usepackage{physics}
% Definition von Unicode-Symbolen; Nach [utf8]inputenc laden!
\usepackage{newunicodechar}
% Unicode-Formeln mit pdfLaTeX
\input{tex/99_pdflatex_unicode-math.tex}

% allgemein ist dW = (W pro sekunde) proportinal zu W = (W pro umdrehung) gesehen werden
% im folgenden soll stets gelten $\rho \dot{=} \rho_\text{H2O}$ und $ c = c_\text{H2O}$
\begin{document}
	\begin{equation} % massenstrom des Kühlwassers [masse pro zeit]
		\dot{M} = \rho \frac{V_\text{MB}}{t_\text{MB}}
	\end{equation} % rho ist dichte, V ist volumen des Messbechers, t ist Füllzeit des volumens
	
	\begin{equation} % Reibungswaerme pro Zeit(ist negativ, weil Reibungsarbeit geleistet wird)
		\dot{Q}_\text{R} = \dot{W}_\text{R} = c \dot{M} (T_\text{Zu} - T_\text{Ab})
	\end{equation} % c ist spez. Wärme (von wasser), M wie oben, T_zu ist zulaufende Temperatur, T_ab ist ablaufende Temperatur
	
	\begin{equation} % analog dazu Wärme von kalt, warm machen
	\dot{Q}_1 = c \dot{M} (T_\text{Zu} - T_\text{Ab})
	\end{equation} % T_zu ist zulaufende Temperatur, T_ab ist ablaufende Temperatur bei kalt/warm machen
	
	\begin{equation} % Wärmeänderung der Probe pro Zeit
		-\dot{Q}_2 = c \rho V_\text{P} \dd T
	\end{equation} % V_P = Volumen der Probe; dT Steigung des Graphen bei Raumtemp
	
	\begin{equation} % absolute Leistung der Maschine, ist dW < 0 so wird Energie aufgenommen, ist dW > 0 so wird Energie abgegeben/Arbeit von der Maschine geleistet; da wir hier Wärmepumpe/Kältemaschine haben ist dW < 0 und der Motor gibt Energie in das System ein
		\dot{W} = \dot{Q}_1 - \dot{Q}_2 + \dot{W}_\text{R}
	\end{equation} % Q_1 ist Wärme des Kühlwassers(negativ)[siehe (3)], Q_2 ist Wärme der Probe/gemessene Temperatur (4) -> Wärmeänderung(Kältemaschine: positiv, Wärmepumpe: negativ)
	
	\begin{equation} % kühlleistung/heizleistung
		\varepsilon = \frac{\abs{Q_2}}{\abs{W}}
	\end{equation} % Q_2 (4), W (5)
	
	\begin{equation} % Wärme beim Kristallisieren des Eises
		Q = c \rho V_\text{P} (T_2 - T_1)
	\end{equation} % T_2: Temperatur kurz nach dem Gefrierens; T_1: Temperatur kurz vor dem Gefrieren
	
	\begin{equation} % Schmelzwärme gefrieren
		Q_\text{G} = Q_2 (t_2 - t_1) + Q
	\end{equation} % t_1 Anfangszeitpunkt des Gefrierens nach kristallisieren(Stufe); t_2: Endzeitpunkt (durch linearisierung des theoretisch homogenen kühlprozesses bestimmt); Q von vorher(7); Q_2 beim gefrieren
	
	\begin{equation} % Schmelzwärme schmelzen (Q_S = Q_G)
		Q_\text{S} = -Q_2 (t_2 - t_1)
	\end{equation} % das gleiche nur ohne Q, weil keine stufe; *-1 weil Energie reingesteckt wird
	
	\begin{align}
		\dot{Q} &= c_\text{Eis} m_\text{P} \dd T\\
		\Leftrightarrow c_\text{Eis} &= \frac{\dot{Q}}{m_\text{P}\dd T}
	\end{align} % c_Eis spez. Wärme Eis; m_P masse Probe; dQ wie oben
	
	\begin{equation}
		m_\text{P} = \rho V_{P}
	\end{equation} % die masse der probe ändert sich nicht, auch wenn es zu eis wird
\end{document} 