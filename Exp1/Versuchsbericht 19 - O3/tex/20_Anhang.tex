\subsection{Unsicherheiten}\label{VGuD}

Jegliche Unsicherheiten werden nach GUM bestimmt und berechnet.
Die Gleichungen dazu finden sich in \ref{fig:GUM_combine} und \ref{fig:GUM_formula}.
Für die Unsicherheitsrechnungen wurde die Python Bibliothek "uncertainties" herangezogen, welche den Richtlinien des GUM folgt.
Alle konkreten Unsicherheitsformeln stehen weiter unten.
Für Unsicherheiten in graphischen Fits wurden die $y$-Unsicherheiten beachtet und die Methode der kleinsten Quadrate angewandt.
Dafür steht in der Bibliothek die Methode "scipy.optimize.curve\_fit()" zur Verfügung.

Für digitale Messungen wird eine Unsicherheit von $u(X) = \frac{\Delta X}{2\sqrt{3}}$ angenommen, bei analogen eine von $u(X) = \frac{\Delta X}{2\sqrt{6}}$.

\begin{description}
	\item[Spannung]	Es wird einheitlich im Bereich bis \SI{20}{\volt} gemessen.
	Das Voltmeter hatte eine digitale Anzeigegenauigkeit von $\Delta U = \SI{0.01}{\volt}$.
	
	\item[Polariseur] Beide Polariseure hatten angegebene analoge Gradstriche mit $\Delta \phi = \SI{1}{\degree}$.
	
	\item[Standbein] Das Standbein auf welchem der Spiegel gedreht wurde, hatte ebenfalls eine Schrittweite von $\Delta \Theta = \SI{1}{\degree}$ und konnte analog abgelesen werden.
	
	\item[$\lambda/2$-Platte] Die Winkelmarkierungen auf der $\lambda/2$-Platte waren in einem Abstand von $\Delta \varphi_{\lambda/2} = \SI{2}{\degree}$ analog abzulesen.
	
	\item[Brewster-Winkel] Der Reflektionswinkel mit minimalem parallelen Anteil wurde aus der Abb, \ref{fig:bragg} abgelesen.
	Dabei kann der Wert auf ein Intervall der Breite $\Delta \varphi_\text{B} = \SI{10}{\degree}$ abgeschätzt werden.
	Anhand der theoretischen Vorhersage nach Abb. \ref{fig:wiki_brewster} kann er als analog angesehen werden.
	
\end{description}

\begin{figure}[ht]
	\begin{equation*}
		x = \sum_{i=1}^{N} x_i
		;\quad
		u(x) = \sqrt{\sum_{i = 1}^{N} u(x_i)^2}
	\end{equation*}
	\caption{Formel für kombinierte Unsicherheiten des selben Typs nach GUM.}
	\label{fig:GUM_combine}
\end{figure}

\begin{figure}[ht]
	\begin{align*}
		f = f(x_1, \dots , x_N)
		;\quad
		u(f) = \sqrt{\sum_{i = 1}^{N}\left(\pdv{f}{x_i} u(x_i)\right) ^2}
	\end{align*}
	\caption{Formel für sich fortpflanzende Unsicherheiten nach GUM.}
	\label{fig:GUM_formula}
\end{figure}

\begin{figure}[ht]
	\begin{align*}
		n = \arctan(\varphi_\text{B})
		;\quad
		u(n) = \frac{u(\varphi_\text{B})}{1 + \varphi_\text{B}^2}
	\end{align*}
	\caption{Unsicherheitsformel für den Brechungsindex des Glases.}
	\label{unc:n_glas}
\end{figure}

\begin{figure}[ht]
	\begin{align*}
		\rho = \frac{U - b}{m}
		;\quad
		u(\rho) = \rho \sqrt{\frac{u^2(U) + u^2(b)}{(U - b)^2} + \frac{u^2(m)}{m^2}}
	\end{align*}
	\caption{Unsicherheitsformel für die Konzentration der unbekannten Zuckerlösung. $m, b$ sind dabei die Fitparameter der Geraden aus Abb. \ref{fig:konzentration}.}
	\label{unc:U_konzentration}
\end{figure}

