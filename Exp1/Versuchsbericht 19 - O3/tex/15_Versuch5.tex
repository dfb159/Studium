\subsection{Teilversuch 5: Kristallpolarisation an einem Kalkspat}

	\subsubsection*{Methoden}
		
		\begin{figure}[ht]
			\centering
			\includegraphics[width=\textwidth]{bilder/Kalkspat.png}
			\caption{Aufbau des letzten Teilversuches\cite{WWU}}
			\label{fig:Kalkspat}	
		\end{figure}
		Abb. \ref{fig:Kalkspat} stellt den Aufbau für den letzten Teilversuch dar.
		Hier werden zwischen die Polarisatoren das $\lambda/2$-Plättchen und der Kalkspat gesetzt.
		Die Photodiode soll bei diesem Versuch nicht verwendet werden.
		Stattdessen soll ein Schirm hinter dem zweiten Polarisator stehen, auf dem die Intensität des Laserlichts beobachtet und diskutiert werden soll.
		
	\subsubsection*{Durchführung}
	
		Nach dem Einsetzen der Versuchsmaterialien und folgender Inbetriebnahme des Lasers ließen sich vier Lichtpunkte auf dem Schirm ausmachen.
		Drehen des $\lambda/2$-Plättchens führte dazu, dass jeweils zwei der Lichtpunkte verschwanden, nach weiteren \SI{45+-0,4}{\degree} wieder auftauchten und dafür die anderen beiden an Intensität verloren.
		Bei dem Drehen des zweiten Polarisators wurde das gleiche, nur mit anderen Paaren der Punkte, bei \SI{90+-0,4}{\degree} beobachtet.
		Waren es bei dem $\lambda/2$-Plättchens die Punktpaare (1,2) und (3,4), so waren es beim dem Polarisastor die Paare (1,4) und (2,3).
		Für den Fall, dass beide so gedreht waren, dass je ein Paar verschwindet, war nur einer der vier Lichtpunkte sichtbar.
		
	\subsubsection*{Diskussion}
	
		Die Beobachteten Verhältnisse lassen sich auf die Polarisation an Kristallen zurückführen.
		Bei dem Strahlengang durch den Kristall, wird der Strahl in zwei Teile geteilt.
		Einerseits in den "ordentlichen" Strahl, welcher sich gemäß dem Snellius'schen Brechungsgesetz verhält und mit dem Brechungsindex $n_1$ in dem Kristall gebrochen wird, sowie andererseits dem "außerordentlichen" Strahl, welcher um einen anderen Winkel durch den Brechungsindex $n_2$ gebrochen wird.
		Die Strahlen sind nach austritt aus dem Kristall orthogonal zueinander polarisiert.
		Im Falle zweier linear polarisierter Strahlen, die orthogonal aufeinander liegen, redet man von zirkulärer Polarisation, ansonsten von elliptischer.
		Trifft Licht mit senkrecht und parallel polarisierten Komponenten auf den Kristall, so werden beide jeweils in einen ordentlichen und einen außerordentliche Strahl gebrochen.
		Daher lassen sich vier Lichtpunkte auf dem Schirm erkennen.
		Wird das $\lambda/2$-Plättchen so gedreht, dass nur der parallele bzw. senkrechte Anteil durchgelassen wird, so wird auch nur einmal in zwei Strahlen geteilt, weswegen nur zwei Lichtflecke dann auf dem Schirm erkennbar sind.
		Dass nach \SI{45+-0,4}{\degree} die orthogonale Polarisation auftritt ist dem zweiten Teilversuch zu entnehmen.
		Analog mit dem zweiten Polarisator nach \SI{90+-0,4}{\degree}.
		Für den Fall, dass nach dem Drehen des $\lambda/2$-Plättchens nur noch zwei Lichtpunkte zu erkennen sind, lässt sich durch Drehen des zweiten Polarisators einer der beiden auslöschen.
		Auch hier wird dann nur der parallele bzw. senkrechte polarisierte Anteil des Lichtstrahls hinter dem Kristall durchgelassen.
		Somit lässt sich dieses Phänomen erklären.