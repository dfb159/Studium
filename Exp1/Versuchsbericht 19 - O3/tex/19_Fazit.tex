\section{Schlussfolgerung}

	Die Ziele dieser Untersuchung eine Übereinstimmung der gemessenen und ermittelten Größen mit der Theorie, die den einzelnen Teilversuchen zu Grunde liegt, zu finden, wurde im Wesentlichen erreicht.
	Das Gesetz von Malus ließ sich durch die Polarisatoren bestätigen, wie auch das Drehen der Polarisationsebene um den zweifachen Winkel um den das $\lambda/2$-Plättchen gedreht wurde.
	Ebenso, wenn auch ungenauer ließ sich die Brewster-Beziehung bei der Glasplatte begründen und mit einem Brechungsindex von $n = \SI{1,5508+-0,0008}{}$ ist der Wert für Glas als Material sinnvoll.
	Für den Strahlenversatz in Abhängigkeit der Konzentration ließ sich das lineare Verhältnis zeigen und die Konzentration von Zucker in einer Röhre ohne Angabe ließ sich dadurch bestimmen.
	Auch die unterschiedlichen Beugungseffekte bei dem Kalkspatkristall ließen sich anhand der Theorie erklären. 
	
	Abschließend lässt sich behaupten, dass die Untersuchung erfolgreich verlief und eine Wiederholung daher nicht nötig ist.
	Zur Erweiterung würde es sich anbieten auch $\lambda/4$-Plättchen und andere Kristalle zu untersuchen. 
	Dies würde jedoch den Rahmen einer experimentellen Übung vermutlich sprengen.
	Desweiteren kann der Brewster-Winkel durch kleinere Schrittweiten um das Minimum genauer bestimmt werden.
	