\section{Kurzfassung}

	Dieser Bericht beschäftigt sich mit der Untersuchung von Polarisation bzw. genauer mit der Wechselwirkung von polarisiertem Licht mit Materie.
	Dazu werden in fünf Teilversuchen verschiedene Phänomene der Polarisation betrachtet.
	Zu diesen gehören lineare Polarisierung anhand von Polarisatioren, der Drehung der Polarisationsebene durch ein $\lambda/2$-Plättchen, dem Reflexionsvermögen einer Glasplatte und dem Sonderfall des Brewster-Winkel, dem Strahlenversatz innerhalb Zuckerlösungen unterschiedlicher Konzentration, sowie Polarisation in Kristallen.
	
	Ziel dieser Untersuchung ist im Wesentlichen die Übereinstimmung der gemessenen und ermittelten Größen mit der Theorie, die den einzelnen Teilversuchen zu Grunde liegt.
	Das Gesetz von Malus ließ sich durch die Polarisatoren bestätigen, wie auch das Drehen der Polarisationsebene um den zweifachen (\SI{1,998+-0,006}{}-fachen) Winkel um den das $\lambda/2$-Plättchen gedreht wurde.
	Ebenso, wenn auch ungenauer ließ sich die Brewster-Beziehung bei der Glasplatte begründen und mit einem Brechungsindex von $n = \SI{1,5508+-0,0008}{}$ ließ sich die Platte allgemein zu Glas als Material zuordnen.
	Für den Strahlenversatz in Abhängigkeit der Konzentration ließ sich das lineare Verhältnis zeigen und die Konzentration von Zucker in einer Röhre ohne Angabe ließ sich dadurch bestimmen.
	Auch bei dem letzten Teilversuch ließen sich die Beobachtungen anhand der Theorie erklären. 
	