
\section{Kurzfassung}

Dieser Bericht befasst sich mit der Nachstellung des Versuches zur Bestimmung der Elementarladung nach Millikan, wofür er im Jahre 1923 einen Nobelpreis erhalten hat.
Dazu werden Öltröpfchen im elektrischen Feld eines Plattenkondensators mit Hilfe eines Mikroskops betrachtet.
Bei der Betrachtung wird die Geschwindigkeit der Tröpfchen über die Zeit für bestimmte Wegstücke ermittelt. 
Mit diesen Geschwindigkeiten und über die Kräfteverhältnisse, die bei an- bzw. ausgeschaltetem Plattenkondensator gelten, lässt sich auf die Ladung der Öltröpfchen schließen.

Das Ziel der Nachstellung des Versuches ist eine Übereinstimmung mit den Ergebnissen von Millikan.
Da hierfür die Ladung der Öltröpfchen aus ganzzahligen Vielfachen der Elementarladung $e$ bestehen sollte, lässt sich aufgrund der Ergebnisse dieses Versuches von u. A. % TODO
	(keine) eindeutige Übereinstimmung mit den nach Millikan zu erwartenden Werten finden.