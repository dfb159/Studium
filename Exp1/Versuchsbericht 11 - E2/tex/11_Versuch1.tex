\section{Bestimmung der Elementarladung nach Millikan} 

% TODO

\subsection{Vorbereitende Überlegungen}

Zur Vorbereitung dieses Versuches sollte eine Reihe von Aufgaben durchgeführt werden. 
Die Bearbeitung dieser ist im folgenden dargestellt:
\vspace{0,5cm}

\noindent \textbf{1. Skizzieren sie die Kräftegleichgewichte.}

	Antwort

\noindent \textbf{2. Leiten Sie aus den Kräftegleichgewichten die Formeln für r und Q her.}

	Antwort

\noindent \textbf{3. Schätzen Sie die Dauer der Beschleunigungsphasen nach den Richtungswechseln eines Öltröpfchens mit dem Radius $r = \SI{0,727}{\mu\m}$, indem Sie in beiden Fällen die Gleichung für das Kräftegleichgewicht nach der Geschwindigkeit auflösen und die Beschleunigung abschätzen. Muss die Beschleunigungsphase bei der Zeitmessung berücksichtigt werden?}

	Antwort

\noindent \textbf{4. Warum ist es wichtig, die Kondenstorplatten waagerecht auszurichten?}

	Antwort

\noindent \textbf{5. Warum sind Öltröpfchen besser geeignet als Wassertröpfchen, wenn man bedenkt, dass die Masse der untersuchten Objekte als konstant angesehen wird?}

	Antwort

\noindent \textbf{6. Bewegen sich gering geladene Tröpfchen im elektrischen Feld schneller oder langsamer als stark geladene? Wie ist (qualitativ) dementsprechend die Spannung zu wählen, wenn man gering bzw. stark geladene Öltröpfchen bei etwa gleicher Geschwindigkeit beobachten will?}

	Antwort

\noindent \textbf{7. Welcher Nachteil ergibt sich für die Auswertung, wenn man die fünf Zeitmessungen mit einer Additionsstoppuhr aufsummiert und durch fünf teilt, anstatt alle fünf Werte zu protokollieren und dann zu mitteln?}

	Antwort

\noindent \textbf{8. Wie hängen die Viskosität $\eta$ und die mittlere freie Weglänge $\lambda$ qualitativ von der Temperatur ab?}

	Antwort
	

\subsection{Methoden}

\subsubsection{Aufbau}

hallo\cite{aufbau}
\cite{Elementarladung}
% TODO

\subsubsection{Unsicherheiten}

% TODO

\subsection{Datenanalyse}

% TODO

\subsection{Diskussion}

% TODO

