\documentclass[11pt,a4paper,titlepage, ngerman]{article}

\usepackage[utf8]{inputenc}	
\usepackage[T1]{fontenc}	
\usepackage{ngerman}			
\usepackage{lmodern}			
\usepackage{graphicx}			
\usepackage{url}				
\usepackage{siunitx}
\usepackage{amsmath}			
\usepackage{subcaption}
\usepackage{wrapfig}

\newcommand{\refeq}[1]{Gl. (\ref{eq:#1})}
\newcommand{\refabb}[1]{Abb. \ref{abb:#1}}
\newcommand{\reftab}[1]{Tab. \ref{tab:#1}}

% Setup SI unit environment
\sisetup{separate-uncertainty = true}
\sisetup{output-decimal-marker = {,}}
\sisetup{
	per-mode=fraction,
	fraction-function=\sfrac
	% or \frac, \tfrac
}

\begin{document}

	\begin{titlepage}
		\centering
		{\scshape\LARGE Versuchsbericht zu \par}
		\vspace{1cm}
		{\scshape\huge M2 -- Gekoppelte Pendel\par}
		\vspace{2.5cm}
		{\LARGE Gruppe 10 Mi\par}
		\vspace{0.5cm}
		{\large Alex Oster (E-Mail: a\_oste16@uni--muenster.de) \par}
		{\large Jonathan Sigrist (E-Mail: j\_sigr01@uni--muenster.de) \par}
		\vfill
		durchgeführt am 29.11.2017\par
		betreut von\par
		{\large Martin \textsc{Körsgen}}		
		\vfill	
		{\large \today\par}
	\end{titlepage}
		
	\tableofcontents
		
	\newpage
	
	\section{Kurzfassung}
		
		Dieser Bericht beschäftigt sich mit der Betrachtung von gekoppelten Pendeln. Dazu werden das gekoppelte Fadenpendel und das Doppelpendel herangezogen und die verschiedenen Schwingungstypen untersucht.
		
		An zwei Fadenpendeln wird eine Reihe von Messungen durchgeführt, bei denen die Kopplung durch zwei verschiedene Federn realisiert wird. Für diese werden unterschiedliche Schwingungstypen separat untersucht. Die verschiedenen erhaltenen Schwingungsdauern werden dann miteinander verglichen und die Kopplungsgrade $k$ des Systems statisch und dynamisch bestimmt. Zudem wird die Bewegung des Doppelpendels beschrieben und kurz mit der Bewegung der gekoppelten Fadenpendel in Verbindung gebracht.
	
	\vspace{2cm}	
	\section{Gekoppelte Fadenpendel}
		
		\subsection{Methoden}
			
			Der Aufbau für den Versuch zu gekoppelten Fadenpendel ist in \refabb{gFadenpendel} dargestellt. 
			\begin{figure}[ht]
				\centering
				\includegraphics[width=\textwidth]{gFadenpendel.png}
				\caption{gekoppelte Fadenpendel}
				\label{abb:gFadenpendel}	
			\end{figure}
			Hier besitzen beide Pendel die gleiche Länge $l$, hier bei ca. \SI{159}{\cm} (von dem Aufhängepunkt zum Schwerpunkt der Pendel), und Gewichte der Masse $m$. Des Weiteren werden die Stangen (\glqq Fäden\grqq) für die Berechnung als masselos angenommen. Zur Kopplung der beiden Fadenpendel dienen hierbei zwei verschiedene Federn, welche bei ungefähr $z=d=\SI{113}{\cm}$ zwischen den Pendeln angebracht werden. Bei diesen handelt es sich um eine Kupfer- und um eine Stahlfeder.
			
			Die Auslenkung des Pendels wird mit Hilfe eines Ultraschall-Entfernungssensors gemessen. Diese wird gegen die Zeit aufgetragen, um die Schwingungsdauer $T$ bzw. die Frequenz $\omega$ zu ermitteln. Der Sensor misst mit \SI{50}{\Hz}. Die Messung wird für die Schwingungsdauer ohne Kopplung, für gleich- und gegen-gesinnte Bewegungen sowie für den Fall der Schwebung für beide Federn durchgeführt.
			Zur statischen Bestimmung des Kopplungsgrades wird die Auslenkung mit einem Maßstab gemessen. 

			Zur Berechnung des Kopplungsgrades wird folgende Formel für den statischen Fall verwendet: 
			\begin{equation}
				k = \frac{x_2}{x_1}. \label{eq:statisch}
			\end{equation}
			Für den dynamischen Fall werden die gemessenen Schwingungsdauern für die gleich- und gegen-gesinnte Bewegung verwendet:
			\begin{align}
				k = \frac{T_\text{gl}^2-T_\text{geg}^2}{T_\text{gl}^2+T_\text{geg}^2}. \label{eq:dynamisch} \\
				\text{bzw. für die Frequenz:} \quad k = \frac{\omega_\text{geg}^2-\omega_\text{gl}^2}{\omega_\text{geg}^2+\omega_\text{gl}^2}.
			\end{align}
			Zuletzt wird die relative Frequenzaufspaltung $\Delta\omega / \omega_0$ mit Hilfe der ermittelten Schwingungsdauern berechnet, dazu:
			\begin{equation}
				\frac{\Delta\omega}{\omega_0} = \frac{\omega_\text{geg}-\omega_\text{gl}}{\omega_\text{gl}} = 2\frac{T_\text{gl}}{T_\text{S}}. \label{eq:freqfrac} 	
			\end{equation}		
			Ebenso lässt sich die Frequenzaufspaltung auch mit der Näherung über die Reihenentwicklung von $\sqrt{1\pm k}$ bis zur 3. Ordnung bestimmen. Mit Umformen von \refeq{dynamisch} und Näherung folgt:
			\begin{equation}
				\frac{\Delta\omega}{\omega_0} = \sqrt\frac{1+k}{1-k}-1 = k + \frac{k^2}{2} + \frac{k^3}{2}. \label{eq:freqfrac2} 
			\end{equation}
		
		\subsection*{Unsicherheiten}
			
			Die Unsicherheiten wurden wie folgt festgelegt:
			\begin{itemize}
				\item Messgerät: Aus der Messrate von $\SI{50}{\hertz}$ folgt $u_C(T) = \frac{\SI{0,02}{\second}}{2\sqrt{3}}$.
				Zudem ist die Ausgabe der gemessenen Werte auf zwei Nachkommastellen bzw. auf \SI{1}{\cm} genau. Daraus folgt $u_C(x) = \frac{\SI{0,01}{m}}{2\sqrt{3}}$.
				\item Maßstab: Von diesem ließ sich mit einer Ungenauigkeit von \SI{0,5}{mm} ablesen. Demnach ist $u(x) = \frac{\SI{0,0005}{m}}{2\sqrt{6}}$.
				\item Messung über mehrere Perioden: Der Mittelwert ist gegeben durch $T = \frac{T_j - T_i}{j-i}$.
				Da $T_i$ und $T_j$ jeweils einzelne Messpunkte sind, gilt $u(T_i) = u(T_j) = u(T)$, somit folgt:
				\begin{equation}
					u(T) = \sqrt{\left( \frac{\partial\, T}{\partial\, T_i} u(T)\right)^2 + \left( \frac{\partial\, T}{\partial\, T_j} u(T)\right)^2}
					=\frac{u(T)}{j-i}.
				\end{equation}
				\item Schwebung: Da sich das Pendel bei der Schwebung an den Knoten über längere Zeit nicht bewegt, ist eine entsprechend große Unsicherheit für einen einzelnen Schwingungsbauch zu wählen (gemäß der Breite des Intervalls der nahezu gleichen Werte).
				Die gemittelte Zeit ist gegeben durch $T_S = 2\frac{T_j - T_i}{j-i}$. Der Faktor $2$ stammt daher, dass eine Periode jeweils zwei Schwingungsbäuche umfasst.
				Mit der Formel für kombinierte Unsicherheiten ergibt sich:
				\begin{equation}
					u(T_S) = \sqrt{\left( \frac{\partial\, T}{\partial\, T_i} u(T_i)\right)^2 + \left( \frac{\partial\, T}{\partial\, T_j} u(T_j)\right)^2}
					=\frac{2}{j-i} \sqrt{u^2(T_i) + u^2(T_j)}.
				\end{equation}
				$T_i$ beschreibt hierbei die Zeit bei dem $i$-ten Knotenpunkt.
			\end{itemize}
			
		\subsection{Eigenschwingung eines ungekoppelten Pendels}
										
			Die Messung ergab für den ungekoppelten Fall, dass die Schwingungsdauer $T_0 = \SI{2,4833 +- 0,0001}{\second}$ bzw. die Frequenz $\omega_0 = \SI{2,5301 +- 0,0005}{\s^{-1}}$ entspricht. Hierzu wurden die Differenzen der Nullstellen bei der Messung gebildet und gemittelt. 
			
		\subsection{Messung mit Kupferfeder als Kopplung}
			
			\subsubsection{Statische Ermittlung des Kopplungsgrades}							
				
				Der Kopplungsgrad wird durch Einsetzen in \refeq{statisch} statisch bestimmt. Dies liefert einen Wert von $k=\SI{0,025528 +- 0,000315}{}$. Hierfür wurden acht Wertepaare verwendet. Zum Vergleich dient Abbildung \ref{abb:CuKopp}.
				\begin{figure}[ht]
					\centering
					\includegraphics[width=\textwidth]{Kopplungsgerade-Cu.pdf}
					\caption{Kopplungsgerade für die Kupferfeder}
					\label{abb:CuKopp}	
				\end{figure}
			
			\subsubsection{Gleich-gesinnte Schwingung}
			
				Für die gleich-gesinnte Bewegung bei der Kupferfeder ergab sich, dass die Schwingungsdauer $T_\text{gl} = \SI{2,4736 +- 0,0001}{\second}$ bzw. die Frequenz $\omega_\text{gl}  = \SI{2,5401 +- 0,0001}{\s^{-1}}$ entspricht. Hierzu wurden die Differenzen der Nullstellen bei der Messung gebildet und gemittelt. 
			
			\subsubsection{Gegen-gesinnte Schwingung}
				
				Bei der gegen-gesinnten Bewegung der Kupferfeder ergab sich, dass die Schwingungsdauer $T_\text{geg} = \SI{2,4157 +- 0,0001}{\second}$ bzw. die Frequenz $\omega_\text{geg}  = \SI{2,6010 +- 0,0001}{\s^{-1}}$ entspricht. Hierzu wurden die Differenzen der Nullstellen bei der Messung gebildet und gemittelt. 
				
			\subsubsection{Schwebung}
				
				Die Messung ergab, dass die Schwingungsdauer $T_\text{S} = \SI{206,03 +-0,68}{\second}$ im Falle der Schwebung entspricht. Da die Periodenzahl bei der Schwebung sehr gering war, wurden für die Bestimmung der Periodendauer die Differenzen der Nullstellen und der Minima gebildet und gemittelt.

			\subsubsection{Relative Frequenzaufspaltung}
				
				Einsetzten in \refeq{freqfrac} liefert eine relative Frequenzaufspaltung von $\Delta\omega / \omega_0 = \SI{0,024010 +- 0,000080}{}$ über die ermittelten Schwingungsdauern. Über die Näherung aus \refeq{freqfrac2} erhält man hierbei eine Aufspaltung von $\Delta\omega / \omega_0 = \SI{0,025863 +- 0,000324}{}$.
				
		\subsection{Messung mit Stahlfeder als Kopplung}	
					
			\subsubsection{Statische Ermittlung des Kopplungsgrades}							
			
				Der Kopplungsgrad wird durch Einsetzen in \refeq{statisch} statisch bestimmt. Dies liefert einen Wert von $k=\SI{0,041683 +- 0,000315}{}$. Hierfür wurden neun Wertepaare verwendet. Zum Vergleich dient Abbildung \ref{abb:FeKopp}.
				\begin{figure}[ht]
					\centering
					\includegraphics[width=\textwidth]{Kopplungsgerade-Fe.pdf}
					\caption{Kopplungsgerade für die Stahlfeder}
					\label{abb:FeKopp}	
				\end{figure}
			
			\subsubsection{Gleich-gesinnte Schwingung}
			
				Für die gleich-gesinnte Bewegung bei der Kupferfeder ergab sich, dass die Schwingungsdauer $T_\text{gl} = \SI{2,4697 +- 0,0001}{\second}$ bzw. die Frequenz $\omega_\text{gl}  = \SI{2,5441 +- 0,0001}{\s^{-1}}$ entspricht. Hierzu wurden die Differenzen der Nullstellen bei der Messung gebildet und gemittelt. 
			
			\subsubsection{Gegen-gesinnte Schwingung}
			
				Bei der gegen-gesinnten Bewegung der Kupferfeder ergab sich, dass die Schwingungsdauer $T_\text{geg} = \SI{2,3753 +- 0,0001}{\second}$ bzw. die Frequenz $\omega_\text{geg}  = \SI{2,6452 +- 0,0001}{\s^{-1}}$ entspricht. Hierzu wurden die Differenzen der Nullstellen bei der Messung gebildet und gemittelt. 
			
			\subsubsection{Schwebung}
			
				Die Messung ergab, dass die Schwingungsdauer $T_\text{S} = \SI{104,52 +- 0,24}{\second}$ im Falle der Schwebung entspricht.  Da die Periodenzahl bei der Schwebung sehr gering war, wurden für die Bestimmung der Periodendauer die Differenzen der Nullstellen und der Minima gebildet und gemittelt.
				% Grafik, Referenz
			
			\subsubsection{Relative Frequenzaufspaltung}
			
				Einsetzten in \refeq{freqfrac} liefert eine relative Frequenzaufspaltung von $\Delta\omega / \omega_0 = \SI{0,047260 +- 0,000107}{}$ über die ermittelten Schwingungsdauern. Über die Näherung aus \refeq{freqfrac2} erhält man hierbei eine Aufspaltung von $\Delta\omega / \omega_0 = \SI{0,042587 +- 0,000329}{}$.
																
		\subsection{Ergebnisse}	
			
			\reftab{result} zeigt die Ergebnisse der Messungen für die verschiedenen Kopplungen, damit diese besser verglichen werden können. Hier ist ebenfalls der dynamisch bestimmte Kopplungsgrad aufgelistet, welcher sich durch Einsetzen in \refeq{dynamisch} berechnen ließ, sowie die durch:
			\begin{equation}
				T_\text{S} = \frac{4\pi}{\omega_\text{geg} - \omega_\text{gl}} 
			\end{equation}
			berechneten Schwebungsdauern.
			\begin{table}[ht]
				\caption{Ergebnisse der Versuchsreihe}
				\centering
				\label{tab:result}
				\begin{tabular}{l|S|S}
						& {Methode} & {Ergebnis}\\
					\hline
					{\textbf{ohne Kopplung}} & &\\
					{$T_0$} & {gemessen} & \SI{2,4833 +- 0,0001}{\second}\\
					{$\omega_0$} & {aus $T_0$} & \SI{2,5301 +- 0,0005}{\s^{-1}}\\
					\hline
					{\textbf{Kupferfeder}} & &\\
					{$T_\text{gl}$} & {gemessen} & \SI{2,4736 +- 0,0001}{\second}\\
					{$\omega_\text{gl}$} & {aus $T_\text{gl}$} & \SI{2,5401 +- 0,0001}{\s^{-1}}\\
					{$T_\text{geg}$} & {gemessen} & \SI{2,4157 +- 0,0001}{\second}\\
					{$\omega_\text{geg}$} & {aus $T_\text{geg}$} & \SI{2,6010 +- 0,0001}{\s^{-1}}\\
					{$T_\text{S}$} & {gemessen} & \SI{206,03 +-0,68}{\second}\\
					{$T_\text{S}$} & {berechnet} & \SI{206,35 +- 0,33}{\second}\\
					{$k$} & {statisch} & \SI{0,025528 +- 0,000315}{}\\
					{$k$} & {dynamisch} &\SI{0,023675 +- 0,000031}{}\\
					{$\Delta\omega / \omega_0$} & {berechnet} & \SI{0,024010 +- 0,000080}{}\\
					{$\Delta\omega / \omega_0$} & {genähert} & \SI{0,025863 +- 0,000324}{}\\
					\hline
					{\textbf{Stahlfeder}} & &\\
					{$T_\text{gl}$} & {gemessen} & \SI{2,4697 +- 0,0001}{\second}\\
					{$\omega_\text{gl}$} & {aus $T_\text{gl}$} & \SI{2,5441 +- 0,0001}{\s^{-1}}\\
					{$T_\text{geg}$} & {gemessen} & \SI{2,3753 +- 0,0001}{\second}\\
					{$\omega_\text{geg}$} & {aus $T_\text{geg}$} & \SI{2,6452 +- 0,0001}{\s^{-1}}\\
					{$T_\text{S}$} & {gemessen} & \SI{104,52 +- 0,24}{\second}\\
					{$T_\text{S}$} & {berechnet} & \SI{124,30 +- 0,13}{\second}\\
					{$k$} & {statisch} & \SI{0,041683 +- 0,000315}{}\\
					{$k$} & {dynamisch} & \SI{0,038939 +- 0,000026}{}\\
					{$\Delta\omega / \omega_0$} & {berechnet} & \SI{0,047260 +- 0,000107}{}\\
					{$\Delta\omega / \omega_0$} & {genähert} & \SI{0,042587 +- 0,000329}{}\\
				\end{tabular}
			\end{table}
			
			Vergleicht man die Schwingungsdauern für die Schwebung bei direkter Messung mit den berechneten, so ist zu erkennen, dass der Unterschied bei der Kupferfeder minimal ist 0,2\%, bei der Stahlfeder jedoch mit 18,9\% im Verhältnis zu der vorausgegangenen Feder stark abweicht.
			
			Bei den Kopplungsgraden besteht bei der statistischen Berechnung ein 7,8\% Unterschied zu den Dynamischen bei der Kupferfeder und 7,0\% bei der Stahlfeder.
			
			Für die Näherung der relativen Frequenzaufspaltung weichen die Werte von den Berechneten um 7,7\% für die Kupferfeder und um 9,9\% für die Stahlfeder ab.
			
		\subsection{Schlussfolgerung}
			
			Da die Werte für die Schwebungsdauer der Stahlfeder stark voneinander abweichen, sollte die Messung zu dieser wiederholt werden. Für die Kopplungsgrade $k$ und der relativen Frequenzaufspaltung $\Delta\omega / \omega_0$ sind die Unsicherheiten sehr gering, was den hohen prozentualen Unterschied der verschiedenen Ergebnisse erklären könnte. Demnach müsste die Berechnung der Unsicherheiten hier erneut überprüft werden. Dass die Stangen, als masselos angenommen werden, hat ebenfalls Einfluss auf die Genauigkeit der Messung, jedoch nicht genug um die starken Abweichungen zu erklären.
			
	\section{Doppelpendel}
	
		\subsection{Aufbau und Funktionsweise}	
			
			Das Doppelpendel besteht aus zwei gekoppelten Pendeln. Hierbei ist das obere Pendel an einem festen Aufhängepunkt angebracht, während das Untere Pendel am Ende des Ersten befestigt ist. Die Abbildungen \ref{abb:DP_Ruhe}) bis \ref{abb:DP_stabil}) zeigen die Funktionsweise des Doppelpendels. Hier sind die verschiedenen Auslenkungsmöglichkeiten zu erkennen, wobei die einzelnen Pendel sich jeweils auf Kreisbahnen um deren Aufhängepunkte bewegen können. Durch die Kopplung ist ein Austausch von Energie zwischen beiden Pendeln möglich.
			\begin{figure}[ht]
				\begin{subfigure}{0.5\textwidth}
					\centering
					\includegraphics[width=0.5\textwidth]{Doppelpendel_Ruhelage.png}
					\caption{Doppelpendel in Ruhelage}
					\label{abb:DP_Ruhe}	
				\end{subfigure}
				\begin{subfigure}{0.5\textwidth}
					\centering
					\includegraphics[width=0.5\textwidth]{Doppelpendel_Auslenkung_P2.png}
					\caption{Oberes Pendel in Ruhelage, Unteres ausgelenkt (Chaos)}
					\label{abb:DP_RobenAunten}
				\end{subfigure}
				\begin{subfigure}{0.5\textwidth}
					\centering
					\includegraphics[width=0.5\textwidth]{Doppelpendel_Auslenkung_P1.png}
					\caption{Unteres Pendel in Ruhelage, Oberes ausgelenkt (Chaos)}
					\label{abb:DP_RuntenAoben}
				\end{subfigure}
				\begin{subfigure}{0.5\textwidth}
					\centering
					\includegraphics[width=0.5\textwidth]{Doppelpendel_Auslenkung_beide.png}
					\caption{Beide Pendel gegensinnig ausgelenkt (Gegenschwingung)}
					\label{abb:DP_ausgelenkt}
				\end{subfigure}		
				\begin{subfigure}{0.5\textwidth}
					\centering
					\includegraphics[width=0.5\textwidth]{Doppelpendel_stabil.png}
					\caption{Beide Pendel gleichsinnig ausgelenkt (Gleichschwingung)}
					\label{abb:DP_stabil}	
				\end{subfigure}
				\caption{Auslenkungsmöglichkeiten des Doppelpendels}
			\end{figure}
		
		\subsection{Beobachtung bei Auslenkung}
		
			Bei der Auslenkung des Doppelpendels lässt sich ein nichtlineares bzw. chaotisches Verhalten erkennen, da nach nur wenigen Schwingungen, bei nahezu gleichen Anfangsbedingungen, bereits stark unterschiedliche Bewegungen beobachtet wurden. Ein periodisches Verhalten ließ sich hierbei nicht erkennen.
		
			Die Ausnahmen hierzu bilden die Schwingungen, bei Auslenkungen ähnlich zu den, wie sie in \refabb{DP_stabil}) und \refabb{DP_ausgelenkt}) dargestellt sind. Bei der ersten linearen Bewegung bewegt sich der Massepunkt des unteren Pendels auf einer Kreisbahn um den Aufhängepunkt des Oberen. Die beiden Pendel liegen bei dieser Bewegung parallel. Für den zweiten Fall einer linearen Bewegung sind die beide Pendel um den gleichen Winkel, jedoch in entgegengesetzte Richtung ausgelenkt, wie es in \refabb{DP_ausgelenkt} dargestellt ist (hier ca. $\pm \ang{45}$). Da sich die Winkel einander negieren, bewegt sich der Massepunkt des unteren Pendels immer nur auf und ab. Dieser Effekt war jedoch nur bei einem Auslenkungswinkel von ca. $\pm \ang{10}$ zu beobachten. Bei größeren Auslenkungen traten erneut nichtlineare Bewegungen auf.
			
			Diese linearen Bewegung ähneln der gleich- (\refabb{DP_stabil}) bzw. gegen-gesinnten Bewegung (\refabb{DP_ausgelenkt}) der gekoppelten Fadenpendel.  


\end{document} 