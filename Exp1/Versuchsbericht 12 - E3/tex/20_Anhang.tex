\subsection{Unsicherheitsrechnung}\label{VGuD}

\begin{figure}[h]
	\begin{equation*}
		x = \sum_{i=1}^{N} x_i
		;\quad
		u(x) = \sqrt{\sum_{i = 1}^{N} u(x_i)^2}
	\end{equation*}
	\caption{Formel für kombinierte Unsicherheiten des selben Typs nach GUM.}
	\label{eq:GUM_combine}
\end{figure}

\begin{figure}[h]
	\begin{equation*}
		f = f(x_1, \dots , x_N)
		;\quad
		u(f) = \sqrt{\sum_{i = 1}^{N}\left(\pdv{f}{x_i} u(x_i)\right) ^2}
	\end{equation*}
	\caption{Formel für sich fortpflanzende Unsicherheiten nach GUM.}
	\label{eq:GUM_formula}
\end{figure}

\begin{figure}[h]
\begin{equation*}
I = \frac{U}{R}
;\quad
u(I) = \sqrt{\left( \frac{u(U)}{R} \right) ^2 + \left( -\frac{U}{R^2} u(R) \right) ^2}
\end{equation*}
\caption{Unsicherheitsrechnung für den Strom an einem ohm'schen Widerstand.}
\label{eq:R=U/I}
\end{figure}

\begin{figure}[h]
\begin{equation*}
L = \frac{1}{\omega_0^2 C}
;\quad
u(L) = \frac{u(C)}{\omega_0^2 C^2}
\end{equation*}
\caption{Unsicherheitsrechnung für die Induktivität eines Schwingkreises im Resonanzfall.}
\label{eq:L=1/w^2C}
\end{figure}

\begin{figure}[h]
\begin{equation*}
\abs{U_L} = \frac{\omega_0 L}{R} \abs{U}
;\quad
u(\abs{U_L}) = \left[ \left( \frac{\omega_0 L}{R} u(\abs{U}) \right) ^2+\left( \frac{\omega_0 \abs{U}}{R} u(L) \right) ^2+\left( -\frac{\omega_0 L \abs{U}}{R^2} u(R) \right) ^2 \right] ^{\frac{1}{2}}
\end{equation*}
\caption{Unsicherheitsrechnung für den Spannungsabfall über die Spule im Resonanzfall bei dem Serienschwingkreis. Dabei sei $\abs{U} = U_\text{Pk-Pk}$.}
\label{eq:U_L=wLU/R}
\end{figure}

\begin{figure}[h]
\begin{equation*}
\abs{U_C} = \frac{1}{\omega_0 C R} \abs{U}
;\quad
u(\abs{U_L}) = \left[ \left( \frac{1}{\omega_0 C R} u(\abs{U}) \right) ^2+\left( -\frac{\abs{U}}{\omega_0 C^2 R} u(C) \right) ^2+\left( -\frac{\abs{U}}{\omega_0 C R^2} u(R) \right) ^2 \right] ^{\frac{1}{2}}
\end{equation*}
\caption{Unsicherheitsrechnung für den Spannungsabfall über den Kondensator im Resonsanzfall bei dem Serienschwingkreis.}
\label{eq:U_C=U/wCR}
\end{figure}

\begin{figure}[h]
\begin{equation*}
\abs{R} = \frac{1}{2\omega_0}\left( \frac{1}{C_2} - \frac{1}{C_1} \right) 
;\quad
u(R) = \frac{1}{2\omega_0} \left[ u\left( \frac{1}{C_2 }\right) ^2 + u\left( \frac{1}{C_1} \right) ^2 \right] ^{\frac{1}{2}}
\end{equation*}
\caption{Unsicherheitsrechnung für den Verlustwiderstand des gesammten Serienschwingkreises.}
\label{eq:R=dC/2w}
\end{figure}

\begin{figure}[h]
\begin{equation*}
\abs{R_i} = R - R_v
;\quad
u(R_i) = \sqrt{u(R)^2 + u(R_v)^2}
\end{equation*}
\caption{Unsicherheitsrechnung für den Innenwiderstand der Spule bei dem Serienschwingkreises.}
\label{eq:R_i-serie}
\end{figure}

\begin{figure}[h]
\begin{equation*}
\abs{R_p'} = \frac{R_v \cdot R}{R_v-R}
;\quad
u(R_p') = u(R) R_v \left( \frac{1}{R_v - R + \frac{R}{(R_v - R)^2}} \right) 
\end{equation*}
\caption{Unsicherheitsrechnung für den Ersatzwiderstand $R_p'$ bei dem Parallelschwingkreis. Dabei sei durch die kleine Unsicherheit des Messgerätes $u(R_v) = 0$.}
\label{eq:R_p-parallel}
\end{figure}

\begin{figure}[h]
\begin{equation*}
\abs{R_i} = \frac{\omega_0^2 L^2}{R_p'}
;\quad
u(R_i) = \omega_0^2 \left[ \left( \frac{2 L}{R_p'} u(L) \right) ^2 + \left( -\frac{L^2}{R_p'^2} u(R_p') \right) ^2 \right] ^{\frac{1}{2}}
\end{equation*}
\caption{Unsicherheitsrechnung für den Innenwiderstand der Spule bei dem Parallelschwingkreis.}
\label{eq:R_i-parallel}
\end{figure}



