
\section{Kurzfassung}

Dieser Bericht befasst sich mit der Betrachtung von elektrischer Resonanz bei Schwingkreisen.
Dazu werden zwei verschiedene Schwingkreise betrachtet.
Hierbei handelt es sich um eine Serien- und um eine Parallelschaltung von Kondensator und Spule.
Über die vorliegenden Widerstände und der gemessenen Spannung wird die Stromstärke ermittelt und in Abhängigkeit der Frequenz über die Kapazität des Kondensators, welche regulierbar ist, aufgetragen.
Aus diesen Resonanzkurven, die für verschiedene Widerstände aufgenommen werden, lassen sich die Induktivitäten der verwendeten Spulen bestimmen.
Ziel der Messung ist die Aufnahme von Resonanzkurven sowie auch Ermittlung von Induktivitäten, die der Theorie entsprechen.
Demnach sind Lorentzkurven für die Resonanzkurven und eine größere Induktivität bei der größeren Spule als bei der kleineren zu erwarten.
Die Ergebnisse der Messung stimmen mit Induktivitäten von ca. \SI{2,4}{\henry} für die große Spule und ca. \SI{0,068}{\henry} für die kleinere mit den Erwartungen überein.
Auch die Resonanzkurven besitzen die erwartete Form.