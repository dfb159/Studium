\documentclass[11pt,a4paper,titlepage, ngerman]{article}

\usepackage[utf8]{inputenc}	% Diese Pakete sind
\usepackage[T1]{fontenc}		% für die Verwendung 
\usepackage{ngerman}			% von Umlauten im tex-file
\usepackage{lmodern}			% Schriftart, die am Bildschirm besser lesbar ist
\usepackage{graphicx}			% Zum Einbinden von Formeln
\usepackage{url}					% Zur Darstellung von Webadressen
\usepackage{siunitx}
\usepackage{amsmath}			% für equation*
\usepackage{subcaption}

\begin{document}
	\setlength{\parindent}{0em} 
	
	\begin{titlepage}
		\centering
		{\scshape\LARGE Versuchsbericht zu \par}
		\vspace{1cm}
		{\scshape\huge S2 -- Experimentieren, und dann?\par}
		\vspace{2.5cm}
		{\LARGE Gruppe 10 Mi\par}
		\vspace{0.5cm}
		{\large Alex Oster (E-Mail: a\_oste16@uni--muenster.de) \par}
		{\large Jonathan Sigrist (E-Mail: j\_sigr01@uni--muenster.de ) \par}
		\vfill
		durchgeführt am 25.10.2017\par
		betreut von\par
		{\large Dr. Anke \textsc{Schmidt}}
		
		\vfill
		
		{\large \today\par}
	\end{titlepage}
		
	\tableofcontents
	
	\newpage
	
	\section{Einleitung}
		\label{Einleitung}
		
		
		\glqq Die Gravitationskonstante $g$ besitzt in der Stadt Münster einen Wert von \SIrange{10,5}{11}{m/s^2}.\grqq
		\\
		Zu dieser Aussage führten die Ergebnisse einer Fallgeschwindigkeitsmessung. Hierbei wurden die Geschwindigkeiten einer fallenden Metallkugel an zwei verschiedenen Punkten, mit Hilfe von Lichtschranken gemessen. Dadurch ließ sich die Fallbeschleunigung, d.h. die Gravitationskonstante, bestimmen.  
		\\
		\\
		In diesem Bericht beschäftigen wir uns damit, diese Aussage zu widerlegen.
		Dazu messen wir die Zeit von Pendelschwingungen und berechnen aus Fadenlänge und gemessener Zeit die Gravitationskonstante. Um ein möglichst genaues Ergebnis zu erhalten, führen wir mehrere Messungen mit verschiedenen Fadenlängen durch.
		
	\section{Durchführung}
		\label{Durchführung}
		
		Für unsere Messung haben wir eine an einem Faden befestigte Metallkugel verwendet, wobei der Faden selber an einer Halterung befestigt war und seine Länge sich beliebig einstellen ließ.
		Die Metallkugel besaß einen Durchmesser von \SI{3}{cm} (bzw. einen Radius von \SI{1,5}{cm}).
		
		\begin{description}
			\item[Seillänge 1:](\SI{111,6}{cm})\\ 
			
			Für die erste Seillänge haben wir elf mal zehn ganze Schwingungsdauern gemessen, insgesamt also 110 Schwingungen. Die durchschnittliche Zeit für eine Pendelschwingung betrug hierbei: \SI{2.134182}{s}
			
			\item[Seillänge 2:](\SI{104,5}{cm})\\ 
			 
			Bei dieser und der folgenden Messung haben wir nur 20 Perioden (vier mal fünf Schwingungen) gemessen. Die durchschnittliche Zeit für eine Pendelschwingung betrug hierbei: \SI{0}{s}
			
			\item[Seillänge 3:](\SI{86,9}{cm})\\ 
			
			Wie auch bei der vorherigen Messung wurden hier nur 20 Schwingungen betrachtet. Die durchschnittliche Zeit für eine Pendelschwingung betrug hierbei: \SI{0}{s}		
			
			\item[Seillänge 4:](\SI{65,8}{cm})\\ 
			
			Diese und nächste Messung betrachten wir 25 Perioden ($5*5$Schwingungen). Die durchschnittliche Zeit für eine Pendelschwingung betrug hierbei: \SI{0}{s}
			
			\item[Seillänge 5:](\SI{116,3}{cm})\\ 
			
			Auch hier wurden nur 25 Schwingungsdauern gemessen. Die durchschnittliche Zeit für eine Pendelschwingung betrug hierbei: \SI{0}{s}
			
			\item[Seillänge 6:](\SI{92,6}{cm})\\ 
			
			Da wir bei den letzten vier Fadenlängen nicht sonderlich viele Schwingungen betrachtet haben, haben wir für die sechste Fadenlänge noch einmal 100 Schwingungsdauern gemessen (zehn mal zehn Schwingungen). 
			Die durchschnittliche Zeit für eine Pendelschwingung betrug hierbei: \SI{0}{s}
			
		\end{description}
		
	\section{Diskussion}	
		\label{Diskussion}		
		
	\section{Anhang}
		\label{Anhang}
	
\end{document} 