\documentclass[11pt,a4paper,titlepage, ngerman]{article}

\usepackage[utf8]{inputenc}	
\usepackage[T1]{fontenc}	
\usepackage{ngerman}			
\usepackage{lmodern}			
\usepackage{graphicx}			
\usepackage{url}				
\usepackage{siunitx}
\usepackage{amsmath}	
\usepackage{xfrac}		
\usepackage{subcaption}
\usepackage{wrapfig}
\usepackage{biblatex}

\newcommand{\refeq}[1]{Gl. (\ref{eq:#1})}
\newcommand{\refabb}[1]{Abb. \ref{abb:#1}}
\newcommand{\reffig}[1]{Fig. \ref{fig:#1}}
\newcommand{\reftab}[1]{Tab. \ref{tab:#1}}

% Setup SI unit environment
\sisetup{separate-uncertainty = true}
\sisetup{output-decimal-marker = {,}}
\sisetup{
	per-mode=fraction,
	fraction-function=\sfrac
	% or \frac, \tfrac
}

\begin{document}
	
	
	\begin{titlepage}
		
		\centering
		{\scshape\LARGE Gleichungen zu \par}
		\vspace{1cm}
		{\scshape\huge M1 -- Drehpendel nach Pohl\par}
		\vspace{2.5cm}
		{\LARGE Gruppe 10 Mi\par}
		\vspace{0.5cm}
		{\large Alex Oster (E-Mail: a\_oste16@uni--muenster.de) \par}
		{\large Jonathan Sigrist (E-Mail: j\_sigr01@uni--muenster.de) \par}
		\vfill
		durchgeführt am 15.11.2017\par
		betreut von\par
		{\large Johann Preuß}		
		\vfill	
		{\large \today\par}
		
	\end{titlepage}
		
	\tableofcontents
		
	\newpage
	
	\section{Unsicherheiten}
	Unsicherheiten mit SI-Befehl:\\
	 \SI{123.45 +- 5.845}{\meter\per\second\cubed}
	
	
	\subsection*{Computer}
	Das Computerprogram hatte eine Abtastrate von $\SI{50}{\hertz}$. Daraus folgt $u_C(T) = \frac{\SI{0,02}{\second}}{2\sqrt{3}}$.
	Der Ultraschallsensor hatte eine Genauigkeit von 2 Nachkommastellen, also \SI{1}{cm}. Es folgt $u_C(x) = \frac{\SI{0,01}{m}}{2\sqrt{3}}$.
	
	\subsection*{Per Hand}
	Wir konnten auf dem Maßstab bis \SI{0,5}{mm} ablesen. Also ist $u(x) = \frac{\SI{0,001}{m}}{2\sqrt{6}}$.
	
	\subsection{Messung über mehrere Perioden}
	Der Mittelwert ist gegeben mit $T = \frac{T_j - T_i}{j-i}$.
	Da $T_i$ und $T_j$ jeweils einzelne Messpunkte sind, gilt $u(T_i) = u(T_j) = u(T)$, somit folgt:
	\begin{equation}
		u(T) = \sqrt{\left( \frac{\partial\, T}{\partial\, T_i} u(T)\right)^2 + \left( \frac{\partial\, T}{\partial\, T_j} u(T)\right)^2}
		=\frac{u(T)}{j-i}
	\end{equation}
	
	\subsection{Schwebung}
	Da sich das Pendel bei der Schwebung an den Knoten nicht bewegt, ist eine entsprechend große Unsicherheit für einen einzelnen Schwingungsbauch zu wählen (Gerade der Breite des Intervalls mit gleichbleibenden Werten).
	Die gemittelte Zeit ist gegeben mit $T_S = 2\frac{T_j - T_i}{j-i}$, wobei der Vorfaktor $2$ daher stammt, dass eine Periode jeweils zwischen zwei Bäuchen liegt.
	Mit der Formel für kombinierte Unsicherheiten ergibt sich:
	\begin{equation}
		u(T_S) = \sqrt{\left( \frac{\partial\, T}{\partial\, T_i} u(T_i)\right)^2 + \left( \frac{\partial\, T}{\partial\, T_j} u(T_j)\right)^2}
		=\frac{2}{j-i} \sqrt{u^2(T_i) + u^2(T_j)}.
	\end{equation}
	$T_i$ beschreibt dabei die Zeit des $i$-ten Bauchs.
	
	\section{Schwingungsperioden}
	Es wurden zum Teil die Minima betrachtet, da man so mehr Messpunkte hatte. Die ausgewerteten Ergebnisse sammt Unsicherheiten sind in Tab. \ref{tab:result} abzulesen.
	
	\begin{table}[ht]
		\caption{Schwingungsdauern einer Periode mit Unsicherheiten.}
		\centering
		\label{tab:result}
		\begin{tabular}{c|S}
			{Schwingungsart} & {Schwingungsdauer $T$} \\
			\hline
			{Eigenschwingung eines Pendels} & \SI{2,4833 +- 0,0001}{\second} \\
			\hline
			{Gleichschwingung der Kupferfeder} & \SI{2,4736 +- 0,0001}{\second} \\
			{Gegenschwingung der Kupferfeder} & \SI{2,4157 +- 0,0001}{\second} \\
			{Schwebung der Kupferfeder} & \SI{206,03 +-0,68}{\second} \\
			\hline
			{Gleichschwingung der Stahlfeder} & \SI{2,4697 +- 0,0001}{\second} \\
			{Gegenschwingung der Stahlfeder} & \SI{2,3753 +- 0,0001}{\second} \\
			{Schwebung der Stahlfeder} & \SI{104,52 +- 0,24}{\second} \\
		\end{tabular}
	\end{table}
	

	\newpage			
	\section*{Literatur}
\end{document} 