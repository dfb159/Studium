\documentclass[11pt,a4paper,titlepage, ngerman]{article}

\usepackage[utf8]{inputenc}	
\usepackage[T1]{fontenc}	
\usepackage{ngerman}			
\usepackage{lmodern}			
\usepackage{graphicx}			
\usepackage{url}				
\usepackage{siunitx}
\usepackage{amsmath}	
\usepackage{xfrac}		
\usepackage{subcaption}
\usepackage{wrapfig}
\usepackage{biblatex}

\newcommand{\refeq}[1]{Gl. (\ref{eq:#1})}
\newcommand{\refabb}[1]{Abb. \ref{abb:#1}}
\newcommand{\reffig}[1]{Fig. \ref{fig:#1}}
\newcommand{\reftab}[1]{Tab. \ref{tab:#1}}

% Setup SI unit environment
\sisetup{separate-uncertainty = true}
\sisetup{output-decimal-marker = {,}}
\sisetup{
	per-mode=fraction,
	fraction-function=\sfrac
	% or \frac, \tfrac
}

\begin{document}
	
	
	\begin{titlepage}
		
		\centering
		{\scshape\LARGE Gleichungen zu \par}
		\vspace{1cm}
		{\scshape\huge M1 -- Drehpendel nach Pohl\par}
		\vspace{2.5cm}
		{\LARGE Gruppe 10 Mi\par}
		\vspace{0.5cm}
		{\large Alex Oster (E-Mail: a\_oste16@uni--muenster.de) \par}
		{\large Jonathan Sigrist (E-Mail: j\_sigr01@uni--muenster.de) \par}
		\vfill
		durchgeführt am 15.11.2017\par
		betreut von\par
		{\large Johann Preuß}		
		\vfill	
		{\large \today\par}
		
	\end{titlepage}
		
	\tableofcontents
		
	\newpage
	
	\section{Unsicherheiten}
	Unsicherheiten mit SI-Befehl:\\
	 \SI{123.45 +- 5.845}{\meter\per\second\cubed}
	
	
	\subsection*{Computer}
	Das Computerprogram hatte eine Abtastrate von $\SI{50}{\hertz}$. Daraus folgt $u_C(T) = \frac{\SI{0,02}{\second}}{2\sqrt{3}}$.
	Der Ultraschallsensor hatte eine Genauigkeit von 2 Nachkommastellen, also \SI{1}{cm}. Es folgt $u_C(x) = \frac{\SI{0,01}{m}}{2\sqrt{3}}$.
	
	\subsection*{Per Hand}
	Wir konnten auf dem Maßstab bis \SI{0,5}{mm} ablesen. Also ist $u(x) = \frac{\SI{0,001}{m}}{2\sqrt{6}}$.
	
	\section{Eigenschwingung}

	\newpage			
	\section*{Literatur}
\end{document} 