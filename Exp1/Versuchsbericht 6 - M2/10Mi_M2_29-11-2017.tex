\documentclass[11pt,a4paper,titlepage, ngerman]{article}

\usepackage[utf8]{inputenc}	
\usepackage[T1]{fontenc}	
\usepackage{ngerman}			
\usepackage{lmodern}			
\usepackage{graphicx}			
\usepackage{url}				
\usepackage{siunitx}
\usepackage{amsmath}			
\usepackage{subcaption}
\usepackage{wrapfig}

\newcommand{\refeq}[1]{Gl. (\ref{eq:#1})}
\newcommand{\refabb}[1]{Abb. \ref{abb:#1}}
\newcommand{\reftab}[1]{Tab. \ref{tab:#1}}

\begin{document}

	\begin{titlepage}
		\centering
		{\scshape\LARGE Versuchsbericht zu \par}
		\vspace{1cm}
		{\scshape\huge M2 -- Gekoppelte Pendel\par}
		\vspace{2.5cm}
		{\LARGE Gruppe 10 Mi\par}
		\vspace{0.5cm}
		{\large Alex Oster (E-Mail: a\_oste16@uni--muenster.de) \par}
		{\large Jonathan Sigrist (E-Mail: j\_sigr01@uni--muenster.de) \par}
		\vfill
		durchgeführt am 29.11.2017\par
		betreut von\par
		{\large Martin \textsc{Nösgen}}		
		\vfill	
		{\large \today\par}
	\end{titlepage}
		
	\tableofcontents
		
	\newpage
	
	\section{Kurzfassung}
		
		%TODO
		Dieser Bericht beschäftigt sich mit der Betrachtung von gekoppelten Pendeln. Dazu wird auf zwei verschiedene Arten solcher Pendel genauer eingegangen. Dabei handelt es sich einerseits um gekoppelte Fadenpendel und andererseits um das sogenannte Doppelpendel. 
		
		Zu den gekoppelten Fadenpendeln wird eine Reihe von Messungen durchgeführt. Bei diesen variieren die Kopplungen und Anfangsauslenkungen. Die verschiedenen erhaltenen Schwingungsdauern werden miteinander verglichen und die Kopplungsgrade $k$ des Systems statisch und dynamisch bestimmt. Zudem wird die Bewegung des Doppelpendels beschrieben.
		
	\section{Gekoppelte Fadenpendel}
				
		%TODO
		
		\subsection{Methoden}
			
			%TODO
			Der Aufbau für den Versuch zu gekoppelten Fadenpendel ist in \refabb{} dargestellt. Dabei besitzen beide Pendel die gleiche Länge $l$ und Masse $m$. Des weiteren werden die Fäden für die Berechnung als masselos angenommen. Zur Kopplung der beiden Fadenpendel dienen hierbei zwei verschiedene Federn. Bei diesen handelt es sich um eine Kupfer- und um eine Stahlfeder.
			
			%TODO
			Zur Berechnung des Kopplungsgrades wird folgende Formel für den statischen Fall verwendet: 
			\begin{equation}
				k = \frac{x_1}{x_2}.
			\end{equation}
			Für den dynamischen Fall werden die gemessenen Schwingungsdauern für die gleich- und gegensinnige Bewegung verwendet:
			\begin{equation}
			k = \frac{T_\text{gl}^2-T_\text{geg}^2}{T_\text{gl}^2+T_\text{geg}^2}.
			\end{equation}
		\subsection{Messung}
			
			%TODO
			
		\subsection{Schlussfolgerung}
							
			%TODO
				
	\section{Doppelpendel}
				
		%TODO
		\refabb{} zeigt den Aufbau des Doppelpendels. Hier ist zu erkennen, dass 
		%TODO
		Bei der Auslenkung lässt sich ein nichtlineares Verhalten des Pendels erkennen, da bei gleichen Auslenkungen unterschiedliche Bewegungen beobachtet wurden. 
		%TODO
		Die Ausnahmen bildeten die Schwingungen, wie sie in \refabb{} dargestellt sind. Hier ließ sich ein lineares Verhalten betrachten.
		
\end{document} 