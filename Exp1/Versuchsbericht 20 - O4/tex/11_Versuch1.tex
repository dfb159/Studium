\section{Methoden} \label{sec:Methoden}
			
	Zum Versuchsaufbau gehören ein Elektromagnet, eine Stromquelle, eine Hall-Sonde, zwei Polarisatoren, ein (roter) Laser, ein Lichtintensitätsmessgerät und eine Co/Pt-Probe.
	Der Elektromagnet besteht aus einem hufeisenförmigen Eisenkern, auf dessen Seiten Spulen mit je 600 Windungen angebracht sind, durch die ein Strom aus der Stromquelle fließen kann. 
	Zudem sind an den Enden des Eisenkerns Polschuhe angebracht, zwischen denen sich ein Magnetfeld bei Stromfluss durch die Spulen bilden sollte.
	Für den ersten Versuchsteil soll die Feldstärke $B$ dieses Magnetfeldes in Abhängigkeit der Stromstärke $I$ über die Hall-Sonde gemessen werden.
	Dazu wird die Sonde so ausgerichtet, dass möglichst parallel zu dem Magnetfeld gemessen werden kann.
	Bei dem zweiten Versuchsteil soll die Hall-Sonde entfernt und die Co/Pt-Probe zwischen den Polschuhen platziert werden.
	Der Laser wird so aufgestellt, dass ein gerader Strahlengang durch den ersten Polarisator auf die Probe und des reflektierten Strahls durch den zweiten Polarisator stattfinden kann.
	Hinter letzterem soll dann die Lichtintensität in Abhängigkeit der Stromstärke $I$ (der magnetischen Feldstärke $B$) gemessen werden.
	Für höhere Empfindlichkeit sollen die Polarisatoren um \SI{45}{\degree} versetzt polarisieren, da die Änderung der Intensität nach dem Gesetz von Malus dann am größten ist.
	Bei beiden Messungen sollen die Messgrößen in Abhängigkeit Stromstärke von \SIrange{-1}{1}{\ampere}	und wieder zurück von \SIrange{1}{-1}{\ampere} in \SI{0,1}{\ampere} Schritten aufgetragen werden.
	% TODO Theorie?
	
\section{Durchführung}
		
	Die Messung der Feldstärke $B$ und der Lichtintensität wurden analog zu der Beschreibung in Abschnitt \ref{sec:Methoden} durchgeführt.
	An der Stromquelle ließen sich keine negativen Ströme einstellen, weswegen bei dem Erreichen von \SI{0}{\ampere} einfach die Eingänge getauscht wurden.
	Für beide Messungen fand die Ausgabe der Messwerte auf einem Computer statt, an den die Messinstrumente angeschlossen waren.
	Da das System sehr empfindlich gegenüber dem Licht in dem Versuchsraum und leichten Erschütterungen war, wurde nach Aufnahme einer Messung der zugehörige Graph geplotted, um zu versichern dass die Messwerte aufgrund kleiner Einflüsse nicht unnatürlich weit voneinander abweichen.
	Dies war bei den aufgenommenen Werten nicht der Fall, weswegen keine zusätzliche Messreihe durchgeführt wurde. 
		
\section{Datenanalyse}
			
	Eine graphische Darstellung der aufgenommenen Messwerte für den ersten Teilversuch ist Abb. \ref{} zu entnehmen.
	Dabei sind ... (Farben)
	Zu erkennen ist ein ... Verlauf
	Die Messkurve für den zweiten Teilversuch ist in Abb. \ref{label} dargestellt.
	Wie bei der ersten Messung sind die Messpunkte in (Farben) getrennt.
	% TODO
			
\section{Diskussion}
			
	% TODO
