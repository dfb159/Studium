\subsection{Unsicherheiten}\label{VGuD}

Jegliche Unsicherheiten werden nach GUM bestimmt und berechnet.
%Die Gleichungen dazu finden sich in \ref{fig:GUM_combine} und \ref{fig:GUM_formula}.
Für die Unsicherheitsrechnungen wurde die Python Bibliothek "uncertainties" herangezogen, welche den Richtlinien des GUM folgt.
% Alle konkreten Unsicherheitsformeln stehen weiter unten.
Im diesem Experiment wurden konkrete Unsicherheiten nur in Diagrammen verwendet. 
Für Unsicherheiten in graphischen Fits wurden die $y$-Unsicherheiten beachtet und die Methode der kleinsten Quadrate angewandt.
Dafür steht in der Bibliothek die Methode "scipy.optimize.curve\_fit()" zur Verfügung.

Für digitale Messungen wird eine Unsicherheit von $u(X) = \frac{\Delta X}{2\sqrt{3}}$ angenommen, bei analogen eine von $u(X) = \frac{\Delta X}{2\sqrt{6}}$.

\begin{description}
	\item[Strommessung] Das Multimeter für die Strommessung zeigte digitale Werte mit einer Stellengenauigkeit von $\Delta I = \SI{0.01}{\ampere}$ an.
	
	\item[Lichtsensor] Das Abtastrate des Lichtsensors wurde auf $T = \SI{5}{\second}$ eingestellt, um Schwankungen zu minimieren.
	Auch wenn der Sensor an sich deutlich genauere Werte lieferte, ist die Ungenauigkeit durch Beobachtungen der Schwankungen durch äußere Einflüsse bei gleichem Spulenstrom auf $\Delta U = \SI{0.005}{\lux}$ abgeschätzt.
	
	\item[Hall-Sonde] Die gemessene Magnetfeldstärke hat eine digitale Unsicherheit $\Delta B = \SI{0.01}{\tesla}$.
\end{description}
