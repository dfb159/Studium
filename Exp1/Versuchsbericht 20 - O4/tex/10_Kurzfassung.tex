\section{Kurzfassung}

	Dieser Bericht beschäftigt sich mit der Untersuchung des Magneto-Optischen Kerr-Effekts.
	Der Effekt beschreibt die Änderung der Polarisation von Licht in einem ferromagnetischem Material, welche von der Magnetisierung dieses Materials abhängt.
	Dazu wird eine Co/Pt-Probe in ein äußeres Magnetfeld gesetzt und mit einem Laser bestrahlt.
	Über einen Analysator wird dann an einer Photodiode die Lichtintensität gemessen.
	Ziel dieser Untersuchung ist es die Ergebnisse in den höheren Kontext, der dem Magneto-Optischen Kerr-Effekt zu Grunde liegt, einordnen zu können.
	
	Die Ergebnisse ließen dies zu.
	Es konnte eine Hystereseschleife aufgenommen werden, die den Magnetisierungsprozess der Co/Pt-Probe in guter Näherung darstellt und somit auch die Polarisationsänderung welche der Magneto-Optische Kerr-Effekt beschreibt.