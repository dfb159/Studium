\subsection{Unsicherheiten}\label{VGuD}

Jegliche Unsicherheiten werden nach GUM bestimmt und berechnet.
Die Gleichungen dazu finden sich in \ref{fig:GUM_combine} und \ref{fig:GUM_formula}.
Für die Unsicherheitsrechnungen wurde die Python Bibliothek "uncertainties" herangezogen, welche den Richtlinien des GUM folgt.

Für digitale Messungen wird eine Unsicherheit von $u(X) = \frac{\Delta X}{\sqrt{3}}$ angenommen, bei analogen eine von $u(X) = \frac{\Delta X}{\sqrt{6}}$.

\begin{description}
	\item[Abtastrate] g
\end{description}

In Diagrammen mit vielen Messpunkten wurden aus Gründen der Übersichtlichkeit nicht für jeden Punkt Unsicherheitsbalken gezeichnet.
\begin{figure}[h]
	\begin{equation*}
		x = \sum_{i=1}^{N} x_i
		;\quad
		u(x) = \sqrt{\sum_{i = 1}^{N} u(x_i)^2}
	\end{equation*}
	\caption{Formel für kombinierte Unsicherheiten des selben Typs nach GUM.}
	\label{fig:GUM_combine}
\end{figure}

\begin{figure}[h]
	\begin{align*}
		f = f(x_1, \dots , x_N)
		;\quad
		u(f) = \sqrt{\sum_{i = 1}^{N}\left(\pdv{f}{x_i} u(x_i)\right) ^2}
	\end{align*}
	\caption{Formel für sich fortpflanzende Unsicherheiten nach GUM.}
	\label{fig:GUM_formula}
\end{figure}

