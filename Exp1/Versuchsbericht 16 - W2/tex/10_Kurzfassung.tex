\section{Kurzfassung}

	Dieser Bericht beschäftigt sich mit der Untersuchung des Adiabatenexponent $\kappa$ von Gasen.
	Zur Bestimmung dieses Exponenten werden im Folgenden zwei verschiedene Methoden betrachtet.
	Zum einen die Bestimmung von $\kappa$ nach der Methode von Rüchardt-Flammersfeld und zum anderen nach der von Clément-Desormes.
	Aus den bei diesen Methoden ermittelten Werten für $\kappa$, lassen sich zudem die Freiheitsgrade der verwendeten Gase bestimmen.
	Ziel der Untersuchung sind Ergebnisse für $\kappa$ bzw. der Anzahl der Freiheitsgrade, die in guter Näherung mit den Literaturwerten übereinstimmten. 
	
	Die bei der Auswertung dieses Versuchs auftretenden Ergebnisse spiegeln eine solche Übereinstimmung mit %TODO Erg