\section{Kurzfassung}

	Dieser Bericht beschäftigt sich mit der Untersuchung des Adiabatenexponent $\kappa$ von Gasen.
	Zur Bestimmung dieses Exponenten werden im Folgenden zwei verschiedene Methoden betrachtet.
	Zum einen die Bestimmung von $\kappa$ nach der Methode von Rüchardt-Flammersfeld und zum anderen nach der von Clément-Desormes.
	
	Ziel der Untersuchung sind Ergebnisse für $\kappa$ die in guter Näherung mit den Literaturwerten übereinstimmten. 
	Die bei der Auswertung dieses Versuchs auftretenden Ergebnisse beliefen sich auf $\kappa = 1,426\pm 0,017$ für Luft $\kappa = 1,633\pm 0,017$ für Argon und $\kappa = 1,289\pm 0,014$ für Kohlenstoffdioxid nach der Methode von Rüchardt-Flammersfeld, sowie $\kappa = 1,3907\pm 0,005$ für Luft bei der Methode nach Clément-Desormes.
	Keiner dieser Werte weicht mehr als 3,5\% von dem Erwartungswert, weswegen die Ziele dieser Untersuchung somit erreicht wurden.