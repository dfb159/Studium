\section{Schlussfolgerung}

	Abschließend lässt sich bestätigen, dass die Ziele der Untersuchung erreicht wurden.
	Die ermittelten Werte für den Adiabatenexponenten $\kappa$ lagen alle für beide Methoden nahe den Erwartungen.
	Sie beliefen sich auf $\kappa = 1,426\pm 0,017$ für Luft $\kappa = 1,633\pm 0,017$ für Argon und $\kappa = 1,289\pm 0,014$ für Kohlenstoffdioxid nach der Methode von Rüchardt-Flammersfeld, sowie $\kappa = 1,3907\pm 0,005$ für Luft bei der Methode nach Clément-Desormes.
	Eine Abweichung von über 3,5\% ist bei keinem Wert aufgetreten, weswegen eine Wiederholung des Versuches nicht nötig ist.
	Für eine Nachstellung der Methode nach Rüchardt-Flammersfeld, wäre eine Art "Tally counter" zum Zählen der Schwingungen hilfreich, um dies nicht in Gedanken durchführen zu müssen.
	 