\section{Methoden}
	
	Dieser Abschnitt befasst sich mit dem Aufbau des Versuches und den dabei auftretenden Unsicherheiten.
	
	\subsection{Aufbau und Funktionsweise}	
		
		Der Versuchsaufbau besteht 
		
	\subsection{Unsicherheiten}
	
		Jegliche Unsicherheiten werden nach GUM bestimmt und berechnet\footnote{Die Gleichungen dazu finden sich im Anhang unter \ref{fig:GUM_combine}, \ref{fig:GUM_formula}.}.
		Für die Unsicherheitsrechnungen wurde die Python Bibliothek "uncertainties" herangezogen, welche den Richtlinien des GUM folgt.
	
		Für digitale Messungen wird eine Unsicherheit von $u(X) = \frac{\Delta X}{\sqrt{3}}$ angenommen, bei analogen eine von $u(X) = \frac{\Delta X}{\sqrt{6}}$.
		
		\begin{description}
			\item[Abtastrate] g
		\end{description}
		
		In Diagrammen mit vielen Messpunkten wurden aus Gründen der Übersichtlichkeit nicht für jeden Punkt Unsicherheitsbalken gezeichnet.
		
\section{Durchführung und Datenanalyse}	
	
\section{Diskussion}
	
	Nun stellt sich die Frage, ob die Ziele der Untersuchung erreicht wurden.
	Dazu wird zunächst betrachtet, ob
