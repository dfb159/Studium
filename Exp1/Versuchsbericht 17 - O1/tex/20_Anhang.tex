\subsection{Unsicherheiten}\label{VGuD}

Jegliche Unsicherheiten werden nach GUM bestimmt und berechnet.
Die Gleichungen dazu finden sich in \ref{fig:GUM_combine} und \ref{fig:GUM_formula}.
Für die Unsicherheitsrechnungen wurde die Python Bibliothek "uncertainties" herangezogen, welche den Richtlinien des GUM folgt.
Alle konkreten Unsicherheitsformeln stehen weiter unten.
Für Unsicherheiten in graphischen Fits wurden die $y$-Unsicherheiten beachtet und die Methode der kleinsten Quadrate angewandt.
Dafür steht in der Bibliothek die Methode "scipy.optimize.curve\_fit()" zur Verfügung.

Für digitale Messungen wird eine Unsicherheit von $u(X) = \frac{\Delta X}{2\sqrt{3}}$ angenommen, bei analogen eine von $u(X) = \frac{\Delta X}{2\sqrt{6}}$.

\begin{description}
	\item[Bodenlinien] Auf der magnetischen Unterlage sind zum genauen platzieren und ausrichten der Komponenten Gitterlinien aufgetragen.
	Diese hatten einen regelmäßigen Abstand und eine analoge Unsicherheit von $\Delta h = \SI{2}{\centi\meter}$.
	
	\item[Messleiste] Die Messleiste wurde senkrecht zum vom Laser ausgesendeten Lichtstrahls ausgerichtet.
	Es konnten Werte analog auf $\Delta x = \SI{0.1}{\milli\meter}$ genau abgelesen werden.
	
	\item[Grad] Auf der Gradskala des Wasserbeckens konnten Winkel auf $\Delta \varphi = \SI{1}{\degree}$ abgelesen werden.
\end{description}

\begin{figure}[ht]
	\begin{equation*}
		x = \sum_{i=1}^{N} x_i
		;\quad
		u(x) = \sqrt{\sum_{i = 1}^{N} u(x_i)^2}
	\end{equation*}
	\caption{Formel für kombinierte Unsicherheiten des selben Typs nach GUM.}
	\label{fig:GUM_combine}
\end{figure}

\begin{figure}[ht]
	\begin{align*}
		f = f(x_1, \dots , x_N)
		;\quad
		u(f) = \sqrt{\sum_{i = 1}^{N}\left(\pdv{f}{x_i} u(x_i)\right) ^2}
	\end{align*}
	\caption{Formel für sich fortpflanzende Unsicherheiten nach GUM.}
	\label{fig:GUM_formula}
\end{figure}

\begin{figure}[ht]
	\begin{align*}
		\tan \delta_m &= \frac{h}{x} \Leftrightarrow \delta_m = \arctan \frac{h}{x}\\
		u(\delta_m) &= \frac{1}{1 + \frac{h^2}{x^2}} \frac{h}{x} 
		\sqrt{\frac{u^2(h)}{h^2} + \frac{u^2(x)}{4x^2}}
	\end{align*}
	\caption{Unsicherheitsformel des Ablenkwinkels nach dem Prisma.}
	\label{unc:winkel}
\end{figure}

\begin{figure}[ht]
	\begin{align*}
		n &= \frac{\sin[(\delta_m + \alpha) / 2]}{\sin [\alpha / 2]}\\
		u(n) &= \frac{\cos[(\delta_m + \alpha) / 2]}{\sin[\alpha / 2]} \frac{u(\delta_m)}{2}
	\end{align*}
	\caption{Unsicherheitsformel des Brechungsindexes von dem Prisma (Flintglas).}
	\label{unc:prisma}
\end{figure}

\begin{figure}[ht]
	\begin{align*}
		n_1 \sin \alpha_1 = n_2 \sin \alpha_2 \Leftrightarrow n_2 &= n_1 \frac{\sin \alpha_1}{\alpha_2}\\
		u(n_2) &= n_1 \frac{\sin \alpha_1}{\sin \alpha_2} \sqrt{\cot^2 \alpha_1 + \cot^2 \alpha_2} u(\alpha)\\
		&= n_2 \cdot \sqrt{\cot^2 \alpha_1 + \cot^2 \alpha_2} u(\alpha)
	\end{align*}
	\caption{Unsicherheitsformel für den Brechungsindex von destiliertem Wasser nach Snellius. $n_1$ ist dabei der Brechungsindex von Luft $n_1 = n_\text{L} \approx 1$. Beide Winkel haben die gleiche Unsicherheit $u(\alpha_1) = u(\alpha_2) = u(\alpha)$.}
	\label{unc:gitter}
\end{figure}

\begin{figure}[ht]
	\begin{align*}
		f_\text{konkav} &= f_\text{konvex} - \Delta h\\
		u(f_\text{konkav}) &= \sqrt{2} u(h)
	\end{align*}
	\caption{Unsicherheitsformel für die Brennweite der konkaven Linse.\\ Dabei ist $u(f_\text{konkav}) = u(Delta h) = u(h)$.}
	\label{unc:differenz}
\end{figure}
