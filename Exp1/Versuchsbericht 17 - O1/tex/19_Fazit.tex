\section{Schlussfolgerung}

	Im Großen und Ganzen lässt sich behaupten, dass das Ziel dieser Untersuchung, die Beobachtung anhand der Theorie zu begründen und teilweise die Bestimmung der Brechungsindizes  gelungen ist.
	Nur bei dem Prisma liegt eine leicht größere Abweichung von der Erwartung vor.
	Da aber auch an dieser Stelle höhere $n$ für kleinere $\lambda$ beobachtet wurden lässt sich kein Widerspruch zur Theorie finden und die restliche Untersuchung unterstützt diese sogar.
	Eine Wiederholung des Versuches ist demnach nicht notwendig, für eine Nachstellung wäre jedoch eine genauere Methode zur Bestimmung der Brennweiten ratsamer und Vorsicht bei dem Prisma geboten, da auch leichte Drehung einen großen Einfluss auf die Ablenkung haben kann.  