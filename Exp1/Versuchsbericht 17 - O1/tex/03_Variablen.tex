% Autor: Simon May
% Datum: 2016-10-13
% Der Befehl \newcommand kann auch benutzt werden, um „Variablen“ zu definieren:

% Nummer laut Praktikumsheft:
\newcommand*{\varNum}{O1}
% Name laut Praktikumsheft:
\newcommand*{\varName}{Geometrische Optik}
% Datum der Durchführung (Format: JJJJ-MM-TT):
\newcommand*{\varDatum}{30.05.2018}
% Autoren des Protokolls:
\newcommand*{\varAutor}{Alex Oster, Jonathan Sigrist}
\newcommand*{\varNameA}{Alex Oster}
\newcommand*{\varNameB}{Jonathan Sigrist}
% Nummer der eigenen Gruppe:
\newcommand*{\varGruppe}{Gruppe Mi 11}
% E-Mail-Adressen der Autoren (kommagetrennt ohne Leerzeichen!):
\newcommand{\varEmail}{a\_oste16@uni--muenster.de,j\_sigr01@uni--muenster.de}
\newcommand{\varEmailA}{a\_oste16@uni--muenster.de}
\newcommand{\varEmailB}{j\_sigr01@uni--muenster.de}
%betreuer Name
\newcommand{\varBetreuer}{\normalsize betreut von Johannes Feldmann} 
% E-Mail-Adresse anzeigen (true/false):
\newcommand*{\varZeigeEmail}{true}
% Kopfzeile anzeigen (true/false):
\newcommand*{\varZeigeKopfzeile}{true}
% Inhaltsverzeichnis anzeigen (true/false):
\newcommand*{\varZeigeInhaltsverzeichnis}{true}
% Literaturverzeichnis anzeigen (true/false):
\newcommand*{\varZeigeLiteraturverzeichnis}{true}
