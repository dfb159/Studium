\section{Methoden}
	
	Dieser Abschnitt befasst sich mit dem Aufbau des Versuches und den dabei auftretenden Unsicherheiten.
	
	\subsection{Aufbau}	
		
		Der Aufbau, wie er in in Abb. \ref{label} dargestellt ist, besteht im Wesentlichen aus einem Geiger-Müller-Zählrohr und einem radioaktiven Präparat, welches in geringem Abstand (ca. \SI{10}{\centi\meter}) von dem Zählrohr steht, sodass kleine Platten aus verschiedenen Stoffen dazwischen gesetzt werden können.
		Mit Hilfe der Apparatur die an das Zählrohr geschlossen ist, lassen sich die Anzahl der Ereignisse und die vergangene Zeit einfach von einem Digitaldisplay ablesen.
		Zudem lässt sich dort die Spannung an dem Zählrohr einstellen.
		Bei den verwendeten radioaktiven Präparaten handelt es sich um den $\beta$-Strahler $^{90}$Sr und den $\gamma$-Strahler $^{137}$Cs.
		Zur Messung der natürlichen Strahlung wird kein Präparat vor dem Zählrohr platziert.
				
	\subsection{Unsicherheiten}
	
		Jegliche Unsicherheiten werden nach GUM bestimmt und berechnet\footnote{Die Gleichungen dazu finden sich im Anhang unter \ref{fig:GUM_combine}, \ref{fig:GUM_formula}.}.
		Für die Unsicherheitsrechnungen wurde die Python Bibliothek "uncertainties" herangezogen, welche den Richtlinien des GUM folgt.
	
		Für digitale Messungen wird eine Unsicherheit von $u(X) = \frac{\Delta X}{\sqrt{3}}$ angenommen, bei analogen eine von $u(X) = \frac{\Delta X}{\sqrt{6}}$.

\section{Durchführung und Datenanalyse}
		
	Zur Bestimmung der Zählrohrcharakteristik wurde der $\beta$-Strahler $^{90}$Sr verwendet und für verschiedene Spannungen die Anzahl der Ereignisse nach \SI{94}{\second} aufgetragen.
	Die Zeit (oder die Zahl der Ereignisse) wurde bei allen Messungen so gewählt, dass die relative Unsicherheit unter 4\%, für alle Messwerte liegt.
	Eine Darstellung der Messwerte ist der Abb. \ref{label} zu entnehmen.
	Diese Kennlinie zeigt, dass zur Messung der Radioaktivität mindestens eine Spannung von x anliegen muss, diese nennt sich Einsatzspannung.
	Auch das charakteristische Plateau ist der Kurve zu entnehmen.
	Zudem ließen sich nur Werte bis zu \SI{500}{\volt} einstellen, da höhere Spannungen zur Beschädigung oder gar Zerstörung des Zählrohrs führen könnten.
	
	Zur Messung der natürlichen Radioaktivität wurde das radioaktive Präparat entfernt und 200 mal die Anzahl der Ereignisse innerhalb von \SI{10}{\second} aufgenommen.
	Aus dieser Verteilung folgen der Mittelwert $ $ wie auch die empirische Standardabweichung $ $.
	Diagramme der absoluten und relativen Häufigkeitsverteilung sind in Abb. \ref{label} vorzufinden.
	Zur Bestimmung der Werte wurde die Poisson-Verteilung herangezogen. % TODO iwie
	Mit Hilfe der mittleren Untergrundaktivität, die aus der natürlichen Radioaktivität hervorgegangen ist, ließ sich nun eine Korrektur, für die folgenden Messungen durchführen.
	
	Zur Bestimmung des Absorptionskoeffizienten $\mu_\gamma$ von Blei wurde die Impulsrate $a_\gamma (x)$ des $\gamma$-Präparats $^{137}$Cs in Abhängigkeit der Schichtdicke des Blei-Absorbers aufgenommen.	
	Hierbei wurde die Zeit gemessen, die benötigt wurde um ca. 650 Ereignisse in dem Zählrohr auszulösen, um die Impulsrate mit einer relativen Unsicherheit unter 4\% aufzunehmen.
	Nach jeder Messreihe wurde eine weitere Platte hinzugefügt und eine neue gestartet, sodass die Impulsrate in Abhängigkeit der Schichtdicke aufgetragen werden konnte.
	Dazu standen vier Blei-Platten zur Verfügung.
	Abb. \ref{label} stellt das Verhältnis logarithmisch aufgetragen dar.
	Da es sich bei steigender Schichtdicke um einen exponentiellen Abfall der Ereignisse handeln sollte, lässt sich der Absorptionskoeffizient aus der Steigung des Graphen bestimmen.
	Diese beläuft sich bei dem Blei auf $\mu_\gamma = $
	
	Analog verlief die Messung der Impulsraten $a_\beta (x)$ des $\beta$-Präparats $^{90}$Sr in Abhängigkeit der Schichtdicken von Aluminium, Plexiglas und Gummi.
	Für die letzteren beiden, wurde die Messung jedoch nur für je eine Schicht durchgeführt.
	Eine graphische Darstellung der Messung ist in Abb. \ref{label} vorzufinden.
	Die sich dadurch berechneten Absorptionskoeffizienten sind in Tab. \ref{label} verzeichnet. 
	
	%TODO Bilder, Werte, Tabelle
	
\section{Diskussion}
	
	% Aufnahme der Zählrohr-Charakteristik
	% Aufnahme der natürlichen Strahlung
	% Gamma-Strahler, mit Blei
	% Beta-Strahler mit Aluminium, Plexiglas und Gummi , Blei als Absorber?
	