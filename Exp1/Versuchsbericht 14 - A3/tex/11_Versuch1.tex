\section{Methoden}
	
	Dieser Abschnitt befasst sich mit dem Aufbau des Franck-Hertz-Versuches, so wie auch den dabei auftretenden Unsicherheiten.
	
	\subsection{Aufbau}	
		
	\subsection{Unsicherheiten}
	
		Jegliche Unsicherheiten werden nach GUM bestimmt und berechnet\footnote{Die Gleichungen dazu finden sich im Anhang unter \ref{fig:GUM_combine}, \ref{fig:GUM_formula}.}.
		Bei der Rechnung mit Werten mit Unsicherheiten wurde die Python Bibliohek "uncertainties" herangezogen, welche ihrerseits nach obigen Regeln rechnet.
	
		Für digitale Messungen wird eine Unsicherheit von $u(X) = \frac{\Delta X}{\sqrt{3}}$, bei analogen Messungen eine von $u(X) = \frac{\Delta X}{\sqrt{6}}$ angenommen.

\section{Durchführung und Datenanalyse}

\section{Diskussion}
