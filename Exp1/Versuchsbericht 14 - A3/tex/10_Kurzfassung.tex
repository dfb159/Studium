\section{Kurzfassung}

Dieser Bericht beschäftigt sich mit der Untersuchung der Absorption von $\beta$- und $\gamma$-Strahlung.
Dazu wird zunächst das Geiger-Müller-Zählrohr betrachtet, welches zur Messung der radioaktiven Strahlung dient.
Im Folgenden wird die Kennlinie des Zählrohrs aufgenommen, die natürliche Radioaktivität gemessen und die Absorptionskoeffizienten von Blei, Aluminium, Plexiglas und Gummi bestimmt.
Zur Bestimmung der Koeffizienten werden die Zeiten gemessen, die für verschiedene Dicken an Absorptionsmaterial benötigt werden, um eine bestimmte Anzahl an Anregungen in dem Zählrohr zu erreichen.
Bei dem Blei wird der $\gamma$-Strahler $^{137}$Cs verwendet und für die restlichen Absorptionskoeffizienten, sowie auch zur Aufnahme der Zählrohrcharakteristik der $\beta$-Strahler $^{90}$Sr.
Ziel dieser Untersuchung ist die Aufnahme einer den Erwartungen entsprechenden Zählrohrcharakteristik, sowie die Übereinstimmung der ermittlten Absorptionskoeffizienten mit den Literaturwerten.

Die Ergebnisse stimmen weitgehend mit den Zielen der Untersuchung überein.
Für die aufgenommene Zählrohrcharakteristik ist der zu erwartende Verlauf zu erkennen und der Massenabsorptionskoeffizient von Blei weicht lediglich um 9,4\% von dem Literaturwert ab.
Bei dem Aluminium ist dies mit einer Abweichung von 82,1\% jedoch nicht der Fall.
An dieser Stelle versagt die Untersuchung.