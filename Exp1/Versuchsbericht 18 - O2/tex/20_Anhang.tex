\subsection{Unsicherheiten}\label{VGuD}

Jegliche Unsicherheiten werden nach GUM bestimmt und berechnet.
Die Gleichungen dazu finden sich in \ref{fig:GUM_combine} und \ref{fig:GUM_formula}.
Für die Unsicherheitsrechnungen wurde die Python Bibliothek "uncertainties" herangezogen, welche den Richtlinien des GUM folgt.
Alle konkreten Unsicherheitsformeln stehen weiter unten.
Für Unsicherheiten in graphischen Fits wurden die $y$-Unsicherheiten beachtet und die Methode der kleinsten Quadrate angewandt.
Dafür steht in der Bibliothek die Methode "scipy.optimize.curve\_fit()" zur Verfügung.

Für digitale Messungen wird eine Unsicherheit von $u(X) = \frac{\Delta X}{2\sqrt{3}}$ angenommen, bei analogen eine von $u(X) = \frac{\Delta X}{2\sqrt{6}}$.

\begin{description}
	\item[Maßstäbe]	Jegliche Maßstäbe waren gleich skaliert und wiesen daher gleiche analoge Unsicherheiten von $\Delta x = \SI{0.1}{\centi\meter}$ auf.
	Teilweise wurden Strecken aus Teilstrecken zusammengesetzt. Die Unsicherheit erhöht sich dann um den Faktor $\sqrt{n}$, wobei n die Anzahl an Teilstrecken ist.
	
	\item[Spannung] Die Spannung wurde mit einem Multimeter gemessen.
	Dieses hatte eine Digitalanzeige mit einer Stellengenauigkeit von $\Delta U = \SI{10}{\milli\volt}$ bei Betrieb bis \SI{20}{Volt} bzw. $\Delta U = \SI{1}{\milli\volt}$ bis \SI{6}{\milli\volt}.
	
	\item[Winkel] Die aufgezeichneten Winkelstriche hatten einen Abstand von $\Delta \phi = \SI{1}{\degree}$.
	
\end{description}

\begin{figure}[ht]
	\begin{equation*}
		x = \sum_{i=1}^{N} x_i
		;\quad
		u(x) = \sqrt{\sum_{i = 1}^{N} u(x_i)^2}
	\end{equation*}
	\caption{Formel für kombinierte Unsicherheiten des selben Typs nach GUM.}
	\label{fig:GUM_combine}
\end{figure}

\begin{figure}[ht]
	\begin{align*}
		f = f(x_1, \dots , x_N)
		;\quad
		u(f) = \sqrt{\sum_{i = 1}^{N}\left(\pdv{f}{x_i} u(x_i)\right) ^2}
	\end{align*}
	\caption{Formel für sich fortpflanzende Unsicherheiten nach GUM.}
	\label{fig:GUM_formula}
\end{figure}

\begin{figure}[ht]
	\begin{align*}
		x_\text{Quell} = \frac{y_2 - y_1}{m_1 - m_2}
		;\quad
		u(x_\text{Quell}) = x_\text{Quell} \sqrt{\frac{u^2(y_1) + u^2(y_2)}{(y_2 - y_1)^2} + \frac{u^2(m_1) + u^2(m_2)}{(m_2 - m_1)^2}}
	\end{align*}
	\caption{Formel für die $x$-Koordinate des virtuellen Quellflecks. $y_{1,2}$ und $m_{1,2}$ sind dabei die aus dem Fit entnommenen Werte für die beiden Geraden mit $f(x) = m x + y$.}
	\label{unc:xquell}
\end{figure}

\begin{figure}[ht]
\begin{align*}
	\phi = \arctan \abs m_1 + \arctan \abs m_2
	;\quad
	u(\phi) = \sqrt{\frac{u^2(m_1)}{\left(1 + m_1^2 \right) ^2} + \frac{u^2(m_2)}{\left(1 + m_2^2 \right) ^2}}
\end{align*}
\caption{Formel für die Aufweitung der Quelle. $m_{1,2}$ sind dabei die aus dem Fit entnommenen Werte für die beiden Geraden mit $f(x) = m x + y$.}
	\label{unc:quellWinkel}
\end{figure}

\begin{figure}[ht]
\begin{align*}
	\lambda = \frac{2(x_4 - x_1)}{3}
	;\quad
	u(\lambda) = \frac{2}{3} \sqrt{u^2(x_1) + u^2(x_4)}
\end{align*}
\caption{Formel für Wellenlänge der Mikrowellen gemittelt über vier Minima.}
	\label{unc:wellenlaenge}
\end{figure}


\begin{figure}[ht]
	\begin{align*}
		n = \frac{\sin \varphi_2}{\sin \varphi_1}
		;\quad
		u(n) = n \sqrt{\cot^2 \varphi_1 u^2(\varphi_1) + \cot^2 \varphi_2 u^2(\varphi_2)}
	\end{align*}
	\caption{Formel für Wellenlänge der Mikrowellen gemittelt über vier Minima.}
	\label{unc:nIndexPVC}
\end{figure}
