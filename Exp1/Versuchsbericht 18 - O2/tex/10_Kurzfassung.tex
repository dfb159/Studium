\section{Kurzfassung}

	Dieser Bericht beschäftigt sich mit der Untersuchung von Mikrowellen.
	Dazu dienen mehrere Teilversuche, die elektromagnetische Phänomene mit Mikrowellen deutlicher darstellen als bei der Verwendung von elektromagnetischer Wellen kleinerer Wellenlänge.
	Untersucht werden die Strahlendivergenz, stehende Wellen, das Snellius'sche Brechungsgesetz, insbesondere des Sonderfalls der frustrierten Totalreflexion, sowie auch der Reflexion an einem Gitter nach Bragg.
	Ziel dieser Untersuchung ist eine Übereinstimmung der Mess- und ermittelten Größen mit der Theorie bzw. Literaturwerten, falls welche gegeben sind.
	Im Wesentlichen wurden diese Ziele erreicht, da eine Übereinstimmung mit der Theorie für fast alle ermittelten Werte und Messungen vorlag.
	Die ermittelte Wellenlänge von $\lambda = \SI{}{}$ liegt innnerhalb des Spektrums von Mikrowellen und bei dem nach Snellius bestimmten Brechungsindex für PVC $n_\text{PVC} = \SI{1.66+-0.17}{}$ ließ der Literaturwert von $n_\text{PVC,Lit} = \SI{1,54}{}$ sich innerhalb einer Unsicherheit wiederfinden und somit liegt auch hier eine Übereinstimmung vor.
	Nur bei dem Gitterabstand $d$ bei der Bragg'schen Reflexion ließ sich keine Aussage treffen, da zwei wesentlich verschiedene Werte sich für zwei unterschiedliche Glanzwinkel ergaben und beide Gitterabstände nicht nahe dem gemessenem lagen (ca. 16\% Abweichung von dem Näheren).