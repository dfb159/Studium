\section{Kurzfassung}

	Dieser Bericht beschäftigt sich mit der Untersuchung von geometrischer Optik.
	Dazu dienen zwei Versuche.
	Bei dem ersten handelt es sich um einen von dem Betreuer durchgeführten Demonstrationsversuch, bei dem die Brechung von Laserlicht in inhomogenem Salzwasser untersucht wird.
	Der zweite Versuch besteht darin verschiedene optische Elemente in einen Laserstrahl einzubringen und die Veränderung des Strahlengangs zu untersuchen.
	Ziel des Ganzen ist alle Beobachtungen über die Theorie der geometrischen Optik erklären zu können und für die ermittelten Brechungsindizes $n$ einiger verwendeten Materialien eine Übereinstimmung mit den Literaturwerten zu finden. 
	
	Die Beobachtungen ließen sich alle begründen und für die Brechungsindizes ergaben sich Werte von  $n_\text{Prisma,rot} = \SI{1.525+-0.013}{}$ und $n_\text{Prisma,blau} = \SI{1.557+-0.013}{}$, sowie $n_\text{Wasser,rot} = \SI{1.317+-0.018}{}$ für den roten Laser und $n_\text{Wasser,blau} = \SI{1.357+-0.031}{}$.
	Bei den ersteren beiden liegt die Abweichung etwa 7\% von dem Literaturwert und ca. 1\% von den letzteren.
	Da die Beobachtungen der Wellenlängenabhängigkeit auch bei den 7\% Abweichungen übereinstimmen lässt sich kein Widerspruch zur Theorie formulieren und die restlichen Ergebnisse stützen diese sogar.