\documentclass[11pt,a4paper,titlepage, ngerman]{article}

\usepackage[utf8]{inputenc}	
\usepackage[T1]{fontenc}	
\usepackage{ngerman}			
\usepackage{lmodern}			
\usepackage{graphicx}			
\usepackage{url}				
\usepackage{siunitx}
\usepackage{amsmath}			
\usepackage{subcaption}
\usepackage{wrapfig}

\begin{document}

	\begin{titlepage}
		\centering
		{\scshape\LARGE Versuchsbericht zu \par}
		\vspace{1cm}
		{\scshape\huge E5 -- Magnetische Suszeptibilität\par}
		\vspace{2.5cm}
		{\LARGE Gruppe 10 Mi\par}
		\vspace{0.5cm}
		{\large Alex Oster (E-Mail: a\_oste16@uni--muenster.de) \par}
		{\large Jonathan Sigrist (E-Mail: j\_sigr01@uni--muenster.de ) \par}
		\vfill
		durchgeführt am 15.11.2017\par
		betreut von\par
		{\large Phillip \textsc{Eickholt}}		
		\vfill	
		{\large \today\par}
	\end{titlepage}
		
	\tableofcontents
		
	\newpage
	
	\section{Kurzfassung}
		
		%TODO
		
		%Kurzfassung, wenn Rest fertig für besseren talk		

	\newpage	
	\section{Arten von Magnetismus}
				
		%TODO
		
		%gibt verschiedene Arten, wie Stoff auf äußeres Magnetfeld reagieren
		
		\subsection{Ferromagnetismus}
		
			%TODO
			
			%Anziehung/Abstoßung, weil
		
		\subsection{Paramagnetismus}
			
			%TODO
			
			%Anziehung (schwach), weil
			
		\subsection{Diamagnetismus}
		
			%TODO
			
			%Abstoßung (sehr schwach), weil
			
	\section{Versuch 1: Demonstrationsversuch} 
		
		%TODO
		
		%Durchgeführt von Betreuer
		%Oberflächenbeugung von Flüssigkeiten (durch aüßeres Magnetfeld)
		%Versuchsbeschreibung(kurz) (Laser/Wasser/Manganwasauchimmer/Magnet) 	
		
		\subsection*{Methoden} 
		
		%TODO
		
		%Detaillierte Beschreibung des Aufbaus
		%Berechnung der Auslenkung + Bild
					
		\subsection*{Fermi-Abschätzung}
			
			%TODO
			
			%Einsetzen der Schätzwerte in die in Methoden beschriebene Rechnung
			%Angabe des Ergebnisses (Größenordnung)
			
		\subsection*{Schlussfolgerung}
						
			%TODO
			
			%Ergebnis-Größenordnung wirkt realistisch (klein) weil diamag. / verhältnismäßig groß, weil paramag (Manganwasauchimmer)
			
	\section{Versuch 2: Volumensuszeptibilität}		
		
		%TODO
		
		%~ Wie verhält sich das gemessene Gewicht, wenn ein starker Magnet über der Waage angebracht ist			
		
		\subsection*{Methoden} 
		
			%TODO
			
			%Magnetismuswaage beschreiben (Schwenkkran immer wegdingensen um stuff zu vermeiden)
			%Bild
			%Materialien: AL, pyrolitisches Graphit, Glas(/nur Halterung)
			%Abdingensen mit Kunsstoffplatte (1mm)
			
			%Ungenauigkeit der Waage aus Digitalanzeige und 0,001g
					
		\subsection*{Messung}
			
			%TODO
			
			%Ergebnisse andingensen (mit Ungenauigkeit)
			
		\subsection*{Schlussfolgerung}	
			
			%TODO
			
			%Glas -> Dia? (Si dia, O para) 
			%Al -> Para
			%C -> Dia
		
	\section{Versuch 3: Fallender Neodymmagnet}		
	
		%TODO
		
		%Lassen Magnet durch ??-Röhre fallen, mit und ohne Schlitz und beobachten die Effekte
		% Schönes Bild
		
		\subsection*{Beobachtung}
			
			%TODO
			
			%ohne Schlitz: Magnet fällt langsam
			%mit: fällt schneller, jedoch langsamer als ohne Rohr
			
		\subsection*{Schlussfolgerung}	
		
			%TODO
			
			%Magnetfeld induziert Kreisstrom -> entgegengesetztes B-Feld ist langsamer
			%Schlitz: Kreisstrom nicht so noice -> schwächeres Gegenfeld -> fällt schneller
					
	\section{Versuch 4: ??} %Unterschied bei verschiedenen Formen AL bei äußerem B-Feld 		
	
		%TODO
		
		%Magnet von verschiedenen Richtungen/Geschwindigkeiten an AL-Platte/Kamm
		%Bild
		%Vergleich
		
		\subsection*{Beobachtung}
		
			%TODO
			
			%Effekte bei Kamm deutlich schwächer
			
		\subsection*{Schlussfolgerung}	
		
			%TODO
			
			%Form von Kamm verhindert Bildung von "größeren" Wirbelströmen -> schwächeres Feld
		
	\begin{thebibliography}{1}
			
		%TODO
		
	\end{thebibliography}	
			
\end{document} 

%figure sample
%\begin{figure}
%\includegraphics[width=\textwidth]{}
%\caption{}
%\label{}
%\end{figure}
