\section{Untersuchung von Akkumulatorzellen} 

% TODO
% Dies ist ein Teilversuch.
% Der Abschnitt sagt grundsätzliches über den konkreten Versuch aus.
% Es werden relevante physikalische Effekte qualitativ beleuchtet.
% Es wird das Ziel des Versuchs angegeben.

% Die Ergebnisse werden kurz, aber mit konkreten Werten erwähnt.
% Dann werden diese in den Kontext kurz eingebunden.

\subsection{Methoden}

\subsubsection{Aufbau}

Der Aufbau dieses Versuchs beschränkt sich auf einen simplen Schaltkreis, welcher zunächst nur aus einer Akkumulatorzelle und einem regulierbaren Widerstand $R_a$ besteht. 
Zusätzlich dazu ist ein Multimeter an die Akkumulatorzelle geschlossen, sodass die dort anliegende Spannung gemessen werden kann. 
Mit diesem Aufbau wird zuerst die Leerlaufspannung $U_0$ der Akkumulatorzelle und dann der Innenwiderstand $R_i$ dieser bestimmt.
Dazu wird die Klemmspannung $U_{kl}$ gemessen und in Abhängigkeit des elektrischen Stroms $I$ gesetzt, welcher sich durch die Spannung und dem anliegenden Widerstand bestimmen lässt.

Dieser Vorgang wird dann für drei in Reihe- und drei parallel geschaltete Akkumulatorzellen wiederholt. 

\subsubsection{Unsicherheiten}

Bei diesem Versuch treten lediglich die Unsicherheit des Multimeters und des Lastwiderstands $R_a$ auf. 
Da das Multimeter eine analoge Darstellung der Messwerte verwendet und sich je nach verwendeter Größenordnung mit einer Genauigkeit von \SI{1}{V} bzw. \SI{0,1}{V} ablesen ließ, werden die zugehörigen Unsicherheiten über eine Dreiecksverteilung bestimmt.
Für die Unsicherheit des Widerstands wird, da es sich hierbei um einen alten Stöpselwiderstand, eine prozentuale Abweichung des angegebenen Werts von 10\% gewählt.
Im Allgemeinen werden zur Berechnung der kombinierten Unsicherheiten die nach GUM vorgesehenen Formeln verwendet. 
Die Berechnung dieser für diesen Versuch erfolgt im Anhang (\ref*{sec:anhang}).

\subsection{Datenanalyse}

Alle aufgenommenen Werte sind dem Laborbuch zu entnehmen. 
Darüber hinaus sind die Daten in den folgenden Diagrammen graphisch dargestellt.

Zur Bestimmung der Leerlaufspannung wird ein "unendlich"
% Rechnungsweg zum Ergebnis angeben und begründen, warum z. B. linearer Fit genommen wurde.
% Dazu auf die Theorie oder die Versuchsanleitung Bezug nehmen.
% Vollständige Fehlerberechnung in den Anhang.
% Da grundsätzlich schon vorher erläutert (GUM) reicht hier eine kurze Anmerkung auf die Fehlerrechnung.
% Ergebnis graphisch darstellen und auf passende Darstellungsweise (Unsicherheiten, signifikante Stellen) achten.

% Falls mehrere voneinander unabhängige Lösungen existieren (z. B. bei der geometrischen Bestimmung des Trägheitsmomentes des Kreisels), können diese in Unterkapitel gegliedert werden.

\subsection{Diskussion}

% Die Ergebnisse in den Kontext einbinden.
% Zusammenhänge noch einmal darstellen.
% Jegliche Aussagen durch Ergebnisse oder Vorwissen untermauern.
% Ergebnisse mit Referenzwerten/Literaturwerten vergleichen und auf Unsicherheiten eingehen (Vertrauensgrad).

\subsection{Schlussfolgerung}

% Das Ziel noch einmal erläutern.
% Fazit des Versuchs angeben (untermauert die These; weicht von den Literaturwerten ab; ...).
% Warum folgt aus den Ergebnissen das Fazit.
% Mit ermittelten Daten und Unsicherheiten begründen.
% Auf Unsicherheiten weiter eingehen und Vertrauensgrad bzw. Relevanz der Schlussfolgerung auf die derzeitige Wissenslage angeben (meistens nicht relevant).
% Verbesserungsvorschläge für das nächste Mal und allgemeine Rekapitulierung über das durchgeführte Experiment.
% Warum muss man das Experiment unbedingt noch mal durchführen.
