

\subsection{Unsicherheitsrechnung}\label{VGuD}

\begin{figure}[h]
	\begin{equation*}
		x = \sum_{i=1}^{N} x_i
		;\quad
		u(x) = \sqrt{\sum_{i = 1}^{N} u(x_i)^2}
	\end{equation*}
	\caption{Formel für kombinierte Unsicherheiten des selben Typs nach GUM.}
	\label{eq:GUM_combine}
\end{figure}

\begin{figure}[h]
	\begin{equation*}
		f = f(x_1, \dots , x_N)
		;\quad
		u(f) = \sqrt{\sum_{i = 1}^{N}\left(\pdv{f}{x_i} u(x_i)\right) ^2}
	\end{equation*}
	\caption{Formel für sich fortpflanzende Unsicherheiten nach GUM.}
	\label{eq:GUM_formula}
\end{figure}

\begin{figure}[h]
	\begin{equation*}
		I = \frac{U}{R}
		;\quad
		u(I) = \sqrt{\left( \frac{u(U)}{R}\right) ^2 + \left( -\frac{U}{R^2}\cdot u(R)\right) ^2}
	\end{equation*}
	\caption{Unsicherheit für den Strom eines ohm'schen Widerstandes. $U$ und $R$ sind direkt gemessen und deren Unsicherheiten sind gegeben mit $u(R) = 0,1 R$ und $u(U)$ (je nach Messbereich) $= \frac{\SI{0,1}{\volt}}{2\sqrt{6}} , \frac{\SI{0,2}{\volt}}{2\sqrt{6}} , \frac{\SI{1}{\volt}}{2\sqrt{6}}$.}
	\label{eq:unc-I}
\end{figure}

\begin{figure}[h]
	\begin{equation*}
		P = U \cdot I
		;\quad
		u(P) = \sqrt{\left( U u(I)\right) ^2 + \left( I u(U)\right) ^2}
	\end{equation*}
	\caption{Unsicherheit für die abgegebene Leistung über einen ohm'schen Widerstand. $U$ und $I$ sind direkt gemessen und deren Unsicherheiten sind gegeben mit $u(I) = \frac{\SI{0,01}{\ampere}}{2\sqrt{6}}$ und $u(U)$ wie oben in Abb. \ref{eq:unc-I=U/R}.}
	\label{eq:unc-P}
\end{figure}

\begin{figure}[h]
	\begin{equation*}
		\phi = \arccos(m)
		;\quad
		u(\phi) = \abs{-\frac{1}{\sqrt{1-m^2}}} = \frac{1}{\sqrt{1-m^2}}
	\end{equation*}
	\caption{Unsicherheit für den Phasenwinkel $\phi$ in Abhängigkeit zu der Steigung $m$ des dazugehörigen Leistungsgraphen. Die Unsicherheit für $m$ wurde aus dem linearen Fit entnommen. Da $m\leq 1$, entfällt der Betrag.}
	\label{eq:unc-phi}
\end{figure}

\begin{figure}[h]
		\begin{equation*}
		L = \frac{\abs{Z}}{\omega} \sin\phi
		;\quad
		u(L) = \frac{1}{\omega}\sqrt{\left( u(\abs{Z}) \sin\phi \right) ^2 + \left( \abs{Z} \cos\phi u(\phi) \right) ^2}
	\end{equation*}
	\caption{Unsicherheit für die Induktivität der Spule. $\omega$ ist die Kreisfrequenz der Spannungsquelle und sei als ideal angenommen. $u(\abs{Z})$ ist der Steigung des jeweiligen $U$-$I$-Graphen zu entnehmen.}
	\label{eq:unc-L}
\end{figure}

\begin{figure}[h]
	\begin{align*}
		C &= \frac{1}{\omega}\frac{1}{\left( \omega L - \abs{Z}\sin\phi \right) }
		\\
		u(C) &=\frac{1}{\omega}\sqrt{\left( \frac{-\omega}{\left( \omega L - \abs{Z} \sin\phi \right) ^2}\right) ^2+\left( \frac{\sin\phi}{\left( \omega L - \abs{Z} \sin\phi \right) ^2}\right) ^2+\left( \frac{\abs{Z}\cos\phi}{\left( \omega L - \abs{Z} \sin\phi \right) ^2}\right) ^2}
	\end{align*}
	\caption{Unsicherheit für die Kapazität des Kondensators.}
	\label{eq:unc-C}
\end{figure}
