
\section{Kurzfassung}

Dieser Bericht beschäftigt sich mit der Untersuchung von Widerständen und Leistungsabgabe von Verbrauchern. 
Dazu werden zwei Teilversuche betrachtet. In beiden werden die Sachverhalte, die die Theorie liefert bestätigt.

Der Erste beschäftigt sich mit der Untersuchung der Innenwiderstände und der Leistungsabgabe von Akkumulatorzellen.
Zur Betrachtung dieser wird ein einfacher Schaltkreis mit diesen und einem äußeren Widerstand herangezogen.
Durch Messung der anliegenden Spannung in Abhängigkeit des angelegten Widerstandes und der Theorie werden dann die Zusammenhänge aus gemessener und ermittelter Leerlaufspannung gebildet, der Innenwiderstand und die Leistungsabgabe bestimmt.
Das Ziel des Versuches ist die Übereinstimmung der gemessenen Werte mit den ermittelten. 
Dies wird bestätigt, da die gemessene Leerlaufspannung von z.B. \SI{1,2+-0,2}{V} für eine einzelne Akkumulatorzelle mit dem ermittelten Wert von \SI{1,260+-0,194}{V} übereinstimmt. Auch die Werte für drei in Reihe bzw. parallel geschaltete Akkumulatorzellen und für die maximale Leistung stimmen mit den Erwartungen überein.

Der zweite Teilversuch beschäftigt sich mit der Untersuchung des Verhaltens von verschiedenen Verbrauchern bei Gleich- und Wechselstrom.
Dazu werden verschiedene Verbraucher an einen Schaltkreis geschlossen.
Es werden die bei den verschieden Strömen erhaltenen Spannungen, Stromstärken und Leistungen verglichen und Beziehungen mit der Theorie, insbesondere der Impedanzen für die Spule und den Kondensator, sowie dessen Kapazität behandelt. 
Die Theorie zu bestätigen ist das Ziel dieses Versuches.
Dies wird auch anhand der Ergebnisse gezeigt.
Das Verhältnis der Widerstände bei Gleich- und Wechselstrom stimmt mit den Erwartungen überein und die ermittelte Kapazität des Kondensators entspricht bis auf eine kleine Unsicherheit exakt der angegebenen. \SI{0,060+-0,004}{\milli\farad} waren angegeben und \SI{0,060+-0,001}{\milli\farad} wurden ermittelt. 

