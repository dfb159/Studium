\section{Stoßprozesse mit einer durch eine Fallrinne beschleunigten Kugel}
Der Versuch stellt einen Zusammenhang zwischen der Höhenenergie einer kleinen Kugel und der Auslenkung eines Pendels nach einem Stoß her.
Dabei wird die kleine Metallkugel in einer Fallrinne bei verschiedenen Positionen losgelassen und der Energieübertrag bei dem Stoß mit der Pendelkugel untersucht.
Es wird der theoretische Anteil der kinetischen Energie mit $\varepsilon = \frac{5}{9}$ der Gesammtenergie überprüft und begründet, dass die Rollreibung des Systems diesen Anteil verfälscht.

	\subsection{Methoden}
	Die Fallrinne sei so justiert, dass die kleine Kugel der Rinne in einem geraden, zentralen Stoß auf die große Kugel des Pendels trifft.
	Die große Kugel ruht dabei vor dem Stoß direkt am Ende der Fallrinne.
	Die kinetische Energie der kleinen Kugel wird bei dem Stoß zu einem großen Teil an die große Kugel übertragen.
	Nach dem Stoß wird die Auslenkung der großen Kugel abgelesen.
	
	In Abb. \ref{fig:fallrinneSkizze} kann man $s$ als Abstand vom oberen Rand der Fallrinne identifizieren.
	Es werden nun für die Abstände $s$ von \SIrange{0}{50}{\centi\meter}, jeweils in \SI{10}{\centi\meter}-Schritten mehrere Messungen zur maximalen Auslenkung der großen Kugel nach dem Stoß durchgeführt.
	
	\begin{figure}[ht]
		\centering
		\includegraphics[width=0.6\textwidth]{FallrinneSkizze.png}
		\caption{Skizze der Fallrinne.}
		\label{fig:fallrinneSkizze}	
	\end{figure}

	\subsubsection*{Unsicherheiten}
	Die direkt gemessenen Größen seien mit den folgenden Unsicherheiten gemessen.
	
	Alle Längenmessungen wurden durch ein Maßstab oder -band gemacht, welches auf einen Millimeter genau abgelese nwerden konnte.
	Soweit nicht weiter aufgeführt, sei hier eine Unsicherheit von $u_0(l) = \frac{\SI{0,1}{\centi\meter}}{2\sqrt{6}} \approx \SI{0,02}{\centi\meter}$ anzunehmen.
	
	Für die Pendellänge wird durch erschwerte Ablesbarkeit an der Halterung eine Unsicherheit von $u(l) = \frac{\SI{1}{\centi\meter}}{2\sqrt{6}} \approx \SI{0,2}{\centi\meter}$ angenommen.
	Da die Halterungen für die beiden Seilenden des Pendels nicht optimal zentriert und auf leicht unterschiedlichen Höhen positioniert waren, konnte die Länge des Pendels von der Mitte der Halterung nicht Millimeter genau abgelesen werden.
	
 	Wie in Abb. \ref{fig:fallrinneSkizze} abgebildet, ist die Fallrinne zum Winkel $\alpha$ gerade weitergeführt worden, um einen Schnittpunkt mit der unteren Schiene zu bekommen.
	Durch diese Abschätzung ist die Unsicherheit für die Länge $L$ der Konstruktion mit $u(L) = \frac{\SI{0,5}{\centi\meter}}{2\sqrt{6}} \approx \SI{0,1}{\centi\meter}$ etwas höher gewählt.
	
	Die Waage hatte eine Unsicherheit von $u(m_1) = u(m_2) = \frac{\SI{0,01}{g}}{2\sqrt{3}} \approx \SI{0,003}{g}$.
	
	\subsection{Messung}
	
	Für die Abmessungen der Fallrinne wurde für $L = \SI{32,7 +- 0,1}{\centi\meter}$ gemessen.
	In Tab. \ref{tab:Messwerte} sind die Höhen über der untere Schiene der Fallrinne (vgl. Abb. \ref{fig:fallrinneSkizze}, Länge $L$) angegeben.
	\begin{table}[ht]
		\centering
		\caption{Höhenmessungen über der unteren Schiene und gemittelte Auslenkung des Pendels nach dem Stoß für die einzelnen Kugelpositionen.}
		\begin{tabular}{S|S|S}
			\hline
			{Abstand $s$} & {Höhe $H(s)$}  & {Auslenkung $a'_2$}\\
			\hline
			\SI{0 +- 0,1}{\centi\meter} & \SI{32,7 +- 0,1}{\centi\meter} & \SI{16,72 +- 0,24}{\centi\meter}\\
			\SI{10 +- 0,1}{\centi\meter} & \SI{27,2 +- 0,1}{\centi\meter} & \SI{15,30 +- 0,35}{\centi\meter}\\
			\SI{20 +- 0,1}{\centi\meter} & \SI{21,9 +- 0,1}{\centi\meter} & \SI{13,34 +- 0,11}{\centi\meter}\\
			\SI{30 +- 0,1}{\centi\meter} & \SI{16,7 +- 0,1}{\centi\meter} & \SI{11,52 +- 0,16}{\centi\meter}\\
			\SI{40 +- 0,1}{\centi\meter} & \SI{11,3 +- 0,1}{\centi\meter} & \SI{9,12 +- 0,15}{\centi\meter}\\
			\SI{50 +- 0,1}{\centi\meter} & \SI{5,7 +- 0,1}{\centi\meter} & \SI{5,94 +- 0,25}{\centi\meter}\\
			\hline
		\end{tabular}
	\end{table}
	Wir haben leider erst in der Auswertung gemerkt, dass wir die Grundhöhe der Rinne $H - h_0$ nicht gemessen haben.
	Deswegen muss diese Höhe dem $y$-Achsenabschnitt der Linearisierung entnommen werden.
	Aus diesem Grund wird die Analyse etwas anders durchgeführt.
	
	Für die kleine Kugel wurde ein Gewicht von \SI{63,69}{g}, für die große Kugel eines von \SI{510,21}{g} gemessen.
	
	\subsection{Datenanalyse}
	
	Die Fallhöhe der kleinen Kugel kann mit Abb. \ref{fig:fallrinneSkizze} hergeleitet werden.
	Es wird schnell deutlich, dass $h = h_0 - s \sin \alpha$ ist.
	Die beiden Größen $L$ und $H$ hängen mit $\tan \alpha = \frac{H}{L}$ zusammen.
	Stellt man diese etwas um, folgt
	\begin{equation}
		\frac{1}{\sin^2 \alpha} = 1 + \cot^2 \alpha = 1 + \frac{L^2}{H^2} \Leftrightarrow \sin \alpha = \frac{1}{\sqrt{1 + 	\frac{L^2}{H^2}}}.
	\end{equation}
	Oben eingesetzt ergibt sich 
	\begin{equation}
		\label{eq:hoehe}
		h = h_0 - \frac{s}{\sqrt{1 + \frac{L^2}{H^2}}}.
	\end{equation}
	Nach Gl. (28) der Einführung ist die maximale Auslenkung der großen Kugel gegeben durch
	\begin{equation}
		\label{eq:auslenkung}
		a'_2 = \frac{2 m_1}{m_1 + m_2} \sqrt{2 \varepsilon l h} \quad \text{mit} \quad \varepsilon = \frac{5}{9}.
	\end{equation}
	Setzt man nun Gl. (\ref{eq:hoehe}) in Gl. (\ref{eq:auslenkung}) ein und quadriert beide Seiten:
	\begin{equation}
		\label{eq:auslenkung2}
		(a'_2)^2 = \left( \frac{2 m_1}{m_1 + m_2}\right) ^2 2 \varepsilon l h = \varepsilon\gamma \left( h_0 - \frac{s}{\sqrt{1 + \frac{L^2}{H^2}}}\right),
	\end{equation}
	so ist $h_0$ von $s$ separiert und tritt als $y$-Achsenabschnitt einer linearen Gleichung nach $s$ auf\footnote{In der Anleitung soll $a'_2$ gegen $\sqrt{h}$ aufgetragen werden. Da wir einen Wert nicht mit aufgenommen hatten, ist die Linearisierung so deutlich einfacher und führt ebenso zum Ziel. Im späteren Verlauf wird der fehlende Wert aus der Linearisierung errechnet und auf Plausibilität geprüft.}.
	
	\begin{figure}[ht]
		\centering
		\includegraphics[width=\textwidth]{AuslenkungProAbstand.pdf}
		\caption{Maximale Auslenkung des Pendels bei unterschiedlichen Werten für $s$. Die Regressionsgerade zeigt den linearen Zusammenhang deutlich und liefert eine Steigung $m$, sowie einen $y$-Achsenabschnitt $y_0$.}
		\label{fig:fallrinneAuslenkung}	
	\end{figure}

	In Abb. \ref{fig:fallrinneAuslenkung} ist diese Linearisierung dargestellt.
	Nun wird der Anteil $\varepsilon$ der dem Stoß zur Verfügung stehenden Energie berechnet.
	Dazu wird aus Gl. (\ref{eq:auslenkung2}) die Steigung der Theorie mit der Steigung der Regressionsgeraden gleichgesetzt\footnote{Unsicherheitsrechnung im Anhang Gl. (\ref{eq:kombUnsicherheit}) mit Gl. (\ref{eq:epsilon}) bis Gl. (\ref{eq:epsilonEnd}).}:
	\begin{equation}
		m = -\frac{\varepsilon \gamma}{\sqrt{1 + \frac{L^2}{H^2}}} \Rightarrow \varepsilon = -\frac{m}{\gamma} \sqrt{1+\frac{L^2}{H^2}} \approx \SI{0,484 +- 0,012}{}.
	\end{equation}
	
	Zudem kann nun die fehlende Messung überprüft werden.
	Dazu wird aus Gl. (\ref{eq:auslenkung2}) für $s = 0$ der $y$-Achsenabschnitt der Ausgleichsgerade eingesetzt.
	Es folgt
	\begin{equation}
		\varepsilon \gamma h_0 = (a'_2)^2= y_0 \Rightarrow h_0 = \frac{y_0}{\varepsilon\gamma} \approx \SI{31,30 +- 0,88}{\centi\meter}
	\end{equation}
	und somit für die nicht gemessene Größe
	\begin{equation}
		h' = H - h_0 \approx \SI{1,40 +- 0,89}{\centi\meter}.
	\end{equation}
	
	\subsection{Schlussfolgerung}
	
	Der Versuch zeigt eine nicht unerhebliche Abweichung vom theoretisch hergeleiteten Faktor $\varepsilon = \frac{5}{9}$.
	Das könnte an der Reibung der Kugel auf der Fallrinne liegen, da die Kugel abgebremst wird und Energie verliert.
	Die Differenz von der potentiellen Höhenenergie am Anfang und bei maximaler Auslenkung ist also größer, als von der Theorieannahme und $\varepsilon$ der nutzbaren Energie somit kleiner.
	
	Die Nachrechnung der Höhe $h'$ ist damit auch nicht mehr genau.
	Mit dem theoretischen Wert von $\varepsilon = \frac{5}{9}$, ist $h' \approx \SI{5,4}{\centi\meter}$ also deutlich näher an dem Realitätswert.
	
	Man kann den Versuch noch einmal mit dem tatsächlichen gemessenen Wert für $h'$ durchführen.
	Es würde jedoch wahrscheinlich auf ein ähnliches Ergebnis hinauslaufen.
	