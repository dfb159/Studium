\documentclass[11pt,a4paper,titlepage, ngerman]{article}

\usepackage[utf8]{inputenc}	
\usepackage[T1]{fontenc}	
\usepackage{ngerman}			
\usepackage{lmodern}			
\usepackage{graphicx}			
\usepackage{url}				
\usepackage{siunitx}
\usepackage[intlimits]{amsmath}
\usepackage{xfrac}
\usepackage{commath}
\usepackage{physics}			
\usepackage{subcaption}
\usepackage{wrapfig}
\usepackage{biblatex}
\usepackage{hyperref}

% Setup SI unit environment
\sisetup{separate-uncertainty = true}
\sisetup{output-decimal-marker = {,}}
\sisetup{
	per-mode=fraction,
	fraction-function=\sfrac
	% or \frac, \tfrac
}
\bibliography{Literatur}
\begin{document}
	\begin{titlepage}
		\centering
		{\scshape\LARGE Versuchsbericht zu \par}
		\vspace{1cm}
		{\scshape\huge M4 -- Stoßgesetze\par}
		\vspace{2.5cm}
		{\LARGE Gruppe 10 Mi\par}
		\vspace{0.5cm}
		{\large Alex Oster (E-Mail: a\_oste16@uni--muenster.de) \par}
		{\large Jonathan Sigrist (E-Mail: j\_sigr01@uni--muenster.de) \par}
		\vfill
		durchgeführt am 13.12.2017\par
		betreut von\par
		{\large Semir \textsc{Vrana}} 
		\vfill	
		{\large \today\par}
	\end{titlepage}
	
	\tableofcontents
	
	\newpage
	
	\section{Kurzfassung}
	
	%TODO at the end kurzfassung{ziel, ergebnisse, wie gemacht}
	
			Bei dem ersten Versuch wird ein ballistischer zentraler Stoß zweier Metallkugeln betrachtet. Dazu werden zwei solcher Metallkugeln unterschiedlicher Masse an Pendeln aufgehängt. Der Stoßvorgang wird durch Auslenkung eines Pendels in Gang gesetzt und dann die Auslenkung der gestoßenen Kugel gemessen. Ziel dieses Versuches ist, dass die Messergebnisse mit den Stoßgesetzen übereinstimmen. Diese Übereinstimmung wird durch die Ergebnisse bestätigt.
		
	\vspace{2cm}	
	
	\section{Stoßprozese zwischen zwei Kugeln unterschiedlicher Massen} 
	\newpage
	\section{Stoßprozesse auf einer Rutschbahn} %TODO Joey
	
	%\section*{Literatur}
	%\printbibliography
	
	\newpage
	
	\section{Anhang}
	
	%TODO einfach einfuegen
	
	\begin{figure}[ht]
		\centering
		\includegraphics[width=\textwidth]{M4_1.jpg}
		\caption{Versuchsaufbau$^{[1]}$}
		\label{abb:VersuchsaufbauStoss}	
	\end{figure}
	
	\section*{Literatur}
	
	%TODO einfach einfuegen
	
	[1] Abb. \ref*{abb:VersuchsaufbauStoss} stammt aus der Vorbereitung zu M4, welche im Learnweb zu finden ist.
	
\end{document} 