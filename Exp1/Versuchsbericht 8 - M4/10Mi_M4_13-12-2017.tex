\documentclass[11pt,a4paper,titlepage, ngerman]{article}

\usepackage[utf8]{inputenc}	
\usepackage[T1]{fontenc}	
\usepackage{ngerman}			
\usepackage{lmodern}			
\usepackage{graphicx}			
\usepackage{url}				
\usepackage{siunitx}
\usepackage[intlimits]{amsmath}
\usepackage{xfrac}
\usepackage{commath}
\usepackage{physics}			
\usepackage{subcaption}
\usepackage{wrapfig}
\usepackage{biblatex}
\usepackage{hyperref}

% Setup SI unit environment
\sisetup{separate-uncertainty = true}
\sisetup{output-decimal-marker = {,}}
\sisetup{
	per-mode=fraction,
	fraction-function=\sfrac
	% or \frac, \tfrac
}
\bibliography{Literatur}
\begin{document}
	\begin{titlepage}
		\centering
		{\scshape\LARGE Versuchsbericht zu \par}
		\vspace{1cm}
		{\scshape\huge M4 -- Stoßgesetze\par}
		\vspace{2.5cm}
		{\LARGE Gruppe 10 Mi\par}
		\vspace{0.5cm}
		{\large Alex Oster (E-Mail: a\_oste16@uni--muenster.de) \par}
		{\large Jonathan Sigrist (E-Mail: j\_sigr01@uni--muenster.de) \par}
		\vfill
		durchgeführt am 13.12.2017\par
		betreut von\par
		{\large Semir \textsc{Vrana}} 
		\vfill	
		{\large \today\par}
	\end{titlepage}
	
	\tableofcontents
	
	\newpage
	
	\section{Kurzfassung}
	
		Dieser Bericht befasst sich mit den Stoßgesetzen. Dazu werden zwei Versuche betrachtet, die Übereinstimmungen zwischen den aufgenommenen Werten und den durch die Stoßgesetze ermittelten Werte zeigen sollen.   
		
		Bei dem ersten Versuch wird ein ballistischer zentraler Stoß zweier Metallkugeln betrachtet. Dazu werden zwei solcher Metallkugeln unterschiedlicher Masse an Pendeln aufgehängt. Der Stoßvorgang wird durch Auslenkung eines der Pendel in Gang gesetzt und dann wird die Auslenkung der gestoßenen Kugel gemessen. Ziel dieses Versuches ist, dass das Messergebnis für das Massenverhältnis mit dem bestimmten Wert dafür übereinstimmt. Diese Übereinstimmung wird durch die Ergebnisse gezeigt.
		
		%TODO at the end kurzfassung{ziel, ergebnisse, wie gemacht}
			
	\vspace{2cm} % ggf. noch \newpage einbringen für niceren Anblick	
	
	\section{Stoßprozese zwischen zwei Kugeln unterschiedlicher Massen} 
	\newpage
	\section{Stoßprozesse auf einer Rutschbahn} %TODO Joey
	
	%\section*{Literatur}
	%\printbibliography
	
	\newpage
	
	\section{Anhang} % entfernen wenn leer
	
	%TODO einfach einfuegen
	
	\begin{figure}[ht]
		\centering
		\includegraphics[width=\textwidth]{M4_1.jpg}
		\caption{Versuchsaufbau$^{[1]}$}
		\label{abb:VersuchsaufbauStoss}	
	\end{figure}
	
	\section*{Literatur}
	
	%TODO einfach einfuegen
	
	[1] Abb. \ref*{abb:VersuchsaufbauStoss} stammt aus der Vorbereitung zu M4, welche im Learnweb zu finden ist.
	
\end{document} 