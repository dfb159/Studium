\documentclass[11pt,a4paper,titlepage, ngerman]{article}

\usepackage[utf8]{inputenc}	
\usepackage[T1]{fontenc}	
\usepackage{ngerman}			
\usepackage{lmodern}			
\usepackage{graphicx}			
\usepackage{url}				
\usepackage{siunitx}
\usepackage{amsmath}			
\usepackage{subcaption}
\usepackage{wrapfig}
\usepackage{biblatex}

\newcommand{\refeq}[1]{Gl. (\ref{eq:#1})}
\newcommand{\refabb}[1]{Abb. \ref{abb:#1}}
\newcommand{\reffig}[1]{Fig. \ref{fig:#1}}
\newcommand{\reftab}[1]{Tab. \ref{tab:#1}}

\begin{document}

	\begin{titlepage}
		
		\centering
		{\scshape\LARGE Gleichungen zu \par}
		\vspace{1cm}
		{\scshape\huge M1 -- Drehpendel nach Pohl\par}
		\vspace{2.5cm}
		{\LARGE Gruppe 10 Mi\par}
		\vspace{0.5cm}
		{\large Alex Oster (E-Mail: a\_oste16@uni--muenster.de) \par}
		{\large Jonathan Sigrist (E-Mail: j\_sigr01@uni--muenster.de) \par}
		\vfill
		durchgeführt am 15.11.2017\par
		betreut von\par
		{\large Johann Preuß}		
		\vfill	
		{\large \today\par}
		
	\end{titlepage}
		
	\tableofcontents
		
	\newpage
		
	\section{Messunsicherheiten der folgenden Versuche}
		
		Für die Zeitungenauigkeit am Computer: $u(t) = \frac{\SI{0,05}{\second}}{2\sqrt{3}}$.
		Das ist gerade die Breite der Abtastrate.
		
		Für die Positionsunsicherheit am Computer: $u(x) = \frac{\SI{1}{mm}}{2\sqrt{3}}$.
		Summe von allem (Seildehnung, CASSY-Unsicherheit), weil nicht gut einzeln bestimmbar.
		
		Für die Stoppuhr: $u_\text{Uhr}(t) = \sqrt{u_\text{Digit}(t)^2 + u_\text{Reaktion}(t)^2}$.
		Dabei ist $u_\text{Digital}(t) = \frac{\SI{10}{ms}}{2\sqrt{3}}$ und $u_\text{Reaktion}(t) = \frac{\SI{100}{ms}}{2\sqrt{6}}$.
		
		Für das Multimeter (0.5\%): $u_\text{Multi}(U) = \sqrt{u_\text{Digit}(U)^2 + u_\text{Prozent}(U)^2}$.
		Messreichweite von \SI{20}{V}.
		Mit $u_\text{Digit}(U) = \frac{\SI{0,01}{V}}{2\sqrt{3}}$ und $u_\text{Prozent}(U) = 0,005 U$.
		
		Stromquelle: $u_\text{Strom}(I) = u_\text{Digital}(I) = \frac{\SI{0,01}{A}}{2\sqrt{3}}$.
		
		Für manuelle Auslenkungsablesung sei $a = \SI{2}{mm}$, weil der Wert manchmal ein wenig schwer abzulesen war.		
		Es sei $u_\text{Analog}(x) = \frac{\SI{2}{mm}}{2\sqrt{6}}$.
	
	\section{Versuch zu freien Schwingungen}
		
		\subsection*{Computer}
		Es wurden 20 Messreihen aufgenommen, da nach 10ter der Faden rausgesprungen ist, nur 10 erste Messreihen betrachten.
		Es werden von jeder Messreihe die Dauer für 10 Schwingungsperioden genommen, dabei ab dem ersten Minima bis zum 11 Minima für höhere genauigkeit.
		
		Periodendauer: $T=\SI{1,8255}{s}$ mit $u(T) = \SI{0,002041241452319}{s}$.
		Errechnet: $dT = T_2 - T_1$ sei zeitl. Abstand zwischen dem 1. Minima($T_1$) und dem 11. Minima ($T_2$).
		$u(T_1) = u(T_2)$.
		\begin{align}
			u(dT) &= \sqrt{\left( \frac{\partial\,dT}{\partial\,T_1} \cdot u(T_1)\right)^2 +\left( \frac{\partial\,dT}{\partial\,T_2} \cdot u(T_2)\right)^2}\\
			&= \sqrt{(-u(T_1))^2 + (u(T_2))^2}\\
			&= \sqrt{2 u(T)^2} = \sqrt{2} u(T)
		\end{align}
		
		Für die Frequenz ergibt sich: $f=\frac{1}{T} = \SI{0,547795124623391}{\per\second}$, 
		$u(f) = \frac{u(T)}{T^2} =\SI{ 0,000612534711454}{\per\second}$.
		
		\subsection*{Stoppuhr}
		20  mal 10 Zeitstoppungen.
		Mittelwert $\bar{T} = \SI{1,8332}{\second}$ und $u(T) = \SI{0,03863949438483623}{\second}$.
		Dabei sei $u(T)$ nicht kombiniert aus Stoppuhr und Messwerte, weil die Uhr schon in den Messwerten drinsteckt.
		Diese Unsicherheit ist doppelt so groß wie die kombinierte.(schlechte Messung? doch keine Reaktionszeit von 100ms?)
		Es seien $f=1/T = \SI{0,545494217761292}{\per\second}$ und $u(f) = \frac{u(T)}{T^2}= \SI{0,011497720251008}{\per\second}$.
		
		Man sieht direkt, dass die Computeranalyse deutlich genauer war, selbst mit den gewählten, immer noch sehr großen Unsicherheiten(für das CASSY).
		
	\section{Versuch zu gedämpften Schwingungen}
	
	%TODO Student'sche Faktoren einbringen auf u(T) und u(f)
		Dabei sind $f$ und $u(f)$ wie in M1.2 berechnet. $T$ und $u(T)$ sind aus SciDaVis Spaltenstatistik der dT's.
		
		\subsection*{0,25A}
		Mittelwert: $T = \SI{1,81785714285714}{s}$ und $u(T) = \SI{0,031666184688832}{s}$.\\
		Mittelwert: $f = \SI{0,550098231827112}{\per\second}$ und $u(f) = \SI{0,009582442863832}{\per\second}$\\
		$A_0 = \SI{0,848098764533296}{m}$ und $u(A_0) = \SI{9,27988502510862E-05}{m}$.\\
		$a = \SI{0,436801095407458}{\per\second}$ und $u(a) = \SI{0,001141232808895}{\per\second}$.\\
		Student: 12 FG, auf 1 sigma
		
		\subsection*{0,5A}
		$T = \SI{1,81875}{s}$ und $u(T) = \SI{0,025877458475339}{s}$.\\
		$f = \SI{0,549828178694158}{\per\second}$ und $u(f) = \SI{0,007823041024181}{\per\second}$.\\
		$A_0 = \SI{0,891244481206147}{m}$ und $u(A_0) = \SI{0,018701665906075}{m}$.\\
		$a = \SI{0,267687202893829}{\per\second}$ und $u(a) = \SI{0,00255559233324}{\per\second}$.\\
		Student: 6 FG, auf 1 sigma
			
		
		\subsection*{0,75A}
		Mittelwert: $T = \SI{1,82}{s}$ und $u(T) = \SI{0,044721359549996}{s}$.\\
		Mittelwert: $f = \SI{0,549450549450549}{\per\second}$ und $u(f) = \SI{0,013501195371935}{\per\second}$.\\
		$A_0 = \SI{1,10527437944171}{m}$ und $u(A_0) = \SI{0,003855449459945}{m}$.\\
		$a = \SI{0,445639253818205}{\per\second}$ und $u(a) = \SI{0,000570284053541}{\per\second}$.\\
		Student: 3 FG, auf 1 sigma
		
		
	\section{Versuch zu erzwungenen Schwingungen}
			
	\section{Versuch zu nichtlinearen Schwingungen}

	\newpage			
	\section*{Literatur}
		\begin{description}
			\item[\refabb{Drehpendel}] Das hier verwendete Bild stammt aus \glqq Drehpendel\_Pohl\_Einführung.pdf\grqq
		\end{description}
\end{document} 