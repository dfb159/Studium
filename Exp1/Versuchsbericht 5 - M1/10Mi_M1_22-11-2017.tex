\documentclass[11pt,a4paper,titlepage, ngerman]{article}

\usepackage[utf8]{inputenc}	
\usepackage[T1]{fontenc}	
\usepackage{ngerman}			
\usepackage{lmodern}			
\usepackage{graphicx}			
\usepackage{url}				
\usepackage{siunitx}
\usepackage{amsmath}			
\usepackage{subcaption}
\usepackage{wrapfig}

\newcommand{\refeq}[1]{Gl. (\ref{eq:#1})}
\newcommand{\reffig}[1]{Fig. \ref{fig:#1}}
\newcommand{\reftab}[1]{Tab. \ref{tab:#1}}

\begin{document}

	\begin{titlepage}
		
		\centering
		{\scshape\LARGE Versuchsbericht zu \par}
		\vspace{1cm}
		{\scshape\huge M1 -- Drehpendel nach Pohl\par}
		\vspace{2.5cm}
		{\LARGE Gruppe 10 Mi\par}
		\vspace{0.5cm}
		{\large Alex Oster (E-Mail: a\_oste16@uni--muenster.de) \par}
		{\large Jonathan Sigrist (E-Mail: j\_sigr01@uni--muenster.de) \par}
		\vfill
		durchgeführt am 15.11.2017\par
		betreut von\par
		{\large Johann Preuß}		
		\vfill	
		{\large \today\par}
		
	\end{titlepage}
		
	\tableofcontents
		
	\newpage
	
	\section{Kurzfassung}
		
		%TODO
		% Kurze Versuchsbeschreibung
		% Ziel des Ganzen
		
		\section{Pohl'sches Rad}
		
			%TODO
			% Bild (Quelle angeben)
			% Funktionsweise
		
		\section{Messunsicherheiten der folgenden Versuche}
		
			%TODO
			% Computermessung (?)
			% Stoppuhr (Digital + Reaktionszeit)
			% Multimeter (Digitalanzeige + 0,5% des Wertes)
			% Stromquelle (Digitalanzeige)
			% Auslenkungsbestimmung (Fehlertoleranzbereich von +- 1mm) ?
		
		\section{Versuch zu freien Schwingungen}
			
			%TODO
			% blah blah
			
			\subsection*{ideale Schwingung}
				
				%TODO
				% Beschreibung ideale Schwingung
				
			\subsection*{Methoden}
				
				%TODO
				% Bestimmung der Eigenfrequenz bei freier Schwingung
					% Messung über Computer/mit Stoppuhr
			
			\subsection*{Messung}
			
				%TODO
				% Messung
				
			\subsection*{Schlussfolgerung}
			
				%TODO
				% Schlussfolgerung
					
		\section{Versuch zu gedämpften Schwingungen}
			
			%TODO
			% blah blah
			
			\subsection*{Gedämpfte Schwingungen}
				
				%TODO
				% Beschreibung gedämpfte Schwingung
				
			\subsection*{Methoden}
				
				%TODO
				% Bestimmung der Eigenfrequenz im gedämpften Fall
					% Wirbelstrombremse mit verschiedenen Stärken [über Stromstärke(Angabe auf Stromquelle)]
			
			\subsection*{Messung}
			
				%TODO
				% Messung
				
			\subsection*{Schlussfolgerung}
			
				%TODO
				% Schlussfolgerung
				
		\section{Versuch zu erzwungenen Schwingungen}
		
			%TODO
			% blah blah
			
			\subsection*{erzwungene Schwingungen}
			
				%TODO
				% Beschreibung erzwungene Schwingung
			
			\subsection*{Methoden}
				
				%TODO
				% Bestimmung der Resonanzfrequenz bei erzwungener Schwingung
				% Frequenz der Anregung (Kalibrierkurve)
					% Frequenz über Ausgangsspannung (mit Multimeter gemessen)
				% Resonanzfrequenzbestimmung bei 3 Dämpfungen mit min. 20 Messpunkten
					% Darstellen dieser Resonanzkurven
				% Betrachtung der Phasenverschiebung zwischen Anreger und Pendel
					% bei schwacher/starker Dämpfung und niedriger/hoher Frequenz
					% Auf Fadenpendel übertragen
			
			\subsection*{Messung}
			
				%TODO
				% Messung
				
			\subsection*{Schlussfolgerung}
			
				%TODO
				% Schlussfolgerung
				
		\section{Versuch zu nichtlinearen Schwingungen}
		
			%TODO
			% blah blah
		
			\subsection*{nichtlineare Schwingungen}
				
				%TODO
				% Beschreibung nichtlineare Schwingung
				
			\subsection*{Methoden}
				
				%TODO
				% Einbringen einer Nichtlinearität und Betrachtung der Resonanz
			
			\subsection*{Messung}
				
				%TODO
				% Messung
				
			\subsection*{Schlussfolgerung}
				
				%TODO
				% Schlussfolgerung
				
\end{document} 