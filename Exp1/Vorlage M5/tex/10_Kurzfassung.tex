%Zusammenfassung in unter 200 Wörtern

\section{Zusammenfassung}


Dieser Bericht beschäftigt sich mit der Untersuchung der Trägheit. Dazu werden zwei Versuche herangeführt, welche diese auf verschiedene Weisen darstellen.

Der erste Versuch zu dem Fallrad zeigt ein Verhalten, welches einem Jo-Jo ähnelt. Ein langsames Abrollen und darauffolgendes Aufrollen. Hierbei wird jedoch das Abrollen genauer betrachtet. Es werden die Fallbeschleunigung und das Trägheitsmoment bestimmt und mit theoretischen Zusammenhängen, wie z. B. mit dem Abrollradius in Verbindung gebracht und die Theorie mit übereinstimmenden Werten bestätigt.

Der zweite Versuch zeigt, dass ein Kugelkreisel durch seine Trägheit im rotierenden Zustand eine gleichförmige Bewegung durchführt.
Es tritt eine Präzessionsbewegung auf, mit welcher sich die Figurenachse des Kreisels langsam aber gleichmäßig um die Senkrechte dreht.
Durch die Präzessionseffekte wird das Trägheitsmoment bestimmt und mit weiteren theoretischen Werten verglichen.
