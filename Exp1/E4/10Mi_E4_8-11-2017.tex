\documentclass[11pt,a4paper,titlepage, ngerman]{article}

\usepackage[utf8]{inputenc}	% Diese Pakete sind
\usepackage[T1]{fontenc}		% für die Verwendung 
\usepackage{ngerman}			% von Umlauten im tex-file
\usepackage{lmodern}			% Schriftart, die am Bildschirm besser lesbar ist
\usepackage{graphicx}			% Zum Einbinden von Formeln
\usepackage{url}					% Zur Darstellung von Webadressen
\usepackage{siunitx}
\usepackage{amsmath}			% für equation*
\usepackage{subcaption}
\usepackage{wrapfig}

\begin{document}
%	\setlength{\parindent}{0em} 
	
	\begin{titlepage}
		\centering
		{\scshape\LARGE Versuchsbericht zu \par}
		\vspace{1cm}
		{\scshape\huge E4 -- Kennlinien\par}
		\vspace{2.5cm}
		{\LARGE Gruppe 10 Mi\par}
		\vspace{0.5cm}
		{\large Alex Oster (E-Mail: a\_oste16@uni--muenster.de) \par}
		{\large Jonathan Sigrist (E-Mail: j\_sigr01@uni--muenster.de ) \par}
		\vfill
		durchgeführt am 8.11.2017\par
		betreut von\par
		{\large David \textsc{Pahl}}
		
		\vfill
		
		{\large \today\par}
	\end{titlepage}
		
	\tableofcontents
	
	\newpage
	
	\section{Kurzfassung}
	
	\section{Methoden}		
		
		\subsection{Versuch 1}
		\subsection{Versuch 2}
		
	\section{Diskussion}		
		
		\subsection{Funktionsweise von Halbleitern}
		
		\subsection{Datenanalyse}
		
	\section{Zusammenfassung}
		
	%\section{Schlussfolgerung} Braucht man das hier überhaupt?
		\newpage
		
		\begin{thebibliography}	
		
		\end{thebibliography}	
			
\end{document} 