\documentclass[11pt,a4paper,titlepage, ngerman]{article}

\usepackage[utf8]{inputenc}	% Diese Pakete sind
\usepackage[T1]{fontenc}		% für die Verwendung 
\usepackage{ngerman}			% von Umlauten im tex-file
\usepackage{lmodern}			% Schriftart, die am Bildschirm besser lesbar ist
\usepackage{graphicx}			% Zum Einbinden von Formeln
\usepackage{url}					% Zur Darstellung von Webadressen
\usepackage{siunitx}
\usepackage{amsmath}			% für equation*
\usepackage{subcaption}
\usepackage{wrapfig}

\begin{document}
%	\setlength{\parindent}{0em} 
	
	\begin{titlepage}
		\centering
		{\scshape\LARGE Versuchsbericht zu \par}
		\vspace{1cm}
		{\scshape\huge E4 -- Kennlinien\par}
		\vspace{2.5cm}
		{\LARGE Gruppe 10 Mi\par}
		\vspace{0.5cm}
		{\large Alex Oster (E-Mail: a\_oste16@uni--muenster.de) \par}
		{\large Jonathan Sigrist (E-Mail: j\_sigr01@uni--muenster.de ) \par}
		\vfill
		durchgeführt am 8.11.2017\par
		betreut von\par
		{\large David \textsc{Pahl}}
		
		\vfill
		
		%TODO Grafikbeschriftungen an die Messung/Versuch anpassen und keine nummer
		
		{\large \today\par}
	\end{titlepage}
		
	\tableofcontents
	
	\newpage
	
	\section{Kurzfassung}
		
		In diesem Bericht befassen wir uns mit Kennlinien. Eine Kennlinie ist die Kurve, die entsteht, wenn man die Spannung gegen den Strom aufträgt. Aus dem Ohm'schen Gesetz $U=RI$ bzw. $I = \frac{1}{R}U $ ergibt sich, dass diese durch den Widerstand dargestellt wird und bei konstanten Widerständen linear verläuft.
		
		Wir betrachten im Folgenden zwei Versuchen, die uns verschiedenen Arten von Widerständen näher bringen sollen.
		Diese hängen unter anderem von der Temperatur, den Materialien und der Bauteilgeometrie ab.
		
		In dem ersten Versuch betrachten wir fünf verschiedene Arten von Widerständen. Eine einfache Diode, eine Zenerdiode, einen NTC-Widerstand, eine Glüh- und eine Glimmlampe. Wir messen hierbei den Strom in Abhängigkeit von der Spannung und werten die Ergebnisse aus. Dazu gehen wir auf die Funktionsweise von (dotierten) Halbleitern und Gasentladungen ein. 
		
		Die Abhängigkeit des Widerstands von der Temperatur wird dann in dem zweiten Versuch für einen Kupferdraht betrachtet.
		Hierzu erhitzen wir diesen Draht in Öl, lassen ihn danach abkühlen und messen durchgehend seinen Widerstand mit Hilfe einer Wheatstone'schen Brücke. Unsere Ergebnisse für den Kupferdraht verknüpfen wir dann mit der elektrischen Leitfähigkeit von Metallen.

	\section{Versuch 1: Strom-Spannungs-Charakteristik}
		
		In diesem Versuch betrachten wir die Kennlinien von verschiedenen Bauteilen. Dafür messen wir den Strom und tragen diesen gegen unsere Eingangsspannung auf. Den sich daraus ergebenen Verlauf, die Kennlinie, erklären wir dann im Sachverhalt. 
		
		\subsection{Methoden} 
		
		Der Versuchsaufbau ist in Abb. \ref{Schaltskizze1} skizziert. Dabei fließt der Strom nur über jeweils eine Teilschaltung \textbf{a)} bis \textbf{e)}. 
		Da die Spannungsquelle keine genaue Angabe der Eingangsspannung liefert, messen wir diese mit einem Multimeter über das relevante Bauteil.
		Den Strom messen wir mit einem zweiten Multimeter, welches in Reihe geschaltet ist.
		
		Für die Teilversuche \textbf{a)} bis \textbf{d)} verwenden wir Eingangsspannungen im Bereich von \SIrange{0}{20}{\V} und für die Glimmlampe in \textbf{e)} Eingangsspannungen von \SIrange{0}{150}{\V}.
		
		\begin{figure}
			\includegraphics[width=\textwidth]{Versuch1.png}
			\caption{Schaltskizze zu Versuch 1}
			\label{Schaltskizze1}
		\end{figure}

		\subsection{a) Diode in Durchlassrichtung} 
			\label{a)}
			
			\subsubsection*{Diode}
				\label{Diode}
				
				% TODO Spannungen? eher Ströme
				Eine Diode ist ein Bauteil, mit dem Spannungen nur von einer Richtung durchgelassen werden.
				Diesen Effekt erhält man, wenn man einen Halbleiter mit einem Element aus der dritten Hauptgruppe dotiert (p-Leiter) und einen anderen mit einem Element aus der fünften Hauptgruppe (n-Leiter) und diese aneinander grenzen lässt.
				Hierbei entsteht eine Raumladungszone im Übergangsbereich, da die Elektronen des n-Leiters zu dem p-Leiter wandern.
				Da in diesem Übergangsbereich weniger freie Ladungsträger sind, als im restlichen Leiter, nimmt die Leitfähigkeit an dieser Stelle stark ab.
				
% Bissl umgeschrieben
				Legt man an den positiven Pol, also den n-Leiter eine äußere Spannung an, so vergrößert sich der Übergangsbereich.
				Bei umgekehrter Polung hingegen, verkleinert sich dieser und verschwindet bei außreichend hoher Spannung komplett, sodass der Strom durch den Leiter fließen kann. 
			
			\subsubsection*{Messung}
				
				Unsere Messwerte ergaben, wie in Abb. \ref{KL a} zu sehen ist, dass die Kennlinie bei einer Diode in Durchflussrichtung exponentiell verläuft.
				Einen messbaren Strom erhielten wir erst ab ca. \SI{0.5}{\V} und es ließen sich ab ungefähr \SI{0.72}{\V} keine weiteren Werte messen. %TODO? da die Spannungsquelle bereits voll aufgedreht war
				%TODO? Im Bereich 0,65V - 0,7V wurde das Einstellen besonders schwer
				
				\begin{figure}
					\centering
					\includegraphics[width=\textwidth]{KennlinieDiode.pdf}
					\caption{Messung zu Versuch 1a)}
					\label{KL a}
				\end{figure}
			
			\subsubsection*{Schlussfolgerung}
			
				%TODO? rewrite: deuten auf eine minimale Spannung hin, um ..
				Unsere Ergebnisse deuten darauf hin, dass mindestens \SI{0.5}{\V} nötig sind, um den Übergangsbereich der Diode soweit zu verkleinern, dass überhaupt ein Strom fließen kann. 
				
				%TODO rewrite: sich keine höheren Spannungen einstellen ließen...
				%TODO eher am innenwiderstand des netzteils würde ich sagen, aber keine ahnung
				Dass sich ab ca. \SI{0.72}{\V} keine weiteren Werte messen ließen, könnte daran liegen, dass die Diode so modifiziert ist, dass nicht mehr Spannung als diese \SI{0.72}{\V} durchgelassen werden können.
				Somit fällt der Widerstand nicht gegen null und wir vermeiden einen Kurzschluss.
				
				%TODO? größe der Spannung
				Die Größe des Übergangsbereichs steht in direktem Zusammenhang mit dem Widerstand.
				Denn bei einer höheren Spannung ist auch der Übergangsbereich kleiner und freie Ladungsträger können ihn leichter passieren.
				Der Widerstand wird also geringer.
				
				Dieser Sachverhalt wird durch den exponentiellen Verlauf der von uns gemessenen Kennlinie unterstützt.
				Somit entspricht unser Ergebnis den Erwartungen.
				
		\subsection{b) Zenerdiode} 
			
			Die Zenerdiode funktioniert ähnlich wie die in \ref{Diode} beschriebene Diode, nur dass bei ihr auch in Sperrrichtung Strom fließen kann, insofern die angelegte Spannung groß genug ist.
			%TODO sind das nicht eher hohe spannungen?
			Dies geschieht durch das Kleiner-werden der Potentialbarrieren bei hochdotierten p-n-Übergängen, wenn starke Ströme fließen.
			Hierbei kommt es zu dem quantenmechanischen Tunneleffekt, durch welchen Valenzelektronen in das Leitungsband gelangen.
			Es bildet sich ein Durchbruchstrom.
			%TODO zusatz: Dieser zerstört die Diode nicht, wie es beim Lawinendurchbruch der Diode der Fall ist.
			
			\subsubsection*{Messung}
			
				 \begin{itemize}
				 	
				 	\item Sperrrichtung: 
				 	
				 	In Abb. \ref{KL b1} erkennt man, dass wir auch hier einen exponentiellen Verlauf der Kennlinie gemessen haben. 				 	
				 	Anders als in der vorausgehenden Messung, fließt hier jedoch erst ab ca. \SI{2.6}{\V} ein Strom und wir konnten dieses mal bis ungefähr \SI{5}{\V} einen Strom messen.
				 	Damit liegt die Durchbruchspannung deutlich über der Durchlassspannung.
				 	
				 	\begin{figure}
				 		\centering
				 		\includegraphics[width=\textwidth]{KennlinieZenerdiodeSperrrichtung.pdf}
				 		\caption{Erste Messung zu Versuch 1b)}
				 		\label{KL b1}
				 	\end{figure}
				 	
				 	\item Durchlassrichtung:  
				 	
				 	Diese Messung verlief analog zu der in \ref{a)} durchgeführten Messung, wie in Abb. \ref{KL b2} zu sehen.
				 	Auch hier beträgt die Durchlassspannung ca. \SI{0,55}{\V}.
				 	%TODO? genauer? eher gleichmäßiger
				 	%TODO? mit welcher messung \ref
				 	Verglichen mit der ersten Messung, ist diese genauer und der zu erkennende exponentielle Anstieg wirkt ein wenig stärker.
				 	
				 	\begin{figure}
				 		\centering
				 		\includegraphics[width=\textwidth]{KennlinieZenerdiodeDurchlassrichtung.pdf}
				 		\caption{Zweite Messung zu Versuch 1b}
				 		\label{KL b2}
				 	\end{figure}
				 					 	
				\end{itemize}
											
			\subsubsection*{Schlussfolgerung}
				
				Die Ergebnisse dieser Messung bezüglich der Durchlassrichtung stehen in direktem Zusammenhang zu dem Ergebnis aus \ref{a)}. 
				Für die Sperrrichtung hingegen ermitteln wir, dass die benötigte Spannung für einen Stromfluss bei ca. \SI{2.6}{\V} liegt. 
				
				Verglichen mit den \SI{0.5}{\V} in Durchflussrichtung ist dies ein beachtlicher und zu erwartender Unterschied.
				Im Gegenzug zu der Verkleinerung des Übergangsbereichs der Diode fließt in Sperrrichtung, wenn überhaupt, nur der Durchbruchstrom und für diesen benötigt man hohe Spannungen. 
				
				Dass auch in Sperrrichtung die Kennlinie exponentiell verläuft, hat einen ähnlichen Grund, wie in Durchflussrichtung.
				Denn mit höheren Spannungen lösen sich auch mehr Elektronen aus dem Valenzband des p-Leiters und folglich ist der Durchbruchstrom stärker.
				Der Widerstand wird also auch hier mit höherer Spannung kleiner. 
				
		\subsection{c) Glühlampe} 
			
			Bei der Glühlampe fließt der Strom durch einen Glühdraht.
			Dieser erwärmt sich dabei und beginnt zu glühen.
			Wir betrachten also nun, wie sich der Widerstand nach Inbetriebnahme der Glühlampe ändert.
			Dazu messen wir den Strom wie bisher und tragen ihn gegen die Spannung auf.
			Um die Änderung des Widerstandes besser zu betrachten tragen wir zudem noch den Widerstand gegen die Spannung auf, wobei sich dieser aus dem Ohm'schen Gesetz ($R = U / I$) und der vorausgegangen Strommessung ergibt.  
			
			\subsubsection*{Messung}
				
				Die Kennlinie der Glühlampe verläuft zunächst krumm, richtet sich im späteren Verlauf jedoch annähernd linear aus. Wir beobachten, dass die Glühlampe ab ca. \SI{2.6}{\V} anfängt schwach zu glühen. In Abb. \ref{KL c} ist zu sehen, dass die Kennlinie sich in diesem Bereich beginnt sich linear auszurichten.
				
				%TODO Beschreibung bei U<2V: linearität un R/U zu erkennen => Widerstand bei Zimmertemperatur
				
				Der letzte messbare Wert war hier bei ca. \SI{7.2}{\V}
				
				\begin{figure}
					\centering
					\includegraphics[width=\textwidth]{KennlinieGluehlampe.pdf}
					\caption{Messung zu Versuch 1c)}
					\label{KL c}
				\end{figure}
				\begin{figure}
					\centering
					\includegraphics[width=\textwidth]{KennlinieGluehlampeWiderstand.pdf}
					\caption{Widerstand in Abhängigkeit der Spannung}
					\label{R c}
				\end{figure}
			
			\subsubsection*{Schlussfolgerung}
			
				%TODO ich würde nicht allzu stark auf diese linearität eingehen 0formeln und eher logarithmisch
				Der annähernd lineare Verlauf der Kennlinie deutet darauf hin, dass der Widerstand sich bei höheren Spannungen kaum ändert.
				In Abb. \ref{R c} erkennen wir auch, dass dies der Fall ist, wobei der Widerstand weiterhin langsam steigt.
				
				Das allgemeine Steigen des Widerstands lässt sich auf die steigende Temperatur zurückführen.
				Bei dem Glühdraht handelt es sich um ein Metall, vermutlich Wolfram.
				Metalle haben die Eigenschaft, dass ihre elektrische Leitfähigkeit mit steigender Temperatur abnimmt.
				
				Zu Beginn ist der Glühdraht bei Zimmertemperatur und wie in Abb. \ref{R c} zu erkennen ist, steigt der Widerstand zunächst stark an, weswegen die Kennlinie erst krumm verläuft.
				Ab dem Punkt, an dem der Draht glüht, steigt der Widerstand langsamer, da die Temperaturunterschiede schwächer werden.
				%TODO zusatz: in Abb6 ist zu erkennen: erste paar Werte linearisieren und R_0 ausrechnen für Zimmertemperatur
				
				Dieses Verhalten stimmt mit der Änderung der elektrischen Leitfähigkeit von Metallen bei erhöhten Temperaturen überein.
				
		\subsection{d) NTC-Widerstand} 
			
			NTC-Widerstände bestehen aus Halbleitern und verhalten sich, was den Innenwiderstand betrifft, anders als Metalle.
			Das \glqq NTC\grqq{} steht für \glqq Negative Temperature Coefficient\grqq{}, dass sie sich also bei erhöhten Temperaturen, im Gegensatz zu Metallen, besser den elektrischen Strom leiten als bei geringeren Temperaturen.
			%TODO kalt (oder warm?) leiter
			
			Bei dieser Messung betrachten wir einen solchen NTC-Widerstand.
			Vor dem aufnehmen eines jeden Messwertes lassen wir den Widerstand zwei bis drei Minuten in Ruhe, damit er sich auf die Temperatur einstellen kann.%TODO del: , die er bei gegebener Spannung annehmen sollte. 
			
			\subsubsection*{Messung}
			
				In Abb. \ref{KL d} sehen wir ähnlich zu den Dioden einen exponentiellen Verlauf der Kennlinie, wobei dieser Widerstand von Anfang an den Strom durchlässt und nicht erst ab einer gewissen Spannung.
				Hier konnten wir nur Werte bis \SI{8}{\V} Eingangsspannung aufnehmen.
				
				\begin{figure}
					\centering
					\includegraphics[width=\textwidth]{KennlinieNTCsubgitter.pdf}
					\caption{Messung zu Versuch 1d)}
					\label{KL d}
				\end{figure}
			
			\subsubsection*{Schlussfolgerung}
			
				Hier ist klar zu erkennen, dass der Widerstand mit steigender Temperatur (durch steigende Ströme) sinkt, was für NTC-Widerstände charakteristisch ist. 
				
				%TODO edit vgl. oben diodenerklärung
				Dass wir auch hier nicht mehr als bis zu \SI{8}{\V} Eingangsspannung messen konnten, wird ebenfalls aufgrund von Kurzschlussvermeidung gewesen sein.
			
		\subsection{e) Glimmlampe} 
			
			Bei der Glimmlampe kann der Strom nicht einfach durchfließen, da kein elektrischer Leiter die beiden Pole direkt miteinander verbindet.
			%TODO edit: technisch: anode -> kathode
			Damit ein Strom fließen kann, muss er von der Kathode zur Anode gelangen.
			Zwischen den beiden befindet sich lediglich ein Gas, welches als Leiter dienen soll.
			%TODO edit: same
			Dieses kann jedoch nur durch Gasentladungen den elektrischen Strom von der Kathode zur Anode führen.
			
			\subsubsection*{Gasentladungen}
			
				Die an der Kathode anliegende Spannung lässt freie Elektronen die Kathode verlassen.
				%TODO elektronen werden schneller beschleunigt -> dunkelraum kleiner -> dann mehr energie -> heller
				Je höher die anliegende Spannung, desto schneller werden die Elektronen beschleunigt und erreichen bei der Anode eine höhere Endgeschwindigkeit.
				Sind sie schnell genug um die Gasatome durch Stoßionisation anzuregen, führt dies zu einer Kettenreaktion von sich lösenden Elektronen.
				%TODO extra satz für neues thema
				Die Spannung, welche man für die Anregung braucht, nennt man Zündspannung. 
				
				An der Anode kommen schließlich aus den Gasatomen gelöste Elektronen an und ein Strom fließt, solange die Ionisation des Gases aufrecht erhalten wird.
				Sinkt die Spannung unter die sogenannte Löschspannung, dann ist dies nicht mehr möglich und die Entladung wird unterbrochen.
				%TODO edit: ionisation nicht mehr, aber anregung?
				Die Löschspannung ist geringer als die Zündspannung, da die Energie für das Starten der Ionisation nicht mehr benötigt wird, jedoch weitere beschleunigte Elektronen.
					
				%TODO edit: beim zurückfallen werden Ionen ausgesandt
				Zudem wird Licht bei der Ionisation des Gases emittiert, was das \glqq Glimmen\grqq{} der Lampe ausmacht.
			\subsubsection*{Messung}
			
				%TODO welche Spannung notwendig ist...
				Wir messen zunächst, wie viel Volt wir benötigen um die Glimmlampe zu zünden.
				%TODO Die Zündspannung haben wir auf \SI{119}{\V} bemessen.
				Diesen Wert haben wir bei ungefähr \SI{119}{\V} gefunden.
				Danach fiel die von uns gemessene Eingangsspannung instantan auf ca. \SI{88,4}{\V} ab und wir haben einen Strom gemessen.
				%TODO chronologisch eher danach, weil nebensächlichkeiten hinten
				Hier fing die Glimmlampe auch an zu leuchten.
				Bis zum Erreichen der Zündspannung war dieser gleich null, weswegen wir nur den relevanten Bereich in Abb. \ref{KL e} dargestellt haben.
				
				%TODO URTEIL(?) bei 4 messpunkten?!
				Wir haben zunächst versucht die Eingangsspannung zu erhöhen, was bis zu ca. \SI{92}{\V} möglich war. In diesem Bereich verlief das Strom-Spannungs-Verhältnis nahezu linear.
				
				%TODO das ist gar nicht im diagramm zu sehen, nur im laborbuch in der aufschriebreihenfolge
				%TODO style: hört sich holperig an
				Das Herunterregeln der Eingangsspannung ergab unerwartete Effekte, wie z. B. dass das Herunterregeln der Spannung am Netzgerät teilweise zu höheren Spannungen, nicht jedoch  geführt hat (siehe Abb. \ref{KL e}).
				%TODO einfach 0
				Der Strom hingegen wurde mit dem Herunterregeln kleiner bis wir schließlich eine Stromstärke von \SI{0}{mA} erreicht haben\footnote{siehe dazu Reihenfolge der Messwerte im Laborbuch, Seite 2}.
				%TODO add: Die Löschspannung konnte folglich nicht eindeutig festgelegt werden, da zwar bei \SI{87,7}{\V} die Lampe erlischt, es jedoch auch deutlich niedrigere Spannungen gab.
				
				Nachdem die Glimmlampe erlischt war, musste die Zündspannung ein weiteres Mal angelegt werden, um sie zu entzünden.
			
				\begin{figure}
					\centering
					\includegraphics[width=\textwidth]{KennlinieGlimmlampe.pdf}
					\caption{Messung zu Versuch 1e)}
					\label{KL e}
				\end{figure}
			
			\subsubsection*{Schlussfolgerung}
			
				Wir entnehmen unserer Messung, dass die Zündspannung der Glimmlampe bei ca. \SI{119}{\V} liegt.
				%TODO können wir nicht sagen, da die lampe eigentlich bei 87,7 ausgegangen ist
				Die Löschspannung liegt bei ungefähr \SI{86,5}{\V}, was unser geringster Wert für die Eingangsspannung war, bei dem wir einen Strom messen konnten.
				Für die Zündspannung wurde uns ein Wert von ungefähr \SI{120}{\V} vorgegeben, welcher mit unserer Messung weitgehend übereinstimmt, lediglich eine Differenz von \SI{1}{\V}.
				
				%TODO würde ich raus lassen, oder komplett mit dunkelraum etc. erklären
				Was genau die unerwarteten Effekte beim Einstellen der Eingangsspannung hervorrief ist unklar, wobei wir vermuten können, dass sich nach dem Finden der Zündspannung ein Widerstand bildete, welcher in Abhängigkeit von der Eingangsspannung einen Teil der Spannung blockierte, was auch den plötzlichen Abfall von \SI{119}{\V} auf \SI{88.4}{\V} erklären könnte. Sodass beim Herunterregeln der Eingangsspannung der Widerstand kleiner wurde und die gemessene Spannung folglich größer.  
			
	\section{Versuch 2: Widerstand in Abhängigkeit der Temperatur}		
		
		In diesem Versuch beschäftigen wir uns mit der Leitfähigkeit eines Kupferdrahtes in Abhängigkeit der Temperatur.
		
		\subsection{Methoden} %Aufbau und wie/was gemessen wird
		
			Wir schließen unseren Kupferdraht an eine Wheatstone'sche Brückenschaltung (siehe Abb. \ref{Schaltskizze2}). Hierbei bezeichnet $R_\text{x}(T)$ unseren Kupferdraht, $R_\text{v}$ einen einfachen \SI{5}{\Omega} Widerstand und $R_\text{e}$ den großen \SI{11,3}{\Omega} Widerstand lässt sich beliebig einstellen. Nimmt man die ganze Länge, so hat man einen Widerstand von \SI{11,3}{\Omega}, bei der Hälfte einen von \SI{5,65}{\Omega}. Hierbei handelt es sich nämlich um einen Widerstand über ein Metall. Diese Widerstände sind proportional zur Länge $L$ und antiproportional zur Querschnittsfläche $A$.
			
			Das Prinzip der Wheatstone'schen Brücke ermöglicht uns mit Hilfe der Kirchhoff'schen Regeln einen unbekannten Widerstand zu berechnen.
			Wird an dem Strommessgerät, wie es in Abb. \ref{Schaltskizze2} verbaut ist, kein Strom gemessen, dann gilt:
			
			\begin{equation*}
				R_\text{x}(T) + R_\text{v} =  R_\text{e}
			\end{equation*}
			Wir stellen unseren Widerstand $R_\text{e}$ also für jede Messung so ein, dass $I = 0$ erfüllt ist. Der Schalter dient für genaueres Einstellen von $I = 0$. Es wird erst über den \SI{20}{k\Omega} kalibriert und dann über den geschlossenem Schalter.
			
			Bevor wir die Messung beginnen, kühlen wir unseren Draht auf ca. \SI{0}{\celsius}, erhitzen ihn auf nahezu \SI{100}{\celsius} und lassen ihn wieder bis zu \SI{15}{\celsius} abkühlen.
			
			Währenddessen messen wir die Länge $L$ des Widerstands $R_\text{e}$, den wir für jede Messung anpassen. Da die Länge proportional zu dem Widerstand ist, tragen die Temperatur gegen die Position des Widerstandreglers auf.
						
			\begin{figure}
				\centering
				\includegraphics[width=0.6\textwidth]{Versuch2.png}
				\caption{Schaltskizze zu Versuch 1}
				\label{Schaltskizze2}
			\end{figure}
		
		\subsection{Messung}
			
			Abb. \ref{Ölbad} stellt die Position des Widerstandreglers in Abhängigkeit von der Temperatur. Die Position des Reglers ist hierbei mit der Länge des Widerstands  $R_\text{e}$ gleichzusetzen. Die Gesamtlänge läge bei \SI{100}{cm}, da wir hier jedoch nur Längen von \SIrange{50}{60}{cm} betrachtet haben, liegen die Werte für den Widerstand $R_\text{x}(T)$\footnote{aus der obigen Formel für $0.5\cdot R_\text{e}$ bis $0.6\cdot R_\text{e}$} zwischen \SI{0,65}{\Omega} und \SI{1,78}{\Omega}.
			
			In dem Diagramm erkennt man, dass der Widerstand des Kupferdrahtes sich mit steigender Temperatur nahezu linear erhöht und bei sinkender Temperatur ähnlich fällt.
			
				\begin{figure}
				\centering
				\includegraphics[width=\textwidth]{MessungDraht.pdf}
				\caption{Position des Widerstandreglers in Abhängigkeit von der Temperatur}
				\label{Ölbad}
			\end{figure}
			
		\subsection{Schlussfolgerung}	
			
			Wie auch schon bei der Glühlampe erkennen wir, dass die Eigenschaft eines Metallleiters, mit steigender Temperatur schlechter den elektrischen Strom zu leiten, von dem Kupferdraht erfüllt wird.
			
			Entgegen der Erwartungen, ist der Widerstand beim Abkühlen, bei gleichen Temperaturen, stets geringer als beim Erhitzen. %TODO Wieso?
		
	\begin{thebibliography}	
		e Abbildungen \ref{Schaltskizze1} und \ref{Schaltskizze2} wurden der Versuchsanleitung entnommen
	\end{thebibliography}	
			
\end{document} 