\section{Methoden}
	
	Dieser Abschnitt befasst sich mit dem Aufbau des Franck-Hertz-Versuches, so wie auch den dabei auftretenden Unsicherheiten.
	
	\subsection{Aufbau}
		
		Wie in Abbildung \ref{label} dargestellt, besteht der Aufbau des Franck-Hertz-Versuches im Wesentlichen aus einer Triode, bei der Elektronen von der Kathode ausgesandt und zur Anode geschickt werden.		
		Vor dem Gitter, welches zwischen Kathode und Anode liegt, werden die Elektronen mit der Beschleunigungsspannung	$U_B$ beschleunigt und dahinter mit der Gegenspannung $U_G$ abgebremst.
		
		Um den Druck in der Quecksilberröhre zu steigern, befindet sich der zugehörige Aufbau in einem Ofen.
		Bei dem ungeheizten Neon hingegen ist der Druck hoch genug, dass wiederholte Stöße von Elektronen mit den Atomen möglich sind.
		
		%TODO U_S?

	\subsection{Unsicherheiten} 

		%TODO

\section{Durchführung und Datenanalyse}

	%TODO

\section{Diskussion}

	%TODO