
\section{Kurzfassung}

Dieser Bericht beschäftigt sich mit der Untersuchung des Franck-Hertz-Versuches.
Der Versuch dient zur Beschreibung der Gesetzmäßigkeiten bei Elektronenstößen an Atomen.
Ziel dieser Arbeit ist es, den Befund von J. Franck und G. Hertz bezüglich dieser Gesetzmäßigkeiten zu unterstützen.
Dazu wird der Versuch mit zwei Franck-Hertz-Röhren, eine mit Neon und eine mit Quecksilber, nachgestellt.
Wie zu erwarten, hatten die in diesem Versuch aufgenommenen Kennlinien den charackteristischen Verlauf einer Franck-Hertz-Kurve.
Des Weiteren %TODO