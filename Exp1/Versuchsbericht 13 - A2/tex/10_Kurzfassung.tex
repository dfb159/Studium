\section{Kurzfassung}

Dieser Bericht beschäftigt sich mit der Untersuchung des Franck-Hertz-Versuches.
Er dient zur Beschreibung der Gesetzmäßigkeiten bei Elektronenstößen an Atomen.
Das Ziel dieser Arbeit ist es, den Befund von J. Franck und G. Hertz bezüglich dieser Gesetzmäßigkeiten zu unterstützen.
Dazu wird der Versuch mit zwei verschiedenen Franck-Hertz-Röhren nachgestellt.
Bei diesen handelt es sich um Trioden, welche mit Neon bzw. Quecksilber gefüllt sind.

Wie zu erwarten, hatten die in diesem Versuch aufgenommenen Kennlinien den charakteristischen Verlauf einer Franck-Hertz-Kurve.
Des Weiteren entsprechen die Ergebnisse in guter Näherung den Literaturwerten, abgesehen von den berechneten freien Weglängen, welche deutlich über den erwarteten Werten lagen.