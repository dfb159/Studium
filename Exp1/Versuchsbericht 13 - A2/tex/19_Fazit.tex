\section{Schlussfolgerung}
	
	Die aufgenommen, wie auch die ermittelten Werte entsprechen weitgehend den zu erwartenden Verhältnissen, wie sie von J.Franck und G.Hertz beschrieben wurden.	 
	Eine Wiederholung des Versuchs ist demnach nicht notwendig, wobei eine automatische Aufnahme der Werte über einen Computer bessere bzw. genauere Werte aufnehmen könnte als das manuelle Auftragen. 

	Lediglich die berechneten mittleren freien Weglängen liegen weit von den Erwartungen, was jedoch durch erneute Werteaufnahme ebenso nicht verbessert werden kann.
	Nach eingehender Fehlersuche und Validierung aller Referenzquellen kann die Rechnung wiederholt werden.
	Da die ermittelte Energie ein realistischer Wert ist, kann diese Fehlersuche auf das $\sigma$ sowie den Druck $p_\text{Raumtemperatur}$ bzw. $p_\text{Ofentemperatur}$ begrenzt werden.
	Genauere Fehlerquellen waren nicht auffindbar.
