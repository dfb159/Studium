\subsection{Unsicherheiten}\label{VGuD}

Jegliche Unsicherheiten werden nach GUM bestimmt und berechnet.
Die Gleichungen dazu finden sich in \ref{fig:GUM_combine} und \ref{fig:GUM_formula}.
Für die Unsicherheitsrechnungen wurde die Python Bibliothek "uncertainties" herangezogen, welche den Richtlinien des GUM folgt.
Alle konkreten Unsicherheitsformeln stehen weiter unten.
Für Unsicherheiten in graphischen Fits wurden die $y$-Unsicherheiten beachtet und die Methode der kleinsten Quadrate angewandt.
Dafür steht in der Bibliothek die Methode "scipy.optimize.curve\_fit()" zur Verfügung.

Für digitale Messungen wird eine Unsicherheit von $u(X) = \frac{\Delta X}{2\sqrt{3}}$ angenommen, bei analogen eine von $u(X) = \frac{\Delta X}{2\sqrt{6}}$.

\begin{description}
	\item[Winkelskala] Die Winkelskala ist in Halbgradschritte eingeteilt.
	Eine anliegende Noniusskala erhöht die Genauigkeit auf $\Delta \vartheta = \SI{1}{'}$. %TODO
	Für die Winkel bei den LED, wird auf Grund der großen Verschmierung $\Delta \vartheta_\text{LED} = \SI{30}{'}$ angenommen.
	
	\item[Spannung] Das digitale Voltmeter konnte Werte auf $\Delta U = \SI{0.01}{\volt}$ angeben.
	
\end{description}
\begin{figure}[ht]
	\begin{equation*}	
		x = \sum_{i=1}^{N} x_i
		;\quad
		u(x) = \sqrt{\sum_{i = 1}^{N} u(x_i)^2}
	\end{equation*}
	\caption{Formel für kombinierte Unsicherheiten des selben Typs nach GUM.}
	\label{fig:GUM_combine}
\end{figure}

\begin{figure}[ht]
	\begin{align*}
		f = f(x_1, \dots , x_N)
		;\quad
		u(f) = \sqrt{\sum_{i = 1}^{N}\left(\pdv{f}{x_i} u(x_i)\right) ^2}
	\end{align*}
	\caption{Formel für sich fortpflanzende Unsicherheiten nach GUM.}
	\label{fig:GUM_formula}
\end{figure}

\begin{figure}[ht]
	\begin{align*}
		\lambda = \frac{g}{m} \sin \vartheta_m
		;\quad
		u(\lambda) = \lambda \cot \vartheta_m u(\vartheta_m)
	\end{align*}
	\caption{Unsicherheitsformel für die Wellenlänge bei einem optischen Gitter mit Gitterkonstante $g$ und Ordnung $m$. $g$ wird dabei als genau angenommen.}
	\label{unc:lambda}
\end{figure}

\begin{figure}[ht]
\begin{align*}
	\frac{1}{\lambda} = \frac{m}{g} \frac{1}{\sin \vartheta_m}
	;\quad
	u(\frac{1}{\lambda}) = \frac{1}{\lambda} \cot \vartheta_m u(\vartheta_m)
\end{align*}
\caption{Unsicherheitsformel für den Kehrwert der Wellenlänge bei einem optischen Gitter mit Gitterkonstante $g$ und Ordnung $m$.}
\label{unc:1-lambda}
\end{figure}
