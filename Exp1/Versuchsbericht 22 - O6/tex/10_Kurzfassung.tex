\section{Kurzfassung}

	Dieser Bericht beschäftigt sich mit der Untersuchung optischer Abbildungen mit Hilfe einer digitalen Kamera.
	Dazu werden mit verschiedenen Objektiven und Blenden Fotos von denselben Schirmen aufgenommen und ausgewertet.
	Die hier zu beobachtenden Größen sind die Schärfentiefe, welche theoretisch und subjektiv empfunden bestimmt werden soll, wie auch das Auflösungsvermögen bei den verschiedenen Blenden.
	Für die Auswertung wurde ein Plugin für das Bildanalyseprogramm "ImageJ" von der Universität bereitgestellt.
	Mit diesem lassen sich die Fotos anhand ihres Kontrastes bezüglich ihrer Schärfe auswerten.
	
	Ziel der Untersuchung ist eine Übereinstimmung von theoretischer und subjektiv empfundener Schärfentiefe bezüglich der Vorgabe, dass Bildpunkte bis zu einer Größe von 1/1500 der Bilddiagonalen scharf wahrgenommen werden.
	Zudem soll gezeigt werden, inwiefern Blenden das Bild verbessern und wie sich die Belichtungszeit variabler Blendengröße verhält.
	
	Eine Übereinstimmung zwischen den berechneten Schärfentiefen ließ sich nicht finden.
	Die ersten Werte liegen noch innerhalb einer Unsicherheit voneinander, für größere Blendenzahlen $k$ jedoch wichen die Werte stark voneinander ab.
	Aufgrund der großen Unsicherheiten können diese Abweichungen jedoch nicht als Widerspruch für $Z = D_\text{B}/1500$ gesehen werden.
	
	Des Weiteren ließ sich für das Auflösungsvermögen bestimmen, dass die Blendzahlen $k = 5,6$ und $k = 8$ die kleinsten Halbwertsbreiten und somit die beste Auflösung besitzen; Blenden die Auflösung also verbessern können, indem sie Linsenfehler, welche am Rand auftreten, ausbessern.
	Unter den Blenden war dies insbesondere bei der \SI{3,75}{\milli\meter}-Blende der Fall.
	Neben der MTF-Methode brachte auch die Auswertung über Siemenssterne das gleiche Ergebnis.
	
	Für die zunächst unbestimmte Größe der Lochblende ließ sich durch die Belichtungszeiten eine Approximation auf \SI{0,235+-0,035}{\milli\meter} durchführen. 
