\section{Schlussfolgerung}
	
	Bis auf die Abweichungen bei Schärfentiefen $S_\text{(sub)}$ bei höheren Blendenzahlen $k$ und dem Lochblenden-Foto ließen sich die restlichen Ziele, zu zeigen dass Blenden einen Einfluss auf das Auflösungsvermögen besitzen, wie auch den Zusammenhang zwischen Blendengröße und Belichtungszeit, erreichen:

	Eine Übereinstimmung zwischen den berechneten Schärfentiefen ließ sich nicht finden.
	Die ersten Werte liegen noch innerhalb einer Unsicherheit voneinander, für größere Blendenzahlen $k$ jedoch wichen die Werte stark voneinander ab.
	Aufgrund der großen Unsicherheiten können diese Abweichungen jedoch nicht als Widerspruch für $Z = D_\text{B}/1500$ gesehen werden.
	
	Des Weiteren ließ sich für das Auflösungsvermögen bestimmen, dass die Blendzahlen $k = 5,6$ und $k = 8$ die kleinsten Halbwertsbreiten und somit die beste Auflösung besitzen; Blenden die Auflösung also verbessern können, indem sie Linsenfehler, welche am Rand auftreten, ausbessern.
	Unter den Blenden war dies insbesondere bei der \SI{3,75}{\milli\meter}-Blende der Fall.
	Neben der MTF-Methode brachte auch die Auswertung über Siemenssterne das gleiche Ergebnis.
	
	Für die zunächst unbestimmte Größe der Lochblende ließ sich durch die Belichtungszeiten eine Approximation auf \SI{0,235+-0,035}{\milli\meter} durchführen. 
	
	Eine Wiederholung des Versuches als Ganzes scheint aufgrund der Ergebnisse nicht sinnvoll, jedoch bietet es sich an aufgrund des Fehlens des Auflösungsvermögens der Lochblende zumindest an dieser Stelle ein neues Foto unter besseren Randbedingungen (Beleuchtung, Schirmabstand etc.) aufzunehmen.
	