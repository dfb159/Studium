\section{Schlussfolgerung}
	
	Ließen sich die Behauptungen von Studentengruppe A bestätigen?
	Die Ergebnisse der Untersuchung stützen diese nur teilweise.
	
	Dass die Na $D$-Linie sich nicht aufspalten lässt, ist nach den Ergebnissen dieser Untersuchung nur für das Prisma der Fall.
	Bei den Gittern hingegen war eine Aufspaltung ab dem zweiten Beugungsmaximum erkennbar.
	Die Behauptung, dass es sich bei der Füllung der Energiesparlampe um Quecksilber handelt, ließ sich insbesondere durch die Wellenlängen der Spektrallinien von \SI{405.81+-0.13}{\nano\meter} und \SI{547.08+-0.16}{\nano\meter} verglichen mit den Literaturwerten \SI{404.656}{\nano\meter} und \SI{546.074}{\nano\meter}\cite{NISTHg} stützen.
	Wohingegen die Behauptung bezüglich der Leuchtdioden, dass es sich nur um einzelne Wellenlängen handelt, als falsch angenommen werden kann, da hier nicht einzelne Spektrallinien sondern Bänder aus unzähligen Wellenlängen beobachtet wurden. 
	Aus den Maxima ließ sich dennoch $hc$ berechnen.
	Dieser Wert belief sich auf $hc = \SI{5,7+-1,5e-7}{\electronvolt\meter}$ und stimmt damit nicht mit dem Produkt aus planck'schem Wirkungsquantum $h$ und Lichtgeschwindigkeit $c$ überein, da diese sich ungefähr um einen Faktor von zwei unterscheiden.
	
	Die Behauptungen von Studentengruppe A ließen sich als nur bezüglich der Energiesparlampe bestätigen und für das Produkt $hc$ lässt sich keine genaue Aussage treffen.
	