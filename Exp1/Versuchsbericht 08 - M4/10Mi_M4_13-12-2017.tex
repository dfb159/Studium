\documentclass[11pt,a4paper,titlepage, ngerman]{article}

\usepackage[utf8]{inputenc}	
\usepackage[T1]{fontenc}	
\usepackage{ngerman}			
\usepackage{lmodern}			
\usepackage{graphicx}			
\usepackage{url}				
\usepackage{siunitx}
\usepackage[intlimits]{amsmath}
\usepackage{xfrac}
\usepackage{commath}
\usepackage{physics}			
\usepackage{subcaption}
\usepackage{wrapfig}
\usepackage{biblatex}
\usepackage{hyperref}

% Setup SI unit environment
\sisetup{separate-uncertainty = true}
\sisetup{output-decimal-marker = {,}}
\sisetup{
	per-mode=fraction,
	fraction-function=\sfrac
	% or \frac, \tfrac
}
\bibliography{Literatur}
\begin{document}
	\begin{titlepage}
		\centering
		{\scshape\LARGE Versuchsbericht zu \par}
		\vspace{1cm}
		{\scshape\huge M4 -- Stoßgesetze\par}
		\vspace{2.5cm}
		{\LARGE Gruppe 10 Mi\par}
		\vspace{0.5cm}
		{\large Alex Oster (E-Mail: a\_oste16@uni--muenster.de) \par}
		{\large Jonathan Sigrist (E-Mail: j\_sigr01@uni--muenster.de) \par}
		\vfill
		durchgeführt am 13.12.2017\par
		betreut von\par
		{\large Semir \textsc{Vrana}} 
		\vfill	
		{\large \today\par}
	\end{titlepage}
	
	\tableofcontents
	
	\newpage
	
	\section{Kurzfassung}
	
		Dieser Bericht befasst sich mit den Stoßgesetzen. Dazu werden zwei Versuche betrachtet, die Übereinstimmungen zwischen den aufgenommenen Werten und den durch die Stoßgesetze ermittelten Werte zeigen sollen.   
		
		Bei dem ersten Versuch wird ein ballistischer zentraler Stoß zweier Metallkugeln betrachtet. Dazu werden zwei solcher Metallkugeln unterschiedlicher Masse an Pendeln aufgehängt. Der Stoßvorgang wird durch Auslenkung eines der Pendel in Gang gesetzt und dann wird die Auslenkung der gestoßenen Kugel gemessen. Ziel dieses Versuches ist, dass das Messergebnis für das Massenverhältnis mit dem bestimmten Wert dafür übereinstimmt. Diese Übereinstimmung wird durch die Ergebnisse gezeigt.
		
		Der zweite Versuch stellt den Zusammenhang zwischen der Höhenenergie einer kleinen Metallkugel und der Auslenkung eines Pendels nach einem Stoß her mit einer größeren an dem Pendel hängenden Metallkugel.
		Die kleinere Kugel wird dabei in einer Fallrinne an verschiedenen Positionen losgelassen und der sich bei dem Stoß ergebene Energieübertrag  untersucht.
		Es wird der theoretische Anteil der kinetischen Energie mit $\varepsilon = \frac{5}{9}$ der Gesamtenergie im System überprüft und begründet, weshalb die Rollreibung des Systems diesen Anteil verfälscht.
			
	\vspace{2cm} 
	
	\section{Stoßprozese zwischen zwei Kugeln unterschiedlicher Massen} 
	\newpage
	\section{Stoßprozesse auf einer Rutschbahn} 
	
	%\section*{Literatur}
	%\printbibliography
	
	\newpage
	
	\section{Anhang} 
		
	\begin{figure}[ht]
		\centering
		\includegraphics[width=\textwidth]{M4_1.jpg}
		\caption{Versuchsaufbau$^{[1]}$}
		\label{abb:VersuchsaufbauStoss}	
	\end{figure}

	\begin{equation}
		\label{eq:unce_auslenkung2}
		u(a^2) = \pdv{a^2}{a} \cdot u(a) = 2 a \cdot u(a)
	\end{equation}

	\begin{align}
		\label{eq:epsilon}
		\varepsilon &= -\frac{m}{2 l} \left( \frac{m_1 + m_2}{2 m_1}\right) ^2 \sqrt{1+\frac{L^2}{H^2}}\\
		\pdv{\varepsilon}{m} &= \frac{\varepsilon}{m}\\
		\pdv{\varepsilon}{l} &= -\frac{\varepsilon}{l}\\
		\pdv{\varepsilon}{m_1} &= +\frac{m}{4 l} \sqrt{1+\frac{L^2}{H^2}} \cdot \frac{m_1 + m_2}{m_1} \cdot \frac{m_2}{m_1^2}\\
		\pdv{\varepsilon}{m_2} &= -\frac{m}{4 l} \sqrt{1+\frac{L^2}{H^2}} \frac{m_1 + m_2}{m_1} \frac{1}{m_1}\\
		\pdv{\varepsilon}{L} &= -\frac{m}{2 l} \left( \frac{m_1 + m_2}{2 m_1}\right) ^2 \frac{L}{H^2}\frac{1}{\sqrt{1+\frac{L^2}{H^2}}}\\
		\pdv{\varepsilon}{H} &= +\frac{m}{2 l} \left( \frac{m_1 + m_2}{2 m_1}\right) ^2 \frac{L^2}{H^3}\frac{1}{\sqrt{1+\frac{L^2}{H^2}}}
		\label{eq:epsilonEnd}
	\end{align}
	
	\section*{Literatur}
	
	[1] Abb. \ref{abb:VersuchsaufbauStoss} stammt aus der Vorbereitung zu M4, welche im Learnweb zu finden ist.
	\newline
	[2] Abb. \ref{fig:fallrinneSkizze} stammt aus der Vorbereitung zu M4, welche im Learnweb zu finden ist.
	
\end{document} 