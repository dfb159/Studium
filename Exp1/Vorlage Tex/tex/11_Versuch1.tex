\section{Versuch1} 

Dies ist ein Teilversuch.
Der Abschnitt sagt grundsätzliches über den konkreten Versuch aus.
Es werden relevante physikalische Effekte qualitativ beleuchtet.
Es wird das Ziel des Versuchs angegeben.

Die Ergebnisse werden kurz, aber mit konkreten Werten erwähnt.
Dann werden diese in den Kontext kurz eingebunden.

\subsection{Methoden}

\subsubsection{Aufbau}

Beschreibung des Aufbaus des Versuches.
Gegebenenfalls weitere Beschreibung einzelner Komponenten.
Falls Abweichung von der Versuchsvorschrift Begründung.
Bei Verwechlungsgefahr kann  auf diese noch einmal genauer darauf eingegangen werden.

Beschreibung der Durchführung.
In welcher Reihenfolge wurde wie was gemacht.
Kurz halten, denn ist ja schon in Anleitung drin.
Bei schwierigen Stellen kann auch hier ein kurzer Vermerk eingebracht werden.

\subsubsection{Unsicherheiten}

Alle primären Unsicherheiten der im Versuch verwendeten Messinstrumente.
Dazu eine Begründung zu jedem (Rechteck, Dreieck, Gauß; z. B. Waagegenauigkeit, Reaktionszeit).
Grundsätzliche Fehlerfortpflanzung nach GUM (kombinierte und fortpflanzende Fehler).

\subsubsection{Komplikationen}

Falls erwähnungsbedürftige oder den Versuchsablauf behindernde Sachverhalte aufgetreten sind, kann man das hier der Welt mitteilen.
Darauf kann später im Fazit noch einmal eingegangen werden.

\subsection{Datenanalyse}

Alle Messwerte und Daten (Laborbuch und weitere Datenfiles) anbeilegen und ggf. darauf Bezug nehmen.
Alle primären Messwerte, soweit nicht völlig sinnfrei, in graphischer Form einfügen.
Rechnungsweg zum Ergebnis angeben und begründen, warum z. B. linearer Fit genommen wurde.
Dazu auf die Theorie oder die Versuchsanleitung Bezug nehmen.
Vollständige Fehlerberechnung in den Anhang.
Da grundsätzlich schon vorher erläutert (GUM) reicht hier eine kurze Anmerkung auf die Fehlerrechnung.
Ergebnis graphisch darstellen und auf passende Darstellungsweise (Unsicherheiten, signifikante Stellen) achten.

Falls mehrere voneinander unabhängige Lösungen existieren (z. B. bei der geometrischen Bestimmung des Trägheitsmomentes des Kreisels), können diese in Unterkapitel gegliedert werden.

\subsection{Diskussion}

Die Ergebnisse in den Kontext einbinden.
Zusammenhänge noch einmal darstellen.
Jegliche Aussagen durch Ergebnisse oder Vorwissen untermauern.
Ergebnisse mit Referenzwerten/Literaturwerten vergleichen und auf Unsicherheiten eingehen (Vertrauensgrad).

\subsection{Schlussfolgerung}

Das Ziel noch einmal erläutern.
Fazit des Versuchs angeben (untermauert die These; weicht von den Literaturwerten ab; ...).
Warum folgt aus den Ergebnissen das Fazit.
Mit ermittelten Daten und Unsicherheiten begründen.
Auf Unsicherheiten weiter eingehen und Vertrauensgrad bzw. Relevanz der Schlussfolgerung auf die derzeitige Wissenslage angeben (meistens nicht relevant).
Verbesserungsvorschläge für das nächste Mal und allgemeine Rekapitulierung über das durchgeführte Experiment.
Warum muss man das Experiment unbedingt noch mal durchführen.
