\documentclass[11pt,a4paper,titlepage, ngerman]{article}
\usepackage[utf8]{inputenc}	% Diese Pakete sind
\usepackage[T1]{fontenc}		% für die Verwendung 
\usepackage{ngerman}			% von Umlauten im tex-file
\usepackage{lmodern}			% Schriftart, die am Bildschirm besser lesbar ist
\usepackage{graphicx}			% Zum Einbinden von Formeln
\usepackage{url}					% Zur Darstellung von Webadressen
\usepackage{siunitx}

\begin{document}
	
\begin{titlepage}
	\centering
	{\scshape\LARGE Versuchsbericht zu \par}
	\vspace{1cm}
	{\scshape\huge S1 -- Was ist Experimentieren?\par}
	\vspace{2.5cm}
	{\LARGE Gruppe 10 Mi\par}
	\vspace{0.5cm}
	{\large Alex Oster (E-Mail: a\_oste16@uni--muenster.de) \par}
	{\large Jonathan Sigrist (E-Mail: j\_sigr01@uni--muenster.de ) \par}
	\vfill
	durchgeführt am 18.10.2017\par
	betreut von\par
	{\large Dr. Anke \textsc{Schmidt}}

	\vfill

	{\large \today\par}
\end{titlepage}


\tableofcontents

\newpage

\section{Kurzfassung}

	In diesem Bericht, zur ersten experimentellen Übung, beschäftigen wir uns mit der Frage, was genau man unter dem Begriff \glqq Experimentieren\grqq {} verstehen sollte. Dazu betrachten wir zunächst folgende Fragen: \\
	\begin{enumerate}
		 \item Was ist mit \glqq Messgröße\grqq {} gemeint? \\
		 \item Warum führt man Experimente in der Naturwissenschaft durch? und\\
		 \item Weshalb kann der \glqq wahre Wert\grqq {} einer Messgröße niemals bestimmt werden?
	\end{enumerate}
	Zur Beantwortung dieser Fragen, wenden wir uns nun drei einfachen Versuchen zu. In dem ersten Versuch haben wir die Leerlaufspannung einer 9V-Batterie gemessen, in dem zweiten Versuch die Länge eines \glqq STABILO point 88\grqq {} Stiftes und in dem dritten Versuch dann die Zeit, die Kugeln verschiedener Masse zum Herunterrollen einer schiefen Ebene benötigen. \\
	Die Auswertungen dieser Versuche werden wir dann in \textbf{3 Diskussion} mit den obigen Fragen verknüpfen und damit den Begriff \glqq Experimentieren\grqq {} erklären.

\section{Durchführung}

	\subsection{Versuch 1: Leerlaufspannung}
	
		Bevor wir die Messung durchgeführt haben, haben wir uns gefragt, welche Werte für die Leerlaufspannung $ U_\textsc{0} $ zu erwarten sind. \\
		Da es sich um eine \SI{9}{\V}-Batterie handelte, und da keine negativen Werte für $ U_\textsc{0} $ vorliegen können, haben wir darauf geschlossen, dass der Wert für die Leerlaufspannung sich mindestens im Bereich zwischen \SI{0}{\V} und \SI{9}{\V} befinden sollte. \\
		Das, am Boden der Batterie, gegebene Mindesthaltbarkeitsdatum (\glqq 2020\grqq) wurde noch nicht überschritten. Bis dahin sollten die \SI{9}{\V} garantiert sein, also haben wir gefolgert, dass die Batterie, im unbenutzten Falle, auch mehr als \SI{9}{\V} Leerlaufspannung besitzen könnte. Deswegen wir unseren Erwartungsbereich von \SIrange{0}{9}{\V} auf \SIrange{0}{10}{\V}. %Hier Verweis auf Diagramm 1 (Hypothese)
		\\ 
		Die Messung der Leerlaufspannung $ U_\textsc{0} $, mit Hilfe eines Multimeters, ergab einen Wert von \SI{9,46}{\V} bzw. \SI{9,47}{\V}. Da das Messgerät rundet betrachten wir hierbei Werte von \SIrange{9,455}{9,475}{\V}. Zudem besitzt das Messgerät eine Ungenauigkeit von $0.5\%$ des angegebenen Wertes, weswegen der eigentliche Wert im Bereich von \SIrange{9,407}{9,522}{\V} ist. \\
		
	\begin{flushleft}
			Somit ergibt sich: \\
		\vspace{0.5cm}
		$ U_\textsc{0} \in [\SI{9.407}{\V},\SI{9.522}{\V}]$\\
		$a = \SI{0.115}{\V}$ \\
		$\sigma = \SI{0,0332}{\V}$ \\
		\vspace{0.5cm}
		wir erhalten einen zu erwartenden Wert von: \\ 
		\vspace{0.5cm}
		$U_\textsc{0} = (9,4645\pm 0,0332)\si{\V}$  %Hier Verweis auf Diagramm 2 (Messung)
	\end{flushleft}
	
	\subsection{Versuch 2: Längenmessung}
		
		Wie auch im ersten Versuch haben wir uns zunächst gefragt, welche Werte für die Länge des Stiftes in Frage kämen. So haben wir die Stiftlänge über die Spannbreite von Daumen und Zeigefinger in ein Intervall von \SIrange{15}{20}{\cm} abgeschätzt. \\ %Hier Verweis auf Diagramm 3 (Schätzung)
		Die Messung an einem Maßband ergab eine Länge von ca. \SI{16,6}{\cm} von beiden Seiten. Die Unsicherheit des Maßbandes wurde hierbei nicht betrachtet, da kein realistischer Wert dafür gegeben war (\SI{6}{\cm} Ungenauigkeit auf \SI{2}{\m}). %Hier Verweis auf Diagramm 4 (Messung) 
		
	\subsection{Versuch 3: Schiefe Ebene}
		
		Für den dritten Versuch haben wir die Hypothese aufgestellt, dass schwerere Kugeln schneller die schiefe Ebene herunterrollen als leichtere Kugeln gleichen Volumens, da das uns bekannte Trägheitsmoment ebenfalls von der Masse des Körpers abhängt. \\
		Um unsere Hypothese zu verifizieren bzw. falsifizieren, haben wir eine Holzkugel und eine Metallkugel mit gleichem Radius mehrfach eine schiefe Ebene herunterrollen lassen und dabei die Zeit gemessen. \\
		Diese Messung haben wir 15 mal pro Kugel durchgeführt, sodass der Mittelwert über alle Messungen sich mit jeder neuen Messung nicht groß geändert hat und der direkte Vergleich beider Mittelwerte auf ein Ergebnis schließen lässt. \\ %Verweis auf Messwerte-Tabelle/Diagramm
		Trotz 15 Messungen hat sich der Mittelwert für die Holzkugel sich noch um ca. \SI{0.02}{s} vom vorherigen Mittelwert unterschieden. Dies könnte an ungenauer Zeitmessung gelegen haben, weswegen wir eventuell noch mehr Messungen durchgeführt haben sollten \\
		Zudem haben wir jedoch beim zeitgleichen Herunterrollen beider Kugeln festgestellt, dass wenn die Holzkugel vorne ist, immer beide Kugeln gleichzeitig unten ankommen.Ist die Metallkugel vorne, so kommt sie eher als die Holzkugel an.
		

\section{Diskussion}
	
	Soon\texttrademark

\begin{thebibliography}{}

\end{thebibliography}


\end{document} 